\usepackage{xspace}
\newcommand{\addarticle}[1]{%
\ifx#1aan\else\ifx#1Aan\else
\ifx#1ean\else\ifx#1Ean\else
\ifx#1ian\else\ifx#1Ian\else
\ifx#1oan\else\ifx#1Oan\else
\ifx#1ian\else\ifx#1Uan\else
a\fi\fi\fi\fi\fi
\fi\fi\fi\fi\fi\xspace#1}
%\newcommand{\addarticle}[1]{a #1}
%cat primitives: checks
\newcommand{\notthecase}[1]{it is not the case that #1}
\newcommand{\Variant}[1]{#1 is implemented}
\newcommand{\NotVariant}[1]{it is not the case that #1 is implemented}
\newcommand{\flag}[1]{By construction, #1}
\newcommand{\assert}[1]{By construction, #1}
% Non-satisfying empty set and relation...
%\newcommand{\noevent}[1]{#1 does not exist)
%\newcommand{\norel}[2]{neither #1 nor #2 exist)
\newcommand{\anyevent}[1]{#1 is any event}
\newcommand{\anyrel}[2]{#1 and #2 are any events}
%cat primitives: relational algebra
\newcommand{\relation}[3]{there is \addarticle#1 from #2 to #3}
\newcommand{\transitive}[3]{there exists a chain of #1 from #2 to #3}
% For relations that follow the pattern "there is <rel> from <E> to <E>"
\newcommand{\newrelation}[2]{%
\expandafter\def\csname #1name\endcsname{#2}%
\expandafter\def\csname #1\endcsname##1##2{there is \addarticle#2 from ##1 to ##2}}
% For relations that follow the pattern "<E> is <rel> <E>"
\newcommand{\newisrelation}[2]{%
\expandafter\def\csname #1name\endcsname{#2}%
\expandafter\def\csname #1\endcsname##1##2{##1 is #2 ##2}}
\newcommand{\newisrelationrev}[2]{%
\expandafter\def\csname #1name\endcsname{#2}%
\expandafter\def\csname #1\endcsname##1##2{##2 is #2 ##1}}
\newcommand{\intervening}[4]{there exists a #1 in #2 between #3 and #4}
%
%cat primitives: relations definitions
\newisrelation{included}{included in}
\newcommand{\sameloc}[2]{#1 and #2 are to the Same Location}
\edef\loc#1#2{\sameloc{#1}{#2}}
\newcommand{\sca}[2]{#1 belongs to the same single-copy-atomic class as #2}
\newcommand{\po}[2]{#1 appears in program-order before #2}
%\edef\poloc#1#2{\po{#1}{#2} and \sameloc{#1}{#2}}
\newisrelation{co}{Coherence-write-before}
\newcommand{\fr}[2]{#1 Reads-before #2}
\newcommand{\rf}[2]{#2 Reads-from-memory #1}
\newcommand{\rfi}[2]{#2 Reads-from-internal #1}
\newcommand{\rfreg}[2]{#2 Reads-from-register #1}
\newcommand{\rmw}[2]{#1 and #2 form a successful Read-Modify-Write pair}
\newcommand{\DATA}[2]{#1 affects the data value written by #2}
\newcommand{\ADDR}[2]{#1 affects the address of the Location accessed by #2}
\newcommand{\sameinstance}[2]{#1 and #2 are generated by the same instruction}
\newcommand{\sm}[2]{\sameinstance{#1}{#2}}
\newcommand{\si}[2]{\sameinstance{#1}{#2}}
\newcommand{\sameEffect}[2]{#1 and #2 are the same Effect}
\newcommand{\ext}[2]{#1 and #2 are from different Processing Elements}

%cat primitives: events
\newcommand{\ME}{Memory Effect}
\newcommand{\MWE}{Memory Write Effect}
\newcommand{\MRE}{Memory Read Effect}
\newcommand{\M}[1]{#1 is a \ME{}}
\newcommand{\W}[1]{#1 is a \MWE{}}
\newcommand{\R}[1]{#1 is a \MRE{}}
\renewcommand{\_}{any Effect}
\newcommand{\B}[1]{#1 is a Branching Effect}
\newcommand{\BCC}[1]{#1 is a Conditional Branching Effect}
\newcommand{\FAULT}[1]{#1 is a Fault Effect}
\newcommand{\NoRet}[1]{#1 is an Effect with no return value}

\newcommand{\RWE}{Register Write Effect}
\newcommand{\RRE}{Register Read Effect}
\newcommand{\RRWEs}{Register Read and Write Effects}

\newcommand{\Wreg}[1]{#1 is a \RWE{}}
\newcommand{\Rreg}[1]{#1 is a \RRE{}}

\newcommand{\RREof}[1]{the \RRE{} of #1}
\newcommand{\RWEof}[1]{the \RWE{} of #1}

%aarch64 notions: locations and registers
\newcommand{\memloc}[1]{the Memory Location #1}
\newcommand{\tagloc}[1]{the Tag Location #1}
\newcommand{\reg}[1]{the Register #1}

%aarch64 notions: effects
%[Explicit|Implicit] [Instruction|Tag|TTD] [Read|Write] [Memory|Register] Effect

\newcommand{\Exp}[1]{#1 is an Explicit Effect}
\newcommand{\ExpMREof}[1]{the Explicit \MRE{} of #1}
\newcommand{\ExpMWEof}[1]{the Explicit \MWE{} of #1}
\newcommand{\ExpM}[1]{#1 is an Explicit \ME}
%\newcommand{\MExp}[1]{\ExpM{#1}}
\newcommand{\ExpW}[1]{#1 is an Explicit \MWE}
\newcommand{\ExpR}[1]{#1 is an Explicit \MRE}

\newcommand{\NExp}[1]{#1 is an Implicit Effect}
\newcommand{\Imp}[1]{\NExp{#1}}
\newcommand{\ImpM}[1]{#1 is an Implicit Memory Effect}
\newcommand{\ImpW}[1]{#1 is an Implicit Memory Write Effect}
\newcommand{\ImpR}[1]{#1 is an Implicit Memory Read Effect}

\newcommand{\Tag}[1]{#1 is a Tag Effect}
\newcommand{\TagCheck}[1]{#1 is a TagCheck Effect}
\newcommand{\ExpTagMRE}{Explicit Tag \MRE{}}
\newcommand{\ExpTagMWE}{Explicit Tag \MWE{}}
\newcommand{\ImpTagMRE}{Implicit Tag \MRE{}}
\newcommand{\ImpTagMWE}{Implicit Tag \MWE{}}
\newcommand{\ImpTagMREof}[1]{the \ImpTagMRE{} of #1}
\newcommand{\ImpTagMWEof}[1]{the \ImpTagMWE{} of #1}
\newcommand{\ImpTagM}[1]{#1 is an Implicit Tag Effect}
\newcommand{\ImpTagW}[1]{#1 is an Implicit Tag Write Effect}
\newcommand{\ImpTagR}[1]{#1 is an Implicit Tag Read Effect}

\newcommand{\ImpTTDM}[1]{#1 is an Implicit TTD Effect}
\newcommand{\ImpTTDW}[1]{#1 is an Implicit TTD Write Effect}
\newcommand{\HU}[1]{#1 is a Hardware Update Effect}
\newcommand{\ImpTTDR}[1]{#1 is an Implicit TTD Read Effect}

\newcommand{\ImpInstrM}[1]{#1 is an Implicit Instruction Effect}
\newcommand{\ImpInstrW}[1]{#1 is an Implicit Instruction Write Effect}
\newcommand{\ImpInstrR}[1]{#1 is an Implicit Instruction Read Effect}

\newcommand{\ISB}[1]{#1 is an ISB Effect}
\newcommand{\EXCENTRY}[1]{#1 is an Exception Entry Effect}
\newcommand{\EXCRET}[1]{#1 is an Exception Return Effect}
\newcommand{\EXCENTRYCSE}[1]{#1 is an Exception Entry Context Synchronisation Effect}
\newcommand{\EXCRETCSE}[1]{#1 is an Exception Return Context Synchronisation Effect}
\newcommand{\CSE}[1]{#1 is a Context Synchronisation Effect}

\newcommand{\A}[1]{#1 is generated by an instruction with Acquire semantics}
\newcommand{\Q}[1]{#1 is generated by an instruction with AcquirePC semantics}
\newcommand{\REL}[1]{#1 is generated by an instruction with Release semantics}

\newcommand{\DMBFULL}[1]{#1 is a DMB FULL Effect}
\newcommand{\DMBSY}[1]{#1 is a DMB SY Effect}
\newcommand{\DMBST}[1]{#1 is a DMB ST Effect}
\newcommand{\DMBLD}[1]{#1 is a DMB LD Effect}
\newcommand{\DSBFULL}[1]{#1 is a DSB FULL Effect}
\newcommand{\DSBSY}[1]{#1 is a DSB SY Effect}
\newcommand{\DSBST}[1]{#1 is a DSB ST Effect}
\newcommand{\DSBLD}[1]{#1 is a DSB LD Effect}
\newcommand{\TLBI}[1]{#1 is a TLBI Effect}
\newcommand{\CTLBI}[1]{#1 is a Completed TLBI Effect}
\newcommand{\invscope}[2]{#1 is in the Invalidation Scope of #2}
\newcommand{\DCCVAU}[1]{#1 is a DC CVAU Effect}
\newcommand{\IC}[1]{#1 is an IC Effect}
\newcommand{\ICIVAU}[1]{#1 is an IC IVAU Effect}

\newcommand{\Fault}[1]{#1 is a Fault Effect}
\newcommand{\TagCheckFAULT}[1]{#1 is a TagCheck Fault Effect}
\newcommand{\MMU}[1]{#1 is an MMU Effect}
\newcommand{\MMUFAULT}[1]{#1 is an MMU Fault Effect}
\newcommand{\TLBUncacheableFAULT}[1]{#1 is a TLBUncacheable Fault Effect}

%aarch64 notions: relations
\newcommand{\amo}{an atomic operation}
\newcommand{\lxsx}{a successful Load-Exclusive/Store-Exclusive pair}

%aarch64deps.cat
\newrelation{iicodata}{Intrinsic Data Dependency}
\newrelation{iicoorder}{Intrinsic Order Dependency}
\newrelation{iicoctrl}{Intrinsic Control Dependency}

\newrelation{dtrm}{Dependency through Registers and Memory}
\newrelation{basicdep}{Basic Dependency}
\newrelation{data}{Data Dependency}
\newrelation{addr}{Address Dependency}
\newrelation{ctrl}{Control Dependency}


\newrelation{pickdtrm}{Pick Dependency through Registers and Memory}
\newrelation{pickbasicdep}{Pick Basic Dependency}
\newrelation{pickdatadep}{Pick Data Dependency}
\newrelation{pickaddrdep}{Pick Address Dependency}
\newrelation{pickctrldep}{Pick Control Dependency}
\newrelation{pickdep}{Pick Dependency}

%aarch64hwreqs.cat
\newisrelation{tcib}{Tag-check-intrinsically-before}
\newisrelation{trib}{Translation-intrinsically-before}
\newisrelation{fib}{Fetch-intrinsically-before}

\newcommand{\sameoa}[2]{#1 and #2 have the Same Output Address}
\newcommand{\sameloworderbits}[2]{#1 and #2 have the Same Low Order Bits}
\newcommand{\valoc}[2]{#1 and #2 are to the Same Virtual Address}
\newisrelationrev{lrs}{the Local read successor of}
\newisrelationrev{lws}{a Local write successor of}
\edef\povaloc#1#2{\po{#1}{#2} and \valoc{#1}{#2}}

\newcommand{\samereg}[2]{#1 and #2 are to the Same Register}
\newcommand{\sameval}[2]{#2 takes its data from #1}
\newcommand{\scl}[2]{#1 and #2 are to the Same Cache Line}
\newisrelation{dob}{dependency-ordered-before}
\newisrelation{pob}{pick-ordered-before}
\newisrelation{aob}{atomic-ordered-before}
\newisrelation{bob}{barrier-ordered-before}
\newisrelation{DSBob}{DSB-ordered-before}
\newisrelation{CSEob}{CSE-ordered-before}
\newisrelation{TLBIb}{TLBI-before}
\newisrelation{TLBIa}{TLBI-after}
\newisrelation{DCb}{DC-before}
\newisrelation{DCa}{DC-after}
\newisrelation{ICb}{IC-before}
\newisrelation{ICa}{IC-after}
\newisrelation{TLBIcb}{TLBI-coherence-before}
\newisrelation{TLBIca}{TLBI-coherence-after}
\newisrelation{ICcb}{IC-coherence-before}
\newisrelation{ICca}{IC-coherence-after}

\newisrelation{cb}{Coherence-before}
\newisrelation{ca}{Coherence-after}


\newisrelation{lob}{Locally-ordered-before}
\newisrelation{picklob}{Pick-locally-ordered-before}
\newisrelation{localhwreqs}{Locally-hardware-required-ordered-before}

\newisrelation{hazob}{Hazard-ordered-before}
\newisrelation{Exphazob}{Explicitly-hazard-ordered-before}
\newisrelation{TTDreadob}{TTD-read-ordered-before}
\newisrelation{TLBIob}{TLBI-ordered-before}
\newisrelation{TLBIafter}{TLBI-after}
\newisrelation{Instrreadob}{Instr-read-ordered-before}
\newisrelation{ICob}{IC-ordered-before}
\newisrelation{ICafter}{IC-after}
\newisrelation{hwreqs}{Hardware-required-ordered-before}
\newisrelation{obs}{Observed-by}
\newisrelation{Expobs}{Explicitly-Observed-by}
\newisrelation{Tagobs}{Tag-Observed-by}
\newisrelation{TTDobs}{TTD-Observed-by}
\newisrelation{Instrobs}{Instr-Observed-by}
\newisrelation{ob}{Ordered-before}

\newisrelation{TLBuncacheablepred}{a TLBUncacheable-Write-Predecessor of}
\newisrelation{TLBuncacheablesucc}{a TLBUncacheable-Write-Successor of}
\newisrelation{HUpred}{a Hardware-Update-Predecessor of}
\newisrelation{HUsucc}{a Hardware-Update-Successor of}

%aarch64: applications of "range"
\newcommand{\byamo}[1]{#1 is generated by \amo\xspace}
\newcommand{\rangeAamoL}[1]{All of the following applies: 
                           \begin{itemize}
                           \item \ExpW{#1}; 
                           \item \A{#1};
                           \item \REL{#1}; and
                           \item \byamo{#1}
                           \end{itemize}}
\newcommand{\rangelxsx}[1]{#1 is generated by a Store Exclusive instruction as part of \lxsx}
\newcommand{\rangetribminus}[3]{\ImpTTDR{#1} and \trib{#1}{an Effect #2 such that #3}}
