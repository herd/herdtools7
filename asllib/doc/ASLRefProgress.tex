\documentclass{book}
\usepackage{amsmath}  % Classic math package
\usepackage{amssymb}  % Classic math package 
\usepackage{mathtools}  % Additional math package 
\usepackage{graphicx}  % For figures
\usepackage{caption}  % For figure captions
\usepackage{subcaption}  % For subfigure captions
\usepackage{url}  % Automatically escapes urls
\usepackage{hyperref}  % Insert links inside pdfs
\usepackage[inline]{enumitem}  % For inline lists
\usepackage[export]{adjustbox}  % For centering too wide figures 
\usepackage{mathpartir}  % For deduction rules and equations paragraphs
\usepackage{comment}
\usepackage{fancyvrb}
\input{ifempty}
\input{ifformal}
%% Should be functional
\ifempty
\newcommand{\isempty}[1]{#1}
\else
\newcommand{\isempty}[1]{}
\fi

%%%Safety net
\iffalse
\makeatletter
\newcommand{\isempty}[1]{#1}
\makeatother
\fi


\usepackage{enumitem}
\renewlist{itemize}{itemize}{20}
\setlist[itemize,1]{label=\textbullet}
\setlist[itemize,2]{label=\textasteriskcentered} 
\setlist[itemize,3]{label=\textendash}
\setlist[itemize,4]{label=$\triangleright$}
\setlist[itemize,5]{label=+}
\setlist[itemize,6]{label=\textbullet}
\setlist[itemize,7]{label=\textasteriskcentered}
\setlist[itemize,8]{label=\textendash}
\setlist[itemize,9]{label=$\triangleright$}
\setlist[itemize,10]{label=+}
\setlist[itemize,11]{label=\textbullet}
\setlist[itemize,12]{label=\textasteriskcentered}
\setlist[itemize,13]{label=\textendash}
\setlist[itemize,14]{label=\textendash}
\setlist[itemize,15]{label=$\triangleright$}
\setlist[itemize,16]{label=+}
\setlist[itemize,17]{label=\textbullet}
\setlist[itemize,18]{label=\textasteriskcentered}
\setlist[itemize,19]{label=\textendash}
\setlist[itemize,20]{label=\textendash}

%%%%%%%%%%%%%%%%%%%%%%%%%%%%%%%%%%%%%%%%%%%%%%%%%%
%Abstract Syntax macros
% Non-terminal names
\newcommand\unop[0]{\textsf{unop}}
\newcommand\binop[0]{\textsf{binop}}
\newcommand\literal[0]{\textsf{literal}}
\newcommand\expr[0]{\textsf{expr}}
\newcommand\slice[0]{\textsf{slice}}
\newcommand\ty[0]{\textsf{ty}}
\newcommand\pattern[0]{\textsf{pattern}}
\newcommand\intconstraint[0]{\textsf{int\_constraint}}
\newcommand\bitsconstraint[0]{\textsf{bits\_constraint}}
\newcommand\bitfield[0]{\textsf{bitfield}}

\newcommand\Field[0]{\textsf{field}}

% Type names
\newcommand\TInt[0]{\texttt{T\_Int}}
\newcommand\TReal[0]{\texttt{T\_Real}}
\newcommand\TString[0]{\texttt{T\_String}}
\newcommand\TBool[0]{\texttt{T\_Bool}}
\newcommand\TBits[0]{\texttt{T\_Bits}}
\newcommand\TEnum[0]{\texttt{T\_Enum}}
\newcommand\TTuple[0]{\texttt{T\_Tuple}}
\newcommand\TArray[0]{\texttt{T\_Array}}
\newcommand\TRecord[0]{\texttt{T\_Record}}
\newcommand\TException[0]{\texttt{T\_Exception}}
\newcommand\TNamed[0]{\texttt{T\_Named}}

\newcommand\identifier[0]{\texttt{<identifier>}}

\newcommand\id[0]{\texttt{id}}        
%%%%%%%%%%%%%%%%%%%%%%%%%%%%%%%%%%%%%%%%%%%%%%%%%%

%%%%%%%%%%%%%%%%%%%%%%%%%%%%%%%%%%%%%%%%%%%%%%%%%%
% Typing macros
\newcommand\tstruct[0]{\texttt{t\_struct}}
%%%%%%%%%%%%%%%%%%%%%%%%%%%%%%%%%%%%%%%%%%%%%%%%%%

\newcommand\ie{i.\,e.}
\newcommand\eg{e.\,g.}
\newcommand\synor{\ |\ }
\newcommand\syntt[1]{\mathtt{#1}}
\newcommand\ife[3]{\text{if}\ #1\ \text{then}\ #2\ \text{else}\ #3\ \text{end}}
\newcommand\inenv[2]{\left\langle #1, #2 \right\rangle}
\newcommand\env[1]{\left\langle #1 \right\rangle}
\newcommand\reducesto{\ \to\ }
\newcommand\llbracket{[|}
\newcommand\rrbracket{|]}
\newcommand\interp[1]{\left\llbracket #1 \right\rrbracket}
\newcommand\st[0]{\ \middle|\ }
\newcommand\field[1]{.\text{#1}}
\newcommand\globals[0]{\field{globals}}
\newcommand\locals[0]{\field{locals}}
\newcommand\X[0]{\mathcal{X}}
\newcommand\N[0]{\mathbb{N}}
\newcommand\asldata[0]{\mathtt{asl\_data}}
\newcommand\aslctrl[0]{\mathtt{asl\_ctrl}}
\newcommand\aslpo[0]{\mathtt{asl\_po}}
\DeclareMathOperator{\dom}{dom}

%%%%%%%%%%%%%%%%%%%%%%%%%%%%%%%%%%%%%%%%%%%%%%%%%%
% Ident info

\newcommand\ident[2]{\texttt{#1}\textsubscript{\texttt{\MakeUppercase{#2}}}}
\newcommand\identi[1]{\ident{I}{#1}}
\newcommand\identr[1]{\ident{R}{#1}}
\newcommand\identd[1]{\ident{D}{#1}}
\newcommand\identg[1]{\ident{G}{#1}}



\author{Arm Architecture Technology Group}
\title{ASL Reference Progress \\
       DDI 0623}
\begin{document}
\maketitle

\tableofcontents{}

\chapter{Non-Confidential Proprietary Notice}

This document is protected by copyright and other related rights and the
practice or implementation of the information contained in this document may be
protected by one or more patents or pending patent applications. No part of
this document may be reproduced in any form by any means without the express
prior written permission of Arm. No license, express or implied, by estoppel or
otherwise to any intellectual property rights is granted by this document
unless specifically stated.

Your access to the information in this document is conditional upon your
acceptance that you will not use or permit others to use the information for
the purposes of determining whether implementations infringe any third party
patents.

THIS DOCUMENT IS PROVIDED “AS IS”. ARM PROVIDES NO REPRESENTATIONS AND NO
WARRANTIES, EXPRESS, IMPLIED OR STATUTORY, INCLUDING, WITHOUT LIMITATION, THE
IMPLIED WARRAN-TIES OF MERCHANTABILITY, SATISFACTORY QUALITY, NON-INFRIN-GEMENT
OR FITNESS FOR A PARTICULAR PURPOSE WITH RESPECT TO THE DOCUMENT. For the
avoidance of doubt, Arm makes no representation with respect to, and has
undertaken no analysis to identify or understand the scope and content of, any
patents, copyrights, trade secrets, trademarks, or other rights.

This document may include technical inaccuracies or typographical errors.

TO THE EXTENT NOT PROHIBITED BY LAW, IN NO EVENT WILL ARM BE LIABLE FOR ANY
DAMAGES, INCLUDING WITHOUT LIMITATION ANY DIRECT, INDIRECT, SPECIAL,
INCIDENTAL, PUNITIVE, OR CONSEQUENTIAL DAMAGES, HOWEVER CAUSED AND REGARDLESS
OF THE THEORY OF LIABILITY, ARISING OUT OF ANY USE OF THIS DOCUMENT, EVEN IF
ARM HAS BEEN ADVISED OF THE POSSIBILITY OF SUCH DAMAGES.

This document consists solely of commercial items. You shall be responsible for
ensuring that any use, duplication or disclosure of this document complies
fully with any relevant export laws and regulations to assure that this
document or any portion thereof is not exported, directly or indirectly, in
violation of such export laws. Use of the word “partner” in reference to Arm’s
customers is not intended to create or refer to any partnership relationship
with any other company. Arm may make changes to this document at any time and
without notice.

This document may be translated into other languages for convenience, and you
agree that if there is any conflict between the English version of this
document and any translation, the terms of the English version of this document
shall prevail.

The Arm corporate logo and words marked with ® or ™ are registered trademarks
or trademarks of Arm Limited (or its affiliates) in the US and/or elsewhere.
All rights reserved.  Other brands and names mentioned in this document may be
the trademarks of their respective owners. Please follow Arm’s trademark usage
guidelines at

\url{https://www.arm.com/company/policies/trademarks.}

Copyright © [2023,2024] Arm Limited (or its affiliates). All rights reserved.

Arm Limited. Company 02557590 registered in England.  110 Fulbourn Road,
Cambridge, England CB1 9NJ.  (LES-PRE-20349)


\chapter{Disclaimer}

This document is part of the ASLRef material.

This material covers both ASLv0 (viz, the existing ASL pseudocode language
which appears in the Arm Architecture Reference Manual) and ASLv1, a new,
experimental, and as yet unreleased version of ASL.

The development version of ASLRef can be found here \url{~/herdtools7/asllib}.

A list of open items being worked on can be found here
\url{~/herdtools7/asllib/doc/ASLRefProgress.tex}.

This material is work in progress, more precisely at Alpha quality as
per Arm’s quality standards. In particular, this means that it would be
premature to base any production tool development on this material.

However, any feedback, question, query and feature request would be most
welcome; those can be sent to Arm’s Architecture Formal Team Lead Jade Alglave
\texttt{(jade.alglave@arm.com)} or by raising issues or PRs to the herdtools7
github repository.


% ------------------------------------------------------------------------------
\chapter{Not implemented in ASLRef}
% ------------------------------------------------------------------------------

This chapter describes what is not yet present in the executable version of ASLRef
versus the LRM (Build \#215 from Feb 22, 2024).

% ------------------------------------------------------------------------------
\section{Syntax}

\subsection{Pragmas}
ASLRef does not currently parse pragmas:
\begin{verbatim}
pragma asl_pragma1;
\end{verbatim}

This is related to \identi{ZGJQ}, \identr{GBNH}, \identr{GFSD}.

\subsection{Declaring Multiple Identifiers with No Initialization}
The following simultaneous declaration of three global variables does not currently parse with ASLRef.
\begin{verbatim}
 var x, y, z : integer;
\end{verbatim}

The same line does parse and correctly handled inside a subprogram.

This is related to \identr{QDQD}.

\subsection{Annotations}
ASLRef does not yet support annotations.
Specifically loop limit and recursion limit annotations are not yet supported.

\subsection{Recursion Limits}
ASLRef does not yet parse and support \texttt{@recurselimit(<LIMIT>)} annotations.

\subsection{Concatenation Declarations}
Declarations of multiple bitvectors via concatenation as in the program
\begin{Verbatim}
func main() => integer
begin
    var [ a[7:0], b, c[3:0] ] = Zeros(13);
    return 0;
end
\end{Verbatim}
do not currently parse.

This is related to \identr{KKDF}.

\subsection{Guards}

Guards are used on \texttt{case} and \texttt{catch} statements, to restrict
matching on the evaluation of a boolean expression.
%
They are not yet implemented in ASLRef.

This relates to \identr{WGSY}.

% ------------------------------------------------------------------------------
\section{Typing}

\subsection{Checking that All Paths in a Function Return a Value}
The following function currently passes type-checking even though
it does not return a value when \texttt{a <= 7}.
\begin{verbatim}
func foo(a : integer) => integer
begin
  if (a > 7) then
    return 0;
  end
end
\end{verbatim}
This requires a control-flow analysis.

A call to \texttt{Unreachable} needs to indicate to the control-flow
analysis that it is okay for the \texttt{else} path to not return a value.
\begin{verbatim}
func foo(a : integer) => integer
begin
  if (a > 7) then
    return 0;
  else
    Unreachable();
  end
end
\end{verbatim}

\subsection{Side-effect-free Subprograms}
ASLRef does yet infer whether a subprorgam is side-effect-free.
Therefore, there are no checks that expressions are side-effect-free when those are expected,
for example, in \texttt{for} loop ranges.

This is related to \identr{WQRN}, \identr{SNQJ}, \identr{DJMC}, \identr{KLDR}.

\subsection{Statically evaluable programs}%
\label{sec:nyi:statically-evaluable-subprograms}

Side effects analysis has not been implemented yet.
%
This makes detection of statically evaluable subprograms impossible.

Furthermore, non-execution time subprograms, expressions, and types have not
been implemented.

This is related to \identi{LZCX}, \identi{NXJR}, \identr{CSFT}, \identi{HYBT},
\identd{CCTY}, \identi{LYKD}, \identi{ZPWM}, \identd{KCKX}, \identi{NTYZ},
\identi{MSZT}, \identd{QNHM}, \identi{XYKC}, \identd{ZPMF}, \identi{XSDY},
\identd{XRBT}, \identi{WVGG}, \identd{JLJD}, \identi{KKDY}, \identd{MTQJ},
\identi{YBGL}, \identi{YMRT}, \identi{QJTN}, \identi{GFZT}.

\subsection{Restriction on use of under-constrained types}

\subsubsection{As storage types}
Restrictions on use of under-constrained types as storage element types are not
implemented.

This is related to \identr{ZCVD}.

\subsubsection{As expression with a constrained type}

Restriction on the use of under-constrained parameters as left-hand-side of a
Asserted Typed Conversion is not implemented in ASLRef.
%
For example, the following will not raise a type-error:
\VerbatimInput[firstline=3,lastline=8]{../tests/regressions.t/under-constrained-used.asl}

This is related to \identi{TBHH}, \identr{ZDKC}.

% ------------------------------------------------------------------------------
\section{Semantics}

\subsection{Non-\texttt{main} Entry Point}
Currently ASLRef only supports \texttt{main} as an entry point.

\subsection{Real exponentiation}

The exponentiation operation \texttt{exp\_real} has not been implemented.
%
Note: this would construct non-rational numbers which are not supported by
ASLRef, e.\,g.\ $2^\frac{1}{2} = \sqrt{2}$.

This is related to \identr{BNCY}.

% ------------------------------------------------------------------------------
\chapter{Not transliterated in ASL reference documents}
% ------------------------------------------------------------------------------

% ------------------------------------------------------------------------------
\section{Typing}

\subsection{Global storage declarations}

Global storage declarations are transliterated.
The formal rules need a bit more refinement (e.g., defining \texttt{declare\_const} and \texttt{reduce\_constants}).

This is related to \identr{FWQM}.

\subsection{Statically evaluable expressions}

The part of statically evaluable expressions that has been implemented in
ASLRef (see Section~\ref{sec:nyi:statically-evaluable-subprograms}) has not
been transliterated.

Equivalence of statically evaluable expressions has been implemented in a
restricted setting (see
Section~\ref{sec:nyi:statically-evaluable-subprograms}).
%
Mainly, expressions that reduce to polynomials can be checked for equivalence.
%
This has not been transliterated either.

This is related to \identr{PKXK}, \identd{YYDW}, \identd{CWVH}, \identd{HLQC},
and \identi{LHLR}, \identr{RFQP}, \identr{VNKT}.

\subsection{Polymorphism}

Polymorphism in ASL is the ability to have multiple subprograms with the same
name that do not have the same signature.

Although polymorphism is implemented in ASLRef, it has not yet been
transliterated.

This is related to \identd{BTBR}, \identi{FSFQ}, \identi{FCTB}, \identi{PFGQ},
\identr{PGFC}, \identi{BTMT}.

\subsection{Calls to setters and getters}

The replacement of implicit calls to getters and setters (written for example as
slices) to explicit calls to subprograms has not been transliterated.
%
In terms of abstract syntax, this corresponds to the translation between a
\texttt{E\_Slice} and a \texttt{E\_Call}.

This is related to \identi{YYQX}, \identi{LJLW}, \identi{MFBC}.

\subsection{Type inference from literals}

Finding the type of a literal value, or a compile-time constant expression, is
not yet transliterated.

This is related to \identr{ZJKY} and \identi{RYRP}.

% ------------------------------------------------------------------------------
\section{Semantics}

\subsection{Base values of types}

Finding the base value of a type is not transliterated.

This is related to \identr{NJDZ}, \identr{CFTD}, \identr{QGGH}, \identi{WVQZ},
\identr{GYCG}, \identr{WKCY}, \identr{LCCN}, \identr{CPCK}, \identr{ZVPT},
\identi{QFZH}, \identr{ZGVR}, \identi{PGSS}, \identr{QWSQ}, \identr{HMRK},
\identr{MBRM}, \identr{SVJB}.

\subsection{Operations}

Operations are not transliterated from ASLRef.

This is related to \identr{NCWM}, \identr{VGZF}, \identr{THSV}, \identr{CRQJ},
\identr{ZTJN}, \identr{SVMM}, \identr{WWTV}, \identr{GHXR}, \identi{NBCT}.

\subsection{Standard library and primitives}

The standard library is not transliterated.

This is related to \identr{RXYN}.

% ------------------------------------------------------------------------------
\section{Side-Effects}

Side-Effects are not yet considered in ASLRef.

\section{In scope for BET1}

Need to populate - for example Side-Effects should be there, as well as some
work on constraints.
Generally we need to review this Progress document and update it.
\end{document}
