\chapter{Not Implemented by ASLRef\label{appendix:UnimplementedInASLRef}}
% ------------------------------------------------------------------------------

This chapter describes what is not yet present in the executable version of ASLRef
(Build \#215 from Nov 14, 2024).

% ------------------------------------------------------------------------------
\section{Syntax}

\subsection{Pragmas}
ASLRef does not currently parse pragmas:
\begin{verbatim}
pragma asl_pragma1;
\end{verbatim}

\lrmcomment{This is related to \identi{ZGJQ}, \identr{GBNH}, \identr{GFSD}.}

\subsection{Declaring Multiple Identifiers Without Initialization}
The following simultaneous declaration of three global variables does not currently parse with ASLRef.
\begin{verbatim}
 var x, y, z : integer;
\end{verbatim}

The same line does parse and correctly handled inside a subprogram.

\lrmcomment{This is related to \identr{QDQD}.}

\subsection{Annotations}
ASLRef does not yet support annotations in general.
Loop limit annotations are supported, but recursion limit annotations are not yet supported.

\subsection{Recursion Limits}
ASLRef does not yet parse and support \texttt{@recurselimit(<LIMIT>)} annotations.

\subsection{Concatenation Declarations}
Declarations of multiple bitvectors via concatenation as in the program
\begin{Verbatim}
func main() => integer
begin
    var [ a[7:0], b, c[3:0] ] = Zeros(13);
    return 0;
end
\end{Verbatim}
do not currently parse.

\lrmcomment{This is related to \identr{KKDF}.}

\subsection{Guards}

Guards are used on \texttt{case} and \texttt{catch} statements, to restrict
matching on the evaluation of a boolean expression.
%
They are not yet implemented in ASLRef.

\lrmcomment{This relates to \identr{WGSY}.}

% ------------------------------------------------------------------------------
\section{Semantics}

\subsection{Enforcing Loop Limits}
\verb|@looplimit| annotations are type-checked but not enforced for \texttt{while} and \texttt{repeat} loops.

\subsection{Non-\texttt{main} Entry Point}
Currently ASLRef only supports \texttt{main} as an entry point.

% ------------------------------------------------------------------------------
\section{Typing}

\subsection{Throwing Exceptions without Braces}
In the following example, the commented out \texttt{throw} statement should type-check,
but it currently fails.

\begin{verbatim}
  type except of exception;

  func main() => integer
  begin
    // throw except; // Should type-check
    throw except{}; // Okay

    return 0;
  end
\end{verbatim}

\subsection{Side-effect-free Subprograms}
ASLRef does yet infer whether a subprorgam is side-effect-free.
Therefore, there are no checks that expressions are side-effect-free when those are expected,
for example, in \texttt{for} loop ranges.

\lrmcomment{This is related to \identr{WQRN}, \identr{SNQJ}, \identr{DJMC}, \identr{KLDR}.}

\subsection{Statically evaluable programs}%
\label{sec:nyi:statically-evaluable-subprograms}

Side effects analysis has not been implemented yet.
%
This makes detection of statically evaluable subprograms impossible.

Furthermore, non-execution time subprograms, expressions, and types have not
been implemented.

\lrmcomment{This is related to \identi{LZCX}, \identi{NXJR}, \identr{CSFT}, \identi{HYBT},
\identd{CCTY}, \identi{LYKD}, \identi{ZPWM}, \identd{KCKX}, \identi{NTYZ},
\identi{MSZT}, \identd{QNHM}, \identi{XYKC}, \identd{ZPMF}, \identi{XSDY},
\identd{XRBT}, \identi{WVGG}, \identd{JLJD}, \identi{KKDY}, \identd{MTQJ},
\identi{YBGL}, \identi{YMRT}, \identi{QJTN}, \identi{GFZT}.}

\subsection{Restriction on Use of Parameterized Integer Types}

\subsubsection{As storage types}
Restrictions on the use of parameterized integer types as storage element types are not
implemented.

\lrmcomment{This is related to \identr{ZCVD}.}

\subsubsection{\texttt{as} Expression With a Constrained Type}

Restriction on the use of parameterized integer types as left-hand-side of a
Asserted Typed Conversion is not implemented in ASLRef.
%
For example, the following will not raise a type-error:
\VerbatimInput[firstline=1,lastline=4]{../tests/regressions.t/under-constrained-used.asl}

\lrmcomment{This is related to \identi{TBHH}, \identr{ZDKC}.}

% ------------------------------------------------------------------------------
\chapter{Issues Not Yet Addressed by the Reference\label{appendix:MissingTransliteration}}
% ------------------------------------------------------------------------------
\section{Semantics}

\subsection{Standard Library and Primitives}

The standard library is not yet defined by a reference.

\lrmcomment{This is related to \identr{RXYN}.}

% ------------------------------------------------------------------------------
\section{Typing}

\subsection{Checking Type Annotations for Absence of Side Effects}
Type annotations that contain expressions must ensure that those expressions are side-effect-free.
This is currently ensured by disallowing such expressions to contain call expressions.
Allowing side-effect-free calls is being considered.

\subsection{Enumeration Labels}
Enumeration labels are not yet treated as first-class literals, but rather as integer literals.
The domain of an enumeration type should be a set of enumeration labels.

% \subsection{Global storage declarations}
% Global storage declarations are transliterated.
% The formal rules need a bit more refinement (e.g., defining \texttt{declare\_const} and \texttt{reduce\_constants}).
% \lrmcomment{This is related to \identr{FWQM}.}

% \subsection{Statically evaluable expressions}

% The part of statically evaluable expressions that has been implemented in
% ASLRef (see Section~\ref{sec:nyi:statically-evaluable-subprograms}) has not
% been transliterated.

% Equivalence of statically evaluable expressions has been implemented in a
% restricted setting (see
% Section~\ref{sec:nyi:statically-evaluable-subprograms}).
% %
% Mainly, expressions that reduce to polynomials can be checked for equivalence.
% %
% This has not been transliterated either.

% \lrmcomment{This is related to \identr{PKXK}, \identd{YYDW}, \identd{CWVH}, \identd{HLQC},
% and \identi{LHLR}, \identr{RFQP}, \identr{VNKT}.}

% \subsection{Polymorphism}

% Polymorphism in ASL is the ability to have multiple subprograms with the same
% name that do not have the same signature.

% Although polymorphism is implemented in ASLRef, it has not yet been
% transliterated.

% \lrmcomment{This is related to \identd{BTBR}, \identi{FSFQ}, \identi{FCTB}, \identi{PFGQ},
% \identr{PGFC}, \identi{BTMT}.}

\subsection{Calls to Setters and Getters}
The replacement of implicit calls to getters and setters (written for example as
slices) to explicit calls to subprograms has not been defined.
%
In terms of abstract syntax, this corresponds to the translation between a
\texttt{E\_Slice} and a \texttt{E\_Call}.

\lrmcomment{This is related to \identi{YYQX}, \identi{LJLW}, \identi{MFBC}.}

% \subsection{Type inference from literals}
% Finding the type of a literal value, or a compile-time constant expression, is
% not yet transliterated.

% \lrmcomment{This is related to \identr{ZJKY} and \identi{RYRP}.}

% ------------------------------------------------------------------------------
\section{Semantics}

\subsection{Standard Library and Primitives}

The standard library is not yet defined by the reference.

\lrmcomment{This is related to \identr{RXYN}.}

% ------------------------------------------------------------------------------
\section{Side-Effects}

Side-Effects are not yet defined by the reference.

% \section{In scope for BET1}

% Need to populate - for example Side-Effects should be there, as well as some
% work on constraints.
% Generally we need to review this Progress document and update it.

%\end{document}
