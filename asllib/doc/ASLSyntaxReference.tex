\documentclass{book}
\usepackage{amsmath}  % Classic math package
\usepackage{amssymb}  % Classic math package 
\usepackage{mathtools}  % Additional math package 
\usepackage{graphicx}  % For figures
\usepackage{caption}  % For figure captions
\usepackage{subcaption}  % For subfigure captions
\usepackage{url}  % Automatically escapes urls
\usepackage{hyperref}  % Insert links inside pdfs
\usepackage[inline]{enumitem}  % For inline lists
\usepackage[export]{adjustbox}  % For centering too wide figures 
\usepackage{mathpartir}  % For deduction rules and equations paragraphs
\usepackage{comment}
\usepackage{fancyvrb}
\input{ifempty}
\input{ifformal}
%% Should be functional
\ifempty
\newcommand{\isempty}[1]{#1}
\else
\newcommand{\isempty}[1]{}
\fi

%%%Safety net
\iffalse
\makeatletter
\newcommand{\isempty}[1]{#1}
\makeatother
\fi


\usepackage{enumitem}
\renewlist{itemize}{itemize}{20}
\setlist[itemize,1]{label=\textbullet}
\setlist[itemize,2]{label=\textasteriskcentered} 
\setlist[itemize,3]{label=\textendash}
\setlist[itemize,4]{label=$\triangleright$}
\setlist[itemize,5]{label=+}
\setlist[itemize,6]{label=\textbullet}
\setlist[itemize,7]{label=\textasteriskcentered}
\setlist[itemize,8]{label=\textendash}
\setlist[itemize,9]{label=$\triangleright$}
\setlist[itemize,10]{label=+}
\setlist[itemize,11]{label=\textbullet}
\setlist[itemize,12]{label=\textasteriskcentered}
\setlist[itemize,13]{label=\textendash}
\setlist[itemize,14]{label=\textendash}
\setlist[itemize,15]{label=$\triangleright$}
\setlist[itemize,16]{label=+}
\setlist[itemize,17]{label=\textbullet}
\setlist[itemize,18]{label=\textasteriskcentered}
\setlist[itemize,19]{label=\textendash}
\setlist[itemize,20]{label=\textendash}

%%%%%%%%%%%%%%%%%%%%%%%%%%%%%%%%%%%%%%%%%%%%%%%%%%
%Abstract Syntax macros
% Non-terminal names
\newcommand\unop[0]{\textsf{unop}}
\newcommand\binop[0]{\textsf{binop}}
\newcommand\literal[0]{\textsf{literal}}
\newcommand\expr[0]{\textsf{expr}}
\newcommand\slice[0]{\textsf{slice}}
\newcommand\ty[0]{\textsf{ty}}
\newcommand\pattern[0]{\textsf{pattern}}
\newcommand\intconstraint[0]{\textsf{int\_constraint}}
\newcommand\bitsconstraint[0]{\textsf{bits\_constraint}}
\newcommand\bitfield[0]{\textsf{bitfield}}

\newcommand\Field[0]{\textsf{field}}

% Type names
\newcommand\TInt[0]{\texttt{T\_Int}}
\newcommand\TReal[0]{\texttt{T\_Real}}
\newcommand\TString[0]{\texttt{T\_String}}
\newcommand\TBool[0]{\texttt{T\_Bool}}
\newcommand\TBits[0]{\texttt{T\_Bits}}
\newcommand\TEnum[0]{\texttt{T\_Enum}}
\newcommand\TTuple[0]{\texttt{T\_Tuple}}
\newcommand\TArray[0]{\texttt{T\_Array}}
\newcommand\TRecord[0]{\texttt{T\_Record}}
\newcommand\TException[0]{\texttt{T\_Exception}}
\newcommand\TNamed[0]{\texttt{T\_Named}}

\newcommand\identifier[0]{\texttt{<identifier>}}

\newcommand\id[0]{\texttt{id}}        
%%%%%%%%%%%%%%%%%%%%%%%%%%%%%%%%%%%%%%%%%%%%%%%%%%

%%%%%%%%%%%%%%%%%%%%%%%%%%%%%%%%%%%%%%%%%%%%%%%%%%
% Typing macros
\newcommand\tstruct[0]{\texttt{t\_struct}}
%%%%%%%%%%%%%%%%%%%%%%%%%%%%%%%%%%%%%%%%%%%%%%%%%%

\newcommand\ie{i.\,e.}
\newcommand\eg{e.\,g.}
\newcommand\synor{\ |\ }
\newcommand\syntt[1]{\mathtt{#1}}
\newcommand\ife[3]{\text{if}\ #1\ \text{then}\ #2\ \text{else}\ #3\ \text{end}}
\newcommand\inenv[2]{\left\langle #1, #2 \right\rangle}
\newcommand\env[1]{\left\langle #1 \right\rangle}
\newcommand\reducesto{\ \to\ }
\newcommand\llbracket{[|}
\newcommand\rrbracket{|]}
\newcommand\interp[1]{\left\llbracket #1 \right\rrbracket}
\newcommand\st[0]{\ \middle|\ }
\newcommand\field[1]{.\text{#1}}
\newcommand\globals[0]{\field{globals}}
\newcommand\locals[0]{\field{locals}}
\newcommand\X[0]{\mathcal{X}}
\newcommand\N[0]{\mathbb{N}}
\newcommand\asldata[0]{\mathtt{asl\_data}}
\newcommand\aslctrl[0]{\mathtt{asl\_ctrl}}
\newcommand\aslpo[0]{\mathtt{asl\_po}}
\DeclareMathOperator{\dom}{dom}

%%%%%%%%%%%%%%%%%%%%%%%%%%%%%%%%%%%%%%%%%%%%%%%%%%
% Ident info

\newcommand\ident[2]{\texttt{#1}\textsubscript{\texttt{\MakeUppercase{#2}}}}
\newcommand\identi[1]{\ident{I}{#1}}
\newcommand\identr[1]{\ident{R}{#1}}
\newcommand\identd[1]{\ident{D}{#1}}
\newcommand\identg[1]{\ident{G}{#1}}



\author{Arm Architecture Technology Group}
\title{ASL Syntax Reference \\
       DDI 0620}
\begin{document}
\maketitle

\tableofcontents{}

\chapter{Non-Confidential Proprietary Notice}

This document is protected by copyright and other related rights and the
practice or implementation of the information contained in this document may be
protected by one or more patents or pending patent applications. No part of
this document may be reproduced in any form by any means without the express
prior written permission of Arm. No license, express or implied, by estoppel or
otherwise to any intellectual property rights is granted by this document
unless specifically stated.

Your access to the information in this document is conditional upon your
acceptance that you will not use or permit others to use the information for
the purposes of determining whether implementations infringe any third party
patents.

THIS DOCUMENT IS PROVIDED “AS IS”. ARM PROVIDES NO REPRESENTATIONS AND NO
WARRANTIES, EXPRESS, IMPLIED OR STATUTORY, INCLUDING, WITHOUT LIMITATION, THE
IMPLIED WARRAN-TIES OF MERCHANTABILITY, SATISFACTORY QUALITY, NON-INFRIN-GEMENT
OR FITNESS FOR A PARTICULAR PURPOSE WITH RESPECT TO THE DOCUMENT. For the
avoidance of doubt, Arm makes no representation with respect to, and has
undertaken no analysis to identify or understand the scope and content of, any
patents, copyrights, trade secrets, trademarks, or other rights.

This document may include technical inaccuracies or typographical errors.

TO THE EXTENT NOT PROHIBITED BY LAW, IN NO EVENT WILL ARM BE LIABLE FOR ANY
DAMAGES, INCLUDING WITHOUT LIMITATION ANY DIRECT, INDIRECT, SPECIAL,
INCIDENTAL, PUNITIVE, OR CONSEQUENTIAL DAMAGES, HOWEVER CAUSED AND REGARDLESS
OF THE THEORY OF LIABILITY, ARISING OUT OF ANY USE OF THIS DOCUMENT, EVEN IF
ARM HAS BEEN ADVISED OF THE POSSIBILITY OF SUCH DAMAGES.

This document consists solely of commercial items. You shall be responsible for
ensuring that any use, duplication or disclosure of this document complies
fully with any relevant export laws and regulations to assure that this
document or any portion thereof is not exported, directly or indirectly, in
violation of such export laws. Use of the word “partner” in reference to Arm’s
customers is not intended to create or refer to any partnership relationship
with any other company. Arm may make changes to this document at any time and
without notice.

This document may be translated into other languages for convenience, and you
agree that if there is any conflict between the English version of this
document and any translation, the terms of the English version of this document
shall prevail.

The Arm corporate logo and words marked with ® or ™ are registered trademarks
or trademarks of Arm Limited (or its affiliates) in the US and/or elsewhere.
All rights reserved.  Other brands and names mentioned in this document may be
the trademarks of their respective owners. Please follow Arm’s trademark usage
guidelines at

\url{https://www.arm.com/company/policies/trademarks.}

Copyright © [2023,2024] Arm Limited (or its affiliates). All rights reserved.

Arm Limited. Company 02557590 registered in England.  110 Fulbourn Road,
Cambridge, England CB1 9NJ.  (LES-PRE-20349)


\chapter{Disclaimer}

This document is part of the ASLRef material.

This material covers both ASLv0 (viz, the existing ASL pseudocode language
which appears in the Arm Architecture Reference Manual) and ASLv1, a new,
experimental, and as yet unreleased version of ASL.

The development version of ASLRef can be found here \url{~/herdtools7/asllib}.

A list of open items being worked on can be found here
\url{~/herdtools7/asllib/doc/ASLRefProgress.tex}.

This material is work in progress, more precisely at Alpha quality as
per Arm’s quality standards. In particular, this means that it would be
premature to base any production tool development on this material.

However, any feedback, question, query and feature request would be most
welcome; those can be sent to Arm’s Architecture Formal Team Lead Jade Alglave
\texttt{(jade.alglave@arm.com)} or by raising issues or PRs to the herdtools7
github repository.


\chapter{Preamble}
An abstract syntax is a form of context-free grammar over structured trees. Compilers and interpreters typically start by parsing the text of a program and producing an abstract syntax tree (AST, for short), and then continue to operate over that tree.
%
The reason for this is that abstract syntax trees abstract away details that are irrelevant to the semantics of the program, such as punctuation and scoping syntax, which are mostly there to facilitate parsing.

A major benefit of employing an abstract syntax for ASL is that it allows us to abstract both the grammar of ASLv0 and the grammar of ASLv1 into a single abstract syntax. This in turn, allows us to define a single type system and a single semantics for both language versions, with very few exceptions\footnote{Note the single appearance of $\version$ in the assignment derivation rule.}.

\chapter{ASL Abstract Syntax}

\section{ASL Abstract Syntax Trees}

In an ASL abstract syntax tree, a node is one the following data types:
\begin{description}
\item[Token Node.] A lexical token, denoted as in the lexical description of ASL;
\item[Label Node.] A label
\item[Unlabelled Tuple Node.] A tuple of children nodes, denoted as $(n_1,\ldots,n_k)$;
\item[Labelled Tuple Node.] A tuple labelled~$L$, denoted as~$L(n_1,\ldots,n_k)$;
\item[List Node.] A list of~$0$ or more children nodes, denoted as~$\emptylist$
when the list is empty and~$[n_1,\ldots,n_k]$ for non-empty lists;
\item[Optional.] An optional node stands for a list of 0 or 1 occurrences of a sub-node $n$. We denote an empty optional by $\langle\rangle$ and the non-empty optional by $\langle n \rangle$;
\item[Record Node.] A record node, denoted as $\{\text{name}_1 : n_1,\ldots,\text{name}_k : n_k\}$, where $\text{name}_1 \ldots \text{name}_k$ are names, which associates names with corresponding nodes.
\end{description}

\newpage

\section{ASL Abstract Syntax Grammar}

An abstract syntax is defined in terms of derivation rules containing variables (also referred to as non-terminals).
%
A \emph{derivation rule} has the form $v ::= \textit{rhs}$ where $v$ is a non-terminal variable and \textit{rhs} is a \emph{node type}. We write $n$, $n_1,\ldots,n_k$ to denote node types.
%
Node types are defined recursively as follows:
\begin{description}
\item[Non-terminal.] A non-terminal variable;
\item[Terminal.] A lexical token $t$ or a label $L$;
\item[Unlabelled Tuple.] A tuple of node types, denoted as~$(n_1,\ldots,n_k)$;
\item[Labelled Tuple.] A tuple labelled~$L$, denoted as~$L(n_1,\ldots,n_k)$;
\item[List.] A list node type, denoted as $n^{*}$;
\item[Optional.] An optional node type, denoted as $n?$;
\item[Record.] A record, denoted as $\{\text{name}_1 : n_1,\ldots,\text{name}_k : n_k\}$ where $\text{name}_i$, which associates names with corresponding node types.
\end{description}

\newpage

An abstract syntax consists of a set of derivation rules and a start non-terminal.

\newcommand\ASTComment[1]{//\quad\textit{#1}\ }

\section{ASL Derivation Rules}

The abstract syntax of ASL is given in terms of the derivation rules below and the start non-terminal $\program$.
%
We sometimes provide extra details to individual derivations by adding comments below them, in the form \ASTComment{this is a comment}.

\[
\begin{array}{rcl}
\unop &::=& \texttt{"!"} \;|\; \texttt{"-"} \;|\; \texttt{"NOT"} \\
\binop &::=& \texttt{"\&\&"} \;|\; \texttt{"||"} \;|\; \texttt{"-->"} \;|\;  \texttt{"<->"}  \\
 & & \ASTComment{\texttt{binop\_boolean}}\\

 &|& \texttt{"=="} \;|\; \texttt{"!="}  \;|\; \texttt{">"}  \;|\; \texttt{">="} \;|\; \texttt{"<"} \;|\; \texttt{"<="}   \\
 & & \ASTComment{\texttt{binop\_comparison}}\\

 &|& \texttt{"+"} \;|\; \texttt{"-"}  \;|\; \texttt{"OR"}  \;|\; \texttt{"XOR"} \;|\; \texttt{"EOR"} \;|\; \texttt{"AND"}   \\
 & & \ASTComment{\texttt{binop\_add\_sub\_logic}}\\

 &|& \texttt{"*"} \;|\; \texttt{"/"}  \;|\; \texttt{"DIV"}  \;|\; \texttt{"DIVRM"} \;|\; \texttt{"MOD"}  \;|\; \texttt{"<<"}  \;|\; \texttt{">>"}    \\
 & & \ASTComment{\texttt{binop\_mul\_div\_shift}}\\

 &|& \texttt{"\^{}"}   \\
 & & \ASTComment{\texttt{binop\_pow}}\\

\literal &=& \texttt{<int\_lit>}  \\
 &|&\texttt{<hex\_lit>} \\
 & & \ASTComment{merged into \texttt{<int\_lit>}?}\\
 &|&\texttt{<boolean\_lit>}  \\
 &|& \texttt{<real\_lit>}  \\
 &|& \texttt{<bitmask\_lit>}   \\
 & & \ASTComment{also represents \texttt{<bitvector\_lit>}}\\
 &|&\texttt{<string\_lit>} \\
 &|& \texttt{"IN"}   \\
 & & \ASTComment{\texttt{binop\_in}}\\

\literal &=& \lint($n$) \\
 & & \ASTComment{$n \in \Z$}\\
 &|& \lbool(b) \\
 & & \ASTComment{$b \in \{\True, \False\}$}\\
 &|& \lreal(q)  \\
 & & \ASTComment{$q \in \Q$}\\
 &|& \lbitvector($B$)   \\
 & & \ASTComment{$B \in \{0, 1\}^*$}\\
 &|& \lstring(S)\\
 & & \ASTComment{$S \in \{C \;|\; \texttt{"$C$"} \in \texttt{<string\_lit>}\} $}\\
\end{array}
\]

\[
\begin{array}{rcl}
\expr &::=& \texttt{E\_Literal}(\literal) \\
	&|& \texttt{E\_Var}(\texttt{\identifier}) \\
	&|& \texttt{E\_CTC}(\texttt{\expr}, \ty) \\
  & & \ASTComment{A checked type constraint}\\
	&|& \texttt{E\_Binop}(\binop, \expr, \expr) \\
	&|& \texttt{E\_Unop}(\unop, \expr) \\
	&|& \texttt{E\_Call}(\identifier, \expr^{*}, (\identifier, \expr)^{*}) \\
	&|& \texttt{E\_Slice}(\expr, \slice^{*}) \\
	&|& \texttt{E\_Cond}(\expr, \expr, \expr) \\
	&|& \texttt{E\_GetArray}(\expr, \expr) \\
	&|& \texttt{E\_GetField}(\expr, \identifier) \\
	&|& \texttt{E\_GetFields}(\expr, \identifier^{*}) \\
	&|& \texttt{E\_Record}(\ty, (\identifier, \expr)^{*})\\
    & & \ASTComment{Both record construction and exception construction}\\
	&|& \texttt{E\_Concat}(\expr^{*}) \\
	&|& \texttt{E\_Tuple}(\expr^{*}) \\
	&|& \texttt{E\_Unknown}(\ty) \\
	&|& \texttt{E\_Pattern}(\expr, \pattern)
\end{array}
\]

\[
\begin{array}{rcl}
\pattern &::=& \texttt{Pattern\_All} \\
	&|& \texttt{E\_Var}(\texttt{\identifier}) \\
  &|& \texttt{Pattern\_Any}(\pattern^{*}) \\
  &|& \texttt{Pattern\_Geq}(\expr) \\
  &|& \texttt{Pattern\_Leq}(\expr) \\
  &|& \texttt{Pattern\_Mask}(\texttt{bitmask\_lit}) \\
  &|& \texttt{Pattern\_Not}(\pattern) \\
  &|& \texttt{Pattern\_Range}(\expr, \expr)\\
  & & \ASTComment{Lower to upper, included.}\\
  &|& \texttt{Pattern\_Single}(\expr) \\
  &|& \texttt{Pattern\_Tuple}(\pattern^{*}) \\
\end{array}
\]

\[
\begin{array}{rcl}
&&\ASTComment{Indexes an array or a bitvector.}\\
&&\ASTComment{All positions mentioned below are inclusive}\\
\hline
\slice &::=& \slicesingle(\expr) \\
  & & \ASTComment{\slicesingle(\texttt{i}) is the slice of length \texttt{1} at position \texttt{i}.}\\
  &|& \slicerange(\expr, \expr) \\
  & & \ASTComment{\slicerange(\texttt{j, i}) denotes the slice from \texttt{i} to \texttt{j - 1}.}\\
  &|& \slicelength(\expr, \expr) \\
  & & \ASTComment{\slicelength(\texttt{i, n}) denotes the slice starting at \texttt{i} of length \texttt{n}.}\\
  &|& \texttt{Slice\_Star}(\expr, \expr) \\
  & & \ASTComment{\slicestar(\texttt{factor, length}) denotes the slice starting at }\\
  & & \ASTComment{\texttt{factor * length} of length n.}\\
\end{array}
\]

\[
\begin{array}{rcl}
\ty &::=& \TInt(\intconstraints) \\
  &|& \TBits(\bitsconstraint, \bitfield^{*}) \\
  &|& \TReal \\
  &|& \TString \\
  &|& \TBool \\
  &|& \TBits(\expr, bitfield^{*}) \\
  & & \ASTComment{\expr\ is a statically evaluable expression denoting the length of the bit-vector.}\\
  &|& \TEnum(\identifier^{*}) \\
  &|& \TTuple(\ty^{*}) \\
  &|& \TArray(\expr, \ty) \\
  &|& \TRecord(\Field^{*}) \\
  &|& \TException(\Field^{*}) \\
  &|& \TNamed(\identifier)\\
  & & \ASTComment{A type variable.}\\
  & & \ASTComment{This is related to \identi{LDNP}}
\end{array}
\]

\[
  \begin{array}{rcl}
    & & \ASTComment{Constraints that can be put on integer types.}  \\
    \hline
    \intconstraints & ::=
      & \texttt{UnConstrained}                                      \\
    & & \ASTComment{This integer type is unconstrained.}            \\
    &|& \texttt{WellConstrained}(\intconstraint^{*})                \\
    & & \ASTComment{Has some explicit constraints}                  \\
    &|& \texttt{UnderConstrained}(\texttt{uid}, \identifier)        \\
    & & \ASTComment{Implicitely constrained integer from function declaration.} \\
    & & \ASTComment{Attributes are:} \\
    & & \ASTComment{- a unique integer identifier and the variable} \\
    & & \ASTComment{- the type was implicitely constructed from.} \\
  \end{array}
\]

\[
\begin{array}{rcl}
& & \ASTComment{A constraint on an integer part.}\\
\hline
\intconstraint &::=& \texttt{Constraint\_Exact}(\expr) \\
  & & \ASTComment{A single value, given by a statically evaluable expression.}\\
  &|& \texttt{Constraint\_Range}(\expr, \expr) \\
  & & \ASTComment{An interval between two statically evaluable expression.}\\
\end{array}
\]

\[
\begin{array}{rcl}
& & \ASTComment{Constraints on bitvector lengths.}\\
\hline
\bitsconstraint &::=& \texttt{BitWidth\_SingleExpr}(\expr) \\
  & & \ASTComment{Statically evaluable expression.}\\
  &|& \texttt{BitWidth\_ConstrainedFormType}(\ty) \\
  & & \ASTComment{Constrained by the domain of another type.}\\
  &|& \texttt{BitWidth\_Constraints}(\intconstraint^{*}) \\
  & & \ASTComment{Constrained directly by a constraint on its width.}\\
\end{array}
\]

\[
\begin{array}{rcl}
& & \ASTComment{Represent static slices on a given bitvector type.}\\
\hline
\bitfield &::=& \texttt{BitField\_Simple}(\identifier, \slice^{*}) \\
  & & \ASTComment{A name and its corresponding slice.}\\
  &|& \texttt{BitField\_Nested}(\identifier, \slice^{*}, \bitfield^{*}) \\
  & & \ASTComment{A name, its corresponding slice and some nested bitfields.}\\
  &|& \texttt{BitField\_Type}(\identifier, \slice^{*}, \ty) \\
  & & \ASTComment{A name, its corresponding slice, and the type of the bitfield.}\\
\end{array}
\]

\[
\begin{array}{rcl}
\Field &::=& (\identifier, \ty)\\
  & & \ASTComment{A field of a record-like structure.}\\
\typedidentifier &::=& (\identifier, \ty)\\
  & & \ASTComment{An identifier declared with its type.}\\
\end{array}
\]

\[
\begin{array}{rcl}
& & \ASTComment{Type of left-hand side of assignments.}\\
\hline
\lexpr &::=& \texttt{LE\_Discard}\\
  & & \ASTComment{\texttt{"-"}}\\
  &|& \texttt{LE\_Var}(\identifier)\\
  &|& \texttt{LE\_Slice}(\lexpr, \slice^*)\\
  &|& \texttt{LE\_SetArray}(\lexpr, \expr)\\
  &|& \texttt{LE\_SetField}(\lexpr, \identifier)\\
  &|& \texttt{LE\_SetFields}(\lexpr, \identifier^*)\\
  &|& \texttt{LE\_Destructuring}(\lexpr^*)\\
  &|& \texttt{LE\_Concat}(\lexpr^*, \texttt{<int\_lit>}^*?)\\
  & & \ASTComment{\texttt{LE\_Concat}$(\texttt{les}, \_)$ unpacks the various lexpr.}\\
  & & \ASTComment{The second argument is a type annotation.}\\
\end{array}
\]

\[
\begin{array}{rcl}
\localdeclkeyword &::=& \texttt{LDK\_Var} \;|\; \texttt{LDK\_Constant} \;|\; \texttt{LDK\_Let}\\
\end{array}
\]

\[
\begin{array}{rcl}
\localdeclitem &::=& \texttt{LDI\_Var}(\identifier, \ty?)\\
  &|& \texttt{LDI\_Discard}(\ty?)\\
  &|& \texttt{LDI\_Tuple}(\localdeclitem^*, \ty?)\\
\fordirection &::=& \texttt{Up} \;|\; \texttt{Down}\\
\end{array}
\]

\[
\begin{array}{rcl}
\version &::=& \texttt{V0} \;|\; \texttt{V1}\\
\end{array}
\]

\[
\begin{array}{rcl}
\stmt &::=& \texttt{S\_Pass}\\
  &|& \texttt{S\_Seq}(\stmt, \stmt)\\
  &|& \texttt{S\_Decl}(\localdeclkeyword, \localdeclitem, \expr?)\\
  &|& \texttt{S\_Assign}(\lexpr, \expr, \version)\\
  &|& \texttt{S\_Call}(\identifier, \expr^*, (\identifier, \expr)^*)\\
  &|& \texttt{S\_Return}(\expr?)\\
  &|& \texttt{S\_Cond}(\expr, \stmt, \stmt)\\
  &|& \texttt{S\_Case}(\expr, \casealt^*)\\
  &|& \texttt{S\_Assert}(\expr)\\
  &|& \texttt{S\_For}(\identifier, \expr, \fordirection, \expr, \stmt)\\
  &|& \texttt{S\_While}(\expr, \stmt)\\
  &|& \texttt{S\_Repeat}(\stmt, \expr)\\
  &|& \texttt{S\_Throw}((\expr, \ty?)?)\\
  & & \ASTComment{The $\ty?$ is a type annotation added by the type-checker}\\
  & & \ASTComment{to be matched later with the catch guards.}\\
  & & \ASTComment{The outer option is to represent the implicit throw,}\\
  & & \ASTComment{such as [throw;].}\\
  &|& \texttt{S\_Try}(\stmt, \catcher^*, \stmt?)\\
  & & \ASTComment{The optional stmt is for the otherwise guard.}\\
  &|& \texttt{S\_Debug}(\expr)
\end{array}
\]

\[
\begin{array}{rcl}
\casealt &::=& (\pattern, \stmt)\\
\catcher &::=& (\identifier?, \ty, \stmt)\\
  & & \ASTComment{The optional name of the matched exception,}\\
  & & \ASTComment{the guard type and the statement}\\
  & & \ASTComment{to be executed if the guard matches.}\\
\end{array}
\]

\[
\begin{array}{rcl}
\subprogramtype &::=& \texttt{ST\_Procedure} \;|\; \texttt{ST\_Function} \;|\; \texttt{ST\_Getter} \;|\; \texttt{ST\_Setter}\\
\subprogrambody &::=& \texttt{SB\_ASL}(\stmt)\\
  &|& \texttt{SB\_Primitive}\\
\end{array}
\]

\[
\begin{array}{rrl}
\func &::= \{& \\
 & & \text{name} : \identifier, \\
 & & \text{parameters} : (\identifier, \ty?)^*,\\
 & & \text{args} : \typedidentifier^*,\\
 & & \text{body} : \subprogrambody,\\
 & & \text{return\_type} : \ty?,\\
 & & \text{subprogram\_type} : \subprogramtype\\
 & \} &
\end{array}
\]

\[
\begin{array}{rcl}
\globaldeclkeyword &::=& \texttt{GDK\_Constant} \;|\; \texttt{GDK\_Config} \;|\; \texttt{GDK\_Let} \;|\; \texttt{GDK\_Var}\\
  & & \ASTComment{Declaration keyword for global storage elements.}\\
\end{array}
\]

\[
\begin{array}{rrl}
\globaldecl &::= \{& \\
 & & \text{keyword} : \globaldeclkeyword, \\
 & & \text{name} : \identifier,\\
 & & \text{ty} : \ty?,\\
 & & \text{initial\_value} : \expr?\\
 & \} &
\end{array}
\]

\[
\begin{array}{rcl}
\decl &::=& \texttt{D\_Func}(\func)\\
  & & \texttt{D\_GlobalStorage}(\globaldecl)\\
  & & \texttt{D\_TypeDecl}(\identifier, \ty, (\identifier, \Field^*)?)\\
  & & \ASTComment{The last component is for \texttt{with} fields.}\\
\program &::=& \decl^*\\
\end{array}
\]

\end{document}
