\documentclass{book}
\usepackage{amsmath}  % Classic math package
\usepackage{amssymb}  % Classic math package
\usepackage{mathtools}  % Additional math package
\usepackage{graphicx}  % For figures
\usepackage{caption}  % For figure captions
\usepackage{subcaption}  % For subfigure captions
\usepackage{url}  % Automatically escapes urls
\usepackage{hyperref}  % Insert links inside pdfs
\usepackage[inline]{enumitem}  % For inline lists
\usepackage[export]{adjustbox}  % For centering too wide figures
\usepackage{mathpartir}  % For deduction rules and equations paragraphs
\usepackage{comment}
\usepackage{fancyvrb}
\input{ASLSemanticsLines}

\usepackage{enumitem}
\renewlist{itemize}{itemize}{10}
\setlist[itemize,1]{label=\textbullet}
\setlist[itemize,2]{label=\textasteriskcentered}
\setlist[itemize,3]{label=\textendash}
\setlist[itemize,4]{label=$\triangleright$}
\setlist[itemize,5]{label=+}
\setlist[itemize,6]{label=\textbullet}
\setlist[itemize,7]{label=$\ast$}
\setlist[itemize,8]{label=\textendash}
\setlist[itemize,9]{label=$\triangleright$}
\setlist[itemize,10]{label=+}

%Macros
\newcommand\ie{i.\,e.}
\newcommand\eg{e.\,g.}
\newcommand\synor{\ |\ }
\newcommand\syntt[1]{\mathtt{#1}}
\newcommand\ife[3]{\text{if}\ #1\ \text{then}\ #2\ \text{else}\ #3\ \text{end}}
\newcommand\inenv[2]{\left\langle #1, #2 \right\rangle}
\newcommand\env[1]{\left\langle #1 \right\rangle}
\newcommand\reducesto{\ \to\ }
\newcommand\llbracket{[|}
\newcommand\rrbracket{|]}
\newcommand\interp[1]{\left\llbracket #1 \right\rrbracket}
\newcommand\st[0]{\ \middle|\ }
\newcommand\field[1]{.\text{#1}}
\newcommand\globals[0]{\field{globals}}
\newcommand\locals[0]{\field{locals}}
\newcommand\X[0]{\mathcal{X}}
\newcommand\N[0]{\mathbb{N}}
\newcommand\asldata[0]{\mathtt{asl\_data}}
\newcommand\aslctrl[0]{\mathtt{asl\_ctrl}}
\newcommand\aslpo[0]{\mathtt{asl\_po}}
\DeclareMathOperator{\dom}{dom}

\author{Arm Architecture Technology Group}
\title{ASL Semantics Reference}
\begin{document}
\maketitle
\tableofcontents{}

\chapter{Copyright and Disclaimer}

The material in this document is copyright Arm Ltd 2023, all rights reserved.

This material covers both ASLv0 (viz, the existing ASL pseudocode language
which appears in the Arm Architecture Reference Manual) and ASLv1, a new,
experimental, and as yet unreleased version of ASL.

This material is work in progress, more precisely at pre-Alpha quality as
per Arm’s quality standards. In particular, this means that it would be
premature to base any production tool development on this material.

However, any feedback, question, query and feature request would be most
welcome; those can be sent to Arm’s Architecture Formal Team Lead Jade Alglave
\texttt{(jade.alglave@arm.com)} or by raising issues or PRs to the herdtools7
github repository.

\chapter{Preamble}

The semantics of ASL are embodied in a function that takes as input an ASL program
and returns all possible behaviours of that program.

\section{Environments}

An environment is what the semantics operates over: a structure
which amongst other things associates values to variables.
Intuitively, the evaluation of a program makes an initial
environment evolve, with new values as given by the operations
of the program. Environments map names to variables and subprograms. 

\section{Variables}
Variables have two different possible scopes: global to the whole program which
come with an initial value and cannot be declared inside a subprogram, or local
to a subprogram. Global variables are initialised at the start of the evaluation
of the program.  

\section{Functions}
Functions are declared statically: they are declared before the start of the
program and are not values, thus a subprogram call uses directly a function name. 

Formally, an environment $E \in \mathcal{E}$ is a pair of a binding from global
variable names to their value, and a binding from local variable names to their
values:

%
\[
  \mathcal{E} = \left( \X \hookrightarrow \mathcal{V} \right)
         % \times \left( \X^\text{F} \hookrightarrow {\X^\text{L}}^* \times \mathcal{S} \right)
         \times \left( \X \hookrightarrow \mathcal{V} \right)
\]

We define some notations, for an environment $E = (G, F, L)$, a variable $x$
and a value $v$:
%
\begin{mathpar}
  E\globals \triangleq G
  \and
  % E^\text{F} \triangleq F
  % \and
  E\locals \triangleq L
  \and
  \text{Env}(G, L) \triangleq E
  \and
  E[x \mapsto v] \triangleq
    \left\{
      \begin{array}{ll}
        (G[x \mapsto v], L) &\text{if}\ x \in \dom (G) \\
        (G, L[x \mapsto v]) &\text{otherwise} \\
      \end{array}
    \right.
\end{mathpar}
%
The notation $E\globals$ refers to the global part of the environment, and
$E\locals$ to the local variable mappings, and $\text{Env} (G, L)$ is the
environment formed with the global mappings in $G$ and the local from $L$.
%
The notation $E[x \mapsto v]$ is the environment $E$ modified so $x$ is bound
to $v$.
%
Furthermore, $v?$ refers to $v$ or $\bot$, and the usage of $v$ without a
question mark implies that $v \neq \bot$.
%
In particular, the over-writing of a variable in an environment depends on
whether the variable is a global or local variable.

\section{Side-effects}

Side-effects can arise due to:
\begin{itemize}
\item Reads and Writes to global variables;
\item Exceptions;
\item Calls to primitives.
\end{itemize}

\section{Evaluation}

Evaluating a program is evaluating its ``main'' subprogram.
Constructively, evaluating a program requires following its
Abstract Syntax Tree and evaluating each of its components.

The semantics of a program are given by applying a set of
\texttt{eval\_<object>} functions. Each \texttt{eval\_<object>} function
describes how to evaluate a specific object, as follows.
\begin{itemize}
\item \texttt{eval\_expr} evaluates expressions: it takes an
environment and an expression and returns a value and a
new environment (see Chapter~\ref{chap:eval_expr});
\item \texttt{eval\_lexpr} evaluates left-hand sides of assignments: it takes
an environment, the left-hand side of an assignment and a value to be written,
and returns a environment updated with the new value (see
Chapter~\ref{chap:eval_lexpr});
\item \texttt{eval\_slices} evaluates slices (see Chapter~\ref{chap:eval_slices});
\item \texttt{eval\_patterns} evaluates patterns (see Chapter~\ref{chap:eval_pattern});
\item \texttt{eval\_local\_decl} evaluates local declarations (see Chapter~\ref{chap:eval_local_decl});
\item \texttt{eval\_stmt} evaluates statements: it takes an
environment and a statement and returns a new
environment, viz, the environment updated with the side-effects
of the statement (see Chapter~\ref{chap:eval_stmt});
\item \texttt{eval\_block} evaluates blocks (see Chapter~\ref{chap:eval_block});
\item \texttt{eval\_loop} evaluates both while and repeat loops (see Chapter~\ref{chap:eval_loop});
\item \texttt{eval\_for} evaluates for loops (see Chapter~\ref{chap:eval_for});
\item \texttt{eval\_catchers} evaluates catchers (see Chapter~\ref{chap:eval_catchers});
\item \texttt{eval\_func} evaluates subprograms: it takes an
environment, a subprogram name and its arguments, and returns a
list of the return values of the subprogram (see Chapter~\ref{chap:eval_func}). 
\end{itemize}

\section{Reading guide}

The definition of each \texttt{eval\_<object>} function is given by a number of
rules, which follow the possible shapes the \texttt{object} can have. For
example, an expression can be a literal, or a binary operator, amongst other
things. Each of those has its own evaluation rule: SemanticsRule.Lit in
Section~\ref{sec:SemanticsRule.Lit}, Semantics.Binop in
Section~\ref{sec:SemanticsRule.Binop} respectively.

Each rule is presented using the following template:
\begin{itemize}
\item a Prose paragraph gives the rule in English, and corresponds as much as possible to the code of the reference interpreter ASLRef;
\item one or several Examples, which as much as possible are also given as regression tests in \url{~/herdtools7/asllib/tests/ASLSemanticsReference.t} 
\item a Code paragraph which gives a verbatim of the corresponding implementation in the code of the reference interpreter ASLRef;
\item a Formal paragraph which gives a formal definition of the rule.
\end{itemize}
%
Formally, the semantics of different language constructs are given by the functions
$\interp{\cdot}_F$ that map an environment to a set of pair of values and
environments, where $F$ is a given set of ASL subprograms.
%
In the following, for simplicity, we will omit the indices $F$ from the
semantics definitions, apart when defined, as they are constant in the whole
evaluation.
%
Furthermore, the variable names inside set definitions are implicitly
quantified. For example the two following definitions are equivalent:
%
\begin{gather*}
  X + Y \triangleq \left\{ n \st{} n_1 \in X \ \text{and}\ n_2 \in Y \text{and}\ n = n_1 + n_2 \right\}
  \\
  X + Y \triangleq \left\{ n \st{} \exists n_1 \in X, \exists n_2 \in Y, n = n_1 + n_2 \right\}
\end{gather*}


\chapter{Evaluation of Expressions \label{chap:eval_expr}}

\texttt{eval\_expr} specifies how to evaluate an expression \texttt{e} in an
environment \texttt{env}. 

Evaluation of the expression \texttt{e} under an environment \texttt{env} is
either a value \texttt{v} together with a potentially updated
environment \texttt{new\_env}, or an error, and one of the following applies:
\begin{itemize}
\item SemanticsRule.Lit (see Section~\ref{sec:SemanticsRule.Lit});
\item SemanticsRule.ELocalVar (see Section~\ref{sec:SemanticsRule.ELocalVar})
\item SemanticsRule.EGlobalVar (see Section~\ref{sec:SemanticsRule.EGlobalVar})
\item SemanticsRule.EUndefIdent (see Section~\ref{sec:SemanticsRule.EUndefIdent})
\item SemanticsRule.BinopAnd (see Section~\ref{sec:SemanticsRule.BinopAnd})
\item SemanticsRule.BinopOr (see Section~\ref{sec:SemanticsRule.BinopOr})
\item SemanticsRule.BinopImpl (see Section~\ref{sec:SemanticsRule.BinopImpl})
\item SemanticsRule.Binop (see Section~\ref{sec:SemanticsRule.Binop})
\item SemanticsRule.Unop (see Section~\ref{sec:SemanticsRule.Unop})
\item SemanticsRule.ECond (see Section~\ref{sec:SemanticsRule.ECond})
\item SemanticsRule.ESlice (see Section~\ref{sec:SemanticsRule.ESlice})
\item SemanticsRule.ECall (see Section~\ref{sec:SemanticsRule.ECall})
\item SemanticsRule.EGetArray (see Section~\ref{sec:SemanticsRule.EGetArray})
\item SemanticsRule.ERecord (see Section~\ref{sec:SemanticsRule.ERecord})
\item SemanticsRule.EGetField (see Section~\ref{sec:SemanticsRule.EGetField})
\item SemanticsRule.EConcat (see Section~\ref{sec:SemanticsRule.EConcat})
\item SemanticsRule.ETuple (see Section~\ref{sec:SemanticsRule.ETuple})
\item SemanticsRule.EUnknown (see Section~\ref{sec:SemanticsRule.EUnknown})
\item SemanticsRule.EPattern (see Section~\ref{sec:SemanticsRule.EPattern})
\item SemanticsRule.CTC (see Section~\ref{sec:SemanticsRule.CTC})
\end{itemize}


\section{SemanticsRule.Lit \label{sec:SemanticsRule.Lit}}

  \subsection{Prose}
  Evaluation of the expression \texttt{e} under environment \texttt{env} is
  \texttt{(res,new\_env)} and all of the following apply:
  \begin{itemize}
  \item \texttt{e} denotes a Literal \texttt{l};
  \item \texttt{res} is a value \texttt{v};
  \item \texttt{v} is the value of \texttt{l};
  \item \texttt{new\_env} is \texttt{env}.
  \end{itemize}

  \subsection{Example}
  The expression [3] evaluates to the value [3].

  \subsubsection{Code}
  \VerbatimInput[firstline=\LitBegin, lastline=\LitEnd]{../Interpreter.ml}

  \subsection{Formally: sequential case}
  The evaluation of a literal value $v$ associates any environment to itself
  coupled with $v$: 

  \begin{align}
  \interp{v}(\texttt{env}) & \triangleq \left\{ (v, \texttt{env}) \right\}
  \label{eq:sem-seq-lit}
  \end{align}

  \subsection{Formally: concurrent case}
  \begin{align}
  \interp{v}(\texttt{env}) & \triangleq \left\{ (v, \texttt{env}, \varnothing) \right\}
  \label{eq:sem-conc-lit}
  \end{align}

  \subsection{Comments}


\section{SemanticsRule.ELocalVar \label{sec:SemanticsRule.ELocalVar}}

  \subsection{Prose}
  Evaluation of the expression \texttt{e} under environment \texttt{env} is
  \texttt{(res,new\_env)} and all of the following apply:
  \begin{itemize}
  \item \texttt{e} denotes a variable \texttt{x} which is bound locally in \texttt{env};
  \item \texttt{res} is a value \texttt{v};
  \item \texttt{v} is the value of \texttt{x} in \texttt{env};
  \item \texttt{new\_env} is \texttt{env}.
  \end{itemize}

  \subsection{Example: SemanticsRuleELocalVar.asl}
    In the program:
    \VerbatimInput{../tests/ASLSemanticsReference.t/SemanticsRule.ELocalVar.asl}
    the evaluation of \texttt{x} within \texttt{assert x == 3;} uses SemanticsRule.ELocalVar.

  \subsubsection{Code}
  \VerbatimInput[firstline=\ELocalVarBegin, lastline=\ELocalVarEnd]{../Interpreter.ml}

  \subsection{Formally: sequential case}
  The evaluation of a variable $x$ in an environment \texttt{env} that maps $x$ to $v$
  is the tuple $(v,\texttt{env})$: 
  \begin{align}
  \interp{x}(\texttt{env}) & \triangleq \left\{ (\texttt{env}[x], \texttt{env})) \right\}
  \label{eq:sem-seq-var}
  \end{align} 
 
  \subsection{Formally: concurrent case}
  \begin{align}
  \interp{x}(\texttt{env}) & \triangleq \left\{ (\texttt{env}[x], \texttt{env}, \text{R} (x)) \right\}
  \label{eq:sem-conc-var}
  \end{align} 
 
  \subsection{Comments}
  When there exists a global variable called \texttt{x} then an ASL program
  cannot use the name \texttt{x} as a local variable. ASLRef checks that this
  property, sometimes called ``banning shadowing'', is true at type-checking.

\section{SemanticsRule.EGlobalVar \label{sec:SemanticsRule.EGlobalVar}}

  \subsection{Prose}

  Evaluation of the expression \texttt{e} under environment \texttt{env} is
  \texttt{(res,new\_env)} and all of the following apply:
  \begin{itemize}
  \item \texttt{e} denotes a variable \texttt{x} which is bound globally in \texttt{env};
  \item \texttt{res} is a value \texttt{v};
  \item \texttt{v} is the value of \texttt{x} in \texttt{env};
  \item \texttt{new\_env} is \texttt{env}.
  \end{itemize}

  \subsection{Example: SemanticsRuleEGlobalVar.asl}
    In the program:
    \VerbatimInput{../tests/ASLSemanticsReference.t/SemanticsRule.EGlobalVar.asl}
    the evaluation of \texttt{global\_x} within \texttt{assert global\_x == 3;}
    uses SemanticsRule.EGlobalVar.

  \subsection{Example: SemanticsRuleEGlobalVarError.asl}
    The program:
    \VerbatimInput{../tests/ASLSemanticsReference.t/SemanticsRule.EGlobalVarError.asl} 
    raises a type error because \texttt{x} is already declared locally in
    \texttt{stdlib.asl}.

  \subsubsection{Code}
  \VerbatimInput[firstline=\EGlobalVarBegin, lastline=\EGlobalVarEnd]{../Interpreter.ml}

  \subsection{Formally: sequential case}
  \begin{align}
  \interp{x}(\texttt{env}) & \triangleq \left\{ (\texttt{env}[x], \texttt{env})) \right\}
  \label{eq:sem-seq-var}
  \end{align} 
 
  \subsection{Formally: concurrent case}
  \begin{align}
  \interp{x}(\texttt{env}) & \triangleq \left\{ (\texttt{env}[x], \texttt{env}, \text{R} (x)) \right\}
  \label{eq:sem-conc-var}
  \end{align} 

  \subsection{Comments}

\section{SemanticsRule.EUndefIdent \label{sec:SemanticsRule.EUndefIdent}}

  \subsection{Prose}

  Evaluation of the expression \texttt{e} under environment \texttt{env} is
  \texttt{res} and all of the following apply:
  \begin{itemize}
  \item \texttt{e} denotes a variable \texttt{x} which is not bound in \texttt{env};
  \item \texttt{res} is an ``\texttt{Undefined Identifier}'' error.
  \end{itemize}

  \subsection{Example: SemanticsRule.EUndefIdent.asl}
    The program:
    \VerbatimInput{../tests/ASLSemanticsReference.t/SemanticsRule.EUndefIdent.asl} 
    raises an \texttt{Undefined Identifier} error: the variable \texttt{y} is
    undefined in the environment $\mathtt{x} \mapsto \mathtt{42}$.

  \subsubsection{Code}
  \VerbatimInput[firstline=\EUndefIdentBegin, lastline=\EUndefIdentEnd]{../Interpreter.ml}

  \subsection{Formally: sequential case}
  \begin{align}
  \end{align} 

  \subsection{Formally: concurrent case}
  \begin{align}
  \end{align} 

  \subsection{Comments}

\section{SemanticsRule.BinopAnd \label{sec:SemanticsRule.BinopAnd}}

  \subsection{Prose}

  Evaluation of the expression \texttt{e} under environment \texttt{env} is
  \texttt{(res,new\_env)} and all of the following apply:
  \begin{itemize}
  \item \texttt{e} denotes a conjunction over two expressions \texttt{e1} and \texttt{e2};
  \item \texttt{(res,new\_env)} is the result of the evaluation of \texttt{if
e1 then e2 else false} (see Section~\ref{sec:SemanticsRule.ECond}).
  \end{itemize}

  \subsection{Examples}

  \subsubsection{Code}
  \VerbatimInput[firstline=\BinopAndBegin, lastline=\BinopAndEnd]{../Interpreter.ml}

  \subsection{Formally: sequential case}
  \begin{align}
  \end{align} 

  \subsection{Formally: concurrent case}
  \begin{align}
  \end{align} 

  \subsection{Comments}
  This aims to encompass LRM Section 7.5.3 R\_BKNT, Section 8.5.1 I\_QRXP and Section
  8.6.1.

\section{SemanticsRule.BinopOr \label{sec:SemanticsRule.BinopOr}}
  \subsection{Prose}

  Evaluation of the expression \texttt{e} under environment \texttt{env} is
  \texttt{(res,new\_env)} and all of the following apply:
  \begin{itemize}
  \item \texttt{e} denotes a disjunction over two expressions \texttt{e1} and \texttt{e2};
  \item \texttt{(res,new\_env)} is the result of the evaluation of \texttt{if 
e1 then true else e2} (see Section~\ref{sec:SemanticsRule.ECond}).
  \end{itemize}

  \subsection{Examples}

  \subsubsection{Code}
  \VerbatimInput[firstline=\BinopOrBegin, lastline=\BinopOrEnd]{../Interpreter.ml}

  \subsection{Formally: sequential case}
  \begin{align}
  \end{align} 

  \subsection{Formally: concurrent case}
  \begin{align}
  \end{align} 

  \subsection{Comments}
  This aims to encompass LRM Section 7.5.3 R\_BKNT, Section 8.5.1 I\_QRXP and Section
  8.6.1.

\section{SemanticsRule.BinopImpl \label{sec:SemanticsRule.BinopImpl}}

  \subsection{Prose}
  All of the following apply:
  \begin{itemize}
  \item \texttt{e} denotes an implication over two expressions \texttt{e1} and \texttt{e2};
  \item \texttt{e} is evaluated as \texttt{if e1 then e2 else true}.
  \end{itemize}

  \subsection{Examples}

  \subsubsection{Code}
  \VerbatimInput[firstline=\BinopImplBegin, lastline=\BinopImplEnd]{../Interpreter.ml}

  \subsection{Comments}
  This aims to encompass LRM Section 7.5.3 R\_BKNT, Section 8.5.1 I\_QRXP and Section
  8.6.1.

\section{SemanticsRule.Binop \label{sec:SemanticsRule.Binop}}

  \subsection{Prose}
  All of the following apply:
  \begin{itemize}
  \item \texttt{e} denotes a Binary Operator \texttt{op} over two expressions \texttt{e1} and \texttt{e2};
  \item The evaluation of the expression \texttt{e1} under \texttt{env} is \texttt{v1};
  \item The evaluation of the expression \texttt{e2} under \texttt{env} is \texttt{v2};
  \item \texttt{v} is the result of applying the Binary Operator \texttt{op} to \texttt{v1} and \texttt{v2}.
  \end{itemize}

  \subsection{Example: SemanticsRule.EBinopPlusAssert.asl}
    In this program:
    \VerbatimInput{../tests/ASLSemanticsReference.t/SemanticsRule.EBinopPlusAssert.asl}
    the expression \texttt{3 + 2} evaluates to the value \texttt{5}.

  \subsection{Example: SemanticsRule.EDIVBackendDefinedError.asl}
    In the program:
    \VerbatimInput{../tests/ASLSemanticsReference.t/SemanticsRule.EBinopDIVBackendDefinedError.asl}
    \begin{itemize}
    \item in ASLv0, the expression \texttt{3 DIV 0} raises a backend-defined
      error, e.g.
      \texttt{ASL Execution error: Illegal application of operator DIV for values 3 and 0.}
    \item in ASLv1, the expression \texttt{3 DIV 0} raises a type error.
    \end{itemize}

  \subsubsection{Code}
  \VerbatimInput[firstline=\BinopBegin, lastline=\BinopEnd]{../Interpreter.ml}

  \subsection{Formally: sequential case}
  
  \begin{align}
  \interp{e_1 + e_2} (E) & \triangleq
    \left\{ (v, E'') \st{}
    \begin{aligned}
      & v_1 + v_2 = v
      \\ \text{and}\ &
      (v_1, E') \in{} \interp{e_1} (E)
      \\ \text{and}\ &
      (v_2, E'') \in{} \interp{e_2} (E')
    \end{aligned}
    \right\}
  \label{eq:sem-seq-plus}
  \end{align} 

  \subsection{Formally: concurrent case}
  For the ordering constraints, the different arguments of a n-ary subprogram
are considered computed in parallel:
  \begin{align}
    \interp{e_1 + e_2} (E) & \triangleq
    \left\{ (v, E'', S_1 \parallel S_2) \st{}
    \begin{aligned}
      & v_1 + v_2 = v
      \\ \text{and}\ &
      (v_1, E', S_1) \in{} \interp{e_1} (E)
      \\ \text{and}\ &
      (v_2, E'', S_2) \in{} \interp{e_2} (E')
    \end{aligned}
    \right\}
  \label{eq:sem-conc-plus}
  \end{align} 

  \subsection{Comments}
  This aims to encompass LRM Section 7.5.3 R\_BKNT.

\section{SemanticsRule.Unop \label{sec:SemanticsRule.Unop}}

  \subsection{Prose}
  All of the following apply:
  \begin{itemize}
  \item \texttt{e} denotes a Unary Operator \texttt{op} over an expression \texttt{e'} in an environment \texttt{env};
  \item The evaluation of the expression \texttt{e'} under \texttt{env} is \texttt{v',new\_env};
  \item \texttt{v, new\_env} is the result of applying the Unary Operator \texttt{op} to \texttt{v'}.
  \end{itemize}

  \subsection{Example: SemanticsRule.EUnopAssert.asl}
    In the program:
    \VerbatimInput{../tests/ASLSemanticsReference.t/SemanticsRule.EUnopAssert.asl}
    the expression \texttt{NOT '1010'} evaluates to the value \texttt{'0101'}.

  \subsubsection{Code}
  \VerbatimInput[firstline=\UnopBegin, lastline=\UnopEnd]{../Interpreter.ml}

  \subsection{Formally: sequential case}
  \begin{align}
  \end{align} 

  \subsection{Formally: concurrent case}
  \begin{align}
  \end{align} 

  \subsection{Comments}

\section{SemanticsRule.ECond \label{sec:SemanticsRule.ECond}}

  \subsection{Prose}
  All of the following apply:
  \begin{itemize}
  \item \texttt{e} denotes a conditional expression \texttt{e\_cond} with two options \texttt{e1} and \texttt{e2};
  \item The evaluation of the conditional expression \texttt{e\_cond} under \texttt{env} is \texttt{m\_cond};
  \item The evaluation of \texttt{e1} or \texttt{e2}, depending on \texttt{m\_cond}, is \texttt{v}.
  \end{itemize}

  \subsection{Example: SemanticsRule.ECondFalse.asl}
    In the program:
    \VerbatimInput{../tests/ASLSemanticsReference.t/SemanticsRule.ECondFalse.asl}
    the expression \texttt{if FALSE then Return42() else 3} evaluates to the value \texttt{3}.

  \subsection{Example: SemanticsRule.ECondUNKNOWN3or42.asl}
    In the program:
    \VerbatimInput{../tests/ASLSemanticsReference.t/SemanticsRule.ECondUNKNOWN3or42.asl}
the expression \texttt{if UNKNOWN: boolean then 3 else Return42()} will
evaluate either \texttt{3} or \texttt{Return42()} depending on how
\texttt{UNKNOWN} is implemented.

  \subsubsection{Code}
  \VerbatimInput[firstline=\ECondBegin, lastline=\ECondEnd]{../Interpreter.ml}

  \subsection{Formally: sequential case}
  \begin{align}
  \end{align} 

  \subsection{Formally: concurrent case}
  \begin{align}
  \end{align} 

  \subsection{Comments}
  This aims to encompass LRM Section 5.1 R\_YCDB.

\section{SemanticsRule.ESlice \label{sec:SemanticsRule.ESlice}}

  \subsection{Prose}
  All of the following apply:
  \begin{itemize}
  \item \texttt{e} denotes an expression \texttt{e\_bv} sliced as per \texttt{slices};
  \item The evaluation of \texttt{e\_bv} under \texttt{env} is \texttt{v\_bv};
  \item The evaluation of \texttt{slices} under \texttt{env} is \texttt{positions};
  \item \texttt{v} is the value read in \texttt{v\_bv} from \texttt{positions}.
  \end{itemize}

  \subsection{Example: SemanticsRule.ESlice.asl}
    In the program:
    \VerbatimInput{../tests/ASLSemanticsReference.t/SemanticsRule.ESlice.asl}    
    the expression \texttt{'11110000'[5:2]} evaluates to the value \texttt{'1100'}.

  \subsubsection{Code}
  \VerbatimInput[firstline=\ESliceBegin, lastline=\ESliceEnd]{../Interpreter.ml}

  \subsection{Formally: sequential case}
  \begin{align}
  \end{align} 

  \subsection{Formally: concurrent case}
  \begin{align}
  \end{align} 

  \subsection{Comments}

\section{SemanticsRule.ECall \label{sec:SemanticsRule.ECall}}

  \subsection{Prose}

  Evaluation of the expression \texttt{e} under environment \texttt{env} is
  \texttt{(res,new\_env)} and all of the following apply:
  \begin{itemize}
  \item \texttt{e} denotes a subprogram call \texttt{(name, params, actual\_args)};
  \item The evaluation of that subprogram call under \texttt{env} is \texttt{ms};
  \item \texttt{res} is a value \texttt{v};
  \item \texttt{v} is the value read from \texttt{ms};
  \item \texttt{new\_env} is \texttt{env}.
  \end{itemize}

  \subsection{Example: SemanticsRule.ECall.asl}
    In the program:
    \VerbatimInput{../tests/ASLSemanticsReference.t/SemanticsRule.ECall.asl}
    the expression \texttt{Return42()} evaluates to the value \texttt{42} because the
    subprogram \texttt{Return42()} is implemented to return the value \texttt{42}.

  \subsubsection{Code}
  \VerbatimInput[firstline=\ECallBegin, lastline=\ECallEnd]{../Interpreter.ml}

  \subsection{Formally: sequential case}
  \begin{align}
  \end{align} 

  \subsection{Formally: concurrent case}
  \begin{align}
  \end{align} 

  \subsection{Comments}

\section{SemanticsRule.EGetArray \label{sec:SemanticsRule.EGetArray}}

  \subsection{Prose}

  Evaluation of the expression \texttt{e} under environment \texttt{env} is
  such that all of the following apply:
  \begin{itemize}
  \item \texttt{e} denotes an array \texttt{e\_array} and an index \texttt{e\_index};
  \item The evaluation of \texttt{e\_array} under \texttt{env} is \texttt{v\_array};
  \item The evaluation of \texttt{e\_index} under \texttt{env} is \texttt{v\_index};
  \item One of the following applies:
        \begin{itemize}
        \item All of the following apply:
          \begin{itemize}
          \item the result of evaluation of \texttt{e} under \texttt{env} is \texttt{(res, env)};
          \item \texttt{res} is a value \texttt{v};
          \item \texttt{v} is the value found at the index \texttt{v\_index} of \texttt{v\_array};
          \item \texttt{new\_env} is \texttt{env};
          \end{itemize}

        \item \texttt{res} is a typing error.
        \end{itemize}
  \end{itemize}

  \subsection{Example: SemanticsRule.EGetArray.asl}
    In the program:
    \VerbatimInput{../tests/ASLSemanticsReference.t/SemanticsRule.EGetArray.asl}
    the expression \texttt{my\_array[2]} evaluates to the value \texttt{42} since the element
    indexed by \texttt{2} in \texttt{my\_array} is \texttt{42}.

  \subsection{Example: SemanticsRule.EGetArrayTooSmall.asl}
    The program:
    \VerbatimInput{../tests/ASLSemanticsReference.t/SemanticsRule.EGetArrayTooSmall.asl} 
    raises a typing error since we are trying to access index \texttt{3} of an array
    which has indexes \texttt{0}, \texttt{1} and \texttt{2} only.

  \subsubsection{Code}
  \VerbatimInput[firstline=\EGetArrayBegin, lastline=\EGetArrayEnd]{../Interpreter.ml}


  \subsection{Formally: sequential case}
  \begin{align}
  \end{align} 

  \subsection{Formally: concurrent case}
  \begin{align}
  \end{align} 

  \subsection{Comments}

\section{SemanticsRule.ERecord \label{sec:SemanticsRule.ERecord}}

  \subsection{Prose}

  Evaluation of the expression \texttt{e} under environment \texttt{env} is
  \texttt{(res,new\_env)} and all of the following apply:
  \begin{itemize}
  \item \texttt{e} denotes a record value \texttt{names} and associated expressions
    \texttt{fields};
  \item \texttt{fields} evaluates in \texttt{env} to \texttt{v\_fields};
  \item \texttt{res} is a value \texttt{v};
  \item \texttt{v} is the record built by associating the names \texttt{names} to \texttt{v\_fields}.
  \item \texttt{new\_env} is \texttt{env}.
  \end{itemize}

  \subsection{Example: SemanticsRule.ERecord.asl}
    In the program:
    \VerbatimInput{../tests/ASLSemanticsReference.t/SemanticsRule.ERecord.asl} 
    the expression \texttt{MyRecordType { a=3, b=42 }} evaluates to the value \texttt{{a:
3, b: 42}}.

  \subsubsection{Code}
  \VerbatimInput[firstline=\ERecordBegin, lastline=\ERecordEnd]{../Interpreter.ml}

  \subsection{Formally: sequential case}
  \begin{align}
  \end{align} 

  \subsection{Formally: concurrent case}
  \begin{align}
  \end{align} 

  \subsection{Comments}

\section{SemanticsRule.EGetField \label{sec:SemanticsRule.EGetField}}

  \subsection{Prose}
  Evaluation of the expression \texttt{e} under environment \texttt{env} is
  \texttt{(res,new\_env)} and all of the following apply:
  \begin{itemize}
  \item \texttt{e} denotes a field name \texttt{field\_name} in a record value \texttt{e\_vec};
  \item the evaluation of \texttt{e\_vec} in \texttt{env} is \texttt{v\_vec};
  \item \texttt{res} is a value \texttt{v};
  \item \texttt{v} is the value mapped by \texttt{field\_name} in \texttt{v\_vec};
  \item \texttt{new\_env} is \texttt{env}.
  \end{itemize}

  \subsection{Example: SemanticsRule.ERecord.asl}
    In the program:
    \VerbatimInput{../tests/ASLSemanticsReference.t/SemanticsRule.ERecord.asl}
    the expression \texttt{my\_record.a} evaluates to the value \texttt{3}.

  \subsubsection{Code}
  \VerbatimInput[firstline=\EGetFieldBegin, lastline=\EGetFieldEnd]{../Interpreter.ml}

  \subsection{Formally: sequential case}
  \begin{align}
  \end{align} 

  \subsection{Formally: concurrent case}
  \begin{align}
  \end{align} 

  \subsection{Comments}

\section{SemanticsRule.EConcat \label{sec:SemanticsRule.EConcat}}

  \subsection{Prose}
  Evaluation of the expression \texttt{e} under environment \texttt{env} is
  \texttt{(res,new\_env)} and all of the following apply:
  \begin{itemize}
  \item \texttt{e} denotes a list of bitvector expressions \texttt{e\_list};
  \item the evaluation of \texttt{e\_list} in \texttt{env} is \texttt{v\_list};
  \item \texttt{res} is a value \texttt{v};
  \item \texttt{v} is the concatenation of \texttt{v\_list};
  \item \texttt{new\_env} is \texttt{env}.
  \end{itemize}

  \subsection{Example: SemanticsRule.EConcat}
    In the program:
    \VerbatimInput{../tests/ASLSemanticsReference.t/SemanticsRule.EConcat.asl}
    the expression \texttt{['10', '11']} evaluates to the value \texttt{'1011'}.

  \subsubsection{Code}
  \VerbatimInput[firstline=\EConcatBegin, lastline=\EConcatEnd]{../Interpreter.ml}

  \subsection{Formally: sequential case}
  \begin{align}
  \end{align} 

  \subsection{Formally: concurrent case}
  \begin{align}
  \end{align} 

  \subsection{Comments}

\section{SemanticsRule.ETuple \label{sec:SemanticsRule.ETuple}}

  \subsection{Prose}
  Evaluation of the expression \texttt{e} under environment \texttt{env} is
  \texttt{(res,new\_env)} and all of the following apply:
  \begin{itemize}
  \item \texttt{e} denotes a list of expression \texttt{e\_list};
  \item the evaluation of \texttt{e\_list} in \texttt{env} is \texttt{v\_list};
  \item \texttt{res} is a value \texttt{v};
  \item \texttt{v} is the tuple built from \texttt{v\_list};
  \item \texttt{new\_env} is \texttt{env}.
  \end{itemize}

  \subsection{Example: SemanticsRule.ETuple.asl}
    In the program:
    \VerbatimInput{../tests/ASLSemanticsReference.t/SemanticsRule.ETuple.asl}
    the expression \texttt{(3, Return42())} evaluates to the value \texttt{(3, 42)}.

  \subsubsection{Code}
  \VerbatimInput[firstline=\ETupleBegin, lastline=\ETupleEnd]{../Interpreter.ml}

  \subsection{Formally: sequential case}
  \begin{align}
  \end{align} 

  \subsection{Formally: concurrent case}
  \begin{align}
  \end{align} 

  \subsection{Comments}

\section{SemanticsRule.EUnknown \label{sec:SemanticsRule.EUnknown}}

  \subsection{Domain of a type}

  The domain of a type $t$, written $D(t)$, is defined as follows:
  \[
  D(t) \triangleq \left\{
    \begin{array}{ll}
      \{\text{true}, \text{false} \} & \text{if } t \text{ is boolean}  \\
      \mathbb{Z}                     & \text{if } t \text{ is integer}  \\
      {\{0, 1\}}^n                   & \text{if } t \text{ is bits} (n) \\
    \end{array}
  \right.
  \]

  \subsection{Prose}
  Evaluation of the expression \texttt{e} under environment \texttt{env} is
  \texttt{(res,new\_env)} and all of the following apply:
  \begin{itemize}
  \item \texttt{e} denotes a type \texttt{t};
  \item \texttt{res} is a value \texttt{v};
  \item \texttt{v} is a value in the domain of \texttt{t};
  \item \texttt{new\_env} is \texttt{env}.
  \end{itemize}

  \subsection{Example: SemanticsRule.EUnknownInteger3.asl}
    In the program:
    \VerbatimInput{../tests/ASLSemanticsReference.t/SemanticsRule.EUnknownInteger3.asl}
    the expression \texttt{[UNKNOWN : integer]} evaluates to an integer value.

  \subsection{Example: SemanticsRule.EUnknownIntegerRange3-42-3.asl}

    In the program:
    \VerbatimInput{../tests/ASLSemanticsReference.t/SemanticsRule.EUnknownIntegerRange3-42-3.asl}
    the expression \texttt{UNKNOWN : integer \{3, 42\}} evaluates to either the value
\texttt{3} or the value \texttt{42}.

  \subsubsection{Code}
  \VerbatimInput[firstline=\EUnknownBegin, lastline=\EUnknownEnd]{../Interpreter.ml}

  \subsection{Formally: sequential case}

  \begin{align}  
  \interp{\syntt{unknown:}\ t} (E) & \triangleq \left\{ (v, E) \st{} v \in{} D(t) \right\}
  \label{eq:sem-seq-unknown}  
  \end{align}  

  \subsection{Formally: concurrent case}

  \begin{align}
  \interp{\syntt{unknown:} t} (E) & \triangleq \left\{ (v, E, \varnothing) \st{} v \in{} D(t) \right\}
  \label{eq:sem-conc-unknown}
  \end{align}

  \subsection{Comments}
  This aims to encompass LRM Section 5.12 R\_WLCH.

\section{SemanticsRule.EPattern \label{sec:SemanticsRule.EPattern}}


  \subsection{Prose}
  Evaluation of the expression \texttt{e} under environment \texttt{env} is
  \texttt{(res,new\_env)} and all of the following apply:
  \begin{itemize}
  \item \texttt{e} denotes a pattern \texttt{e,p};
  \item \texttt{res} is a value \texttt{v};
  \item \texttt{v} is the boolean determining whether the evaluation of \texttt{e} in \texttt{env} matches \texttt{p};
  \item \texttt{new\_env} is \texttt{env}.
  \end{itemize}

  \subsection{Example: SemanticsRule.EPatternFALSE.asl}
    In the program:
    \VerbatimInput{../tests/ASLSemanticsReference.t/SemanticsRule.EPatternFALSE.asl}
    the expression \texttt{42 IN \{0..3, -4\}} evaluates to the value \texttt{FALSE}.

  \subsection{Example: SemanticsRule.EPatternTRUE.asl}
    In the program:
    \VerbatimInput{../tests/ASLSemanticsReference.t/SemanticsRule.EPatternTRUE.asl}
    the expression \texttt{42 IN \{0..3, 42\}} evaluates to \texttt{TRUE}.

  \subsubsection{Code}
  \VerbatimInput[firstline=\EPatternBegin, lastline=\EPatternEnd]{../Interpreter.ml}

  \subsection{Formally: sequential case}
  \begin{align}
  \end{align} 

  \subsection{Formally: concurrent case}
  \begin{align}
  \end{align} 

  \subsection{Comments}

\section{SemanticsRule.CTC \label{sec:SemanticsRule.CTC}}

  \subsection{Prose}
  Evaluation of the expression \texttt{e} under environment \texttt{env} is
  \texttt{(res,new\_env)} and all of the following apply:
  \begin{itemize}
  \item \texttt{(e,\_t)} denotes a Checked-Typed-Conversion;
  \item 
  \end{itemize} 

  \subsection{Example: SemanticsRule.CTCValue.asl}
  \VerbatimInput{../tests/ASLSemanticsReference.t/SemanticsRule.CTCValue.asl}

  \subsection{Example: SemanticsRule.CTCError.asl}
  \VerbatimInput{../tests/ASLSemanticsReference.t/SemanticsRule.CTCError.asl}

  \subsubsection{Code}
  \VerbatimInput[firstline=\ESideEffectFreeExprBegin, lastline=\ESideEffectFreeExprEnd]
                {../Interpreter.ml}

  \subsection{Code}
  \VerbatimInput[firstline=332, lastline=337]{../Interpreter.ml}

  \subsection{Formally: sequential case}
  
  \subsection{Formally: concurrent case}

  \subsection{Comments}


\chapter{Evaluation of Left-Hand-Side Expressions \label{chap:eval_lexpr}}

Evaluation of the left-hand-side expression \texttt{le} associated with a
value \texttt{v} under an environment \texttt{env} is either \texttt{new\_env}
or an error is raised and one of the following applies:
\begin{itemize}
\item SemanticsRule.LEIgnore (see Section~\ref{sec:SemanticsRule.LEIgnore});
\item SemanticsRule.LELocalVar (see Section~\ref{sec:SemanticsRule.LELocalVar});
\item SemanticsRule.LEGlobalVar (see Section~\ref{sec:SemanticsRule.LEGlobalVar});
\item SemanticsRule.LEUndefIdentV0 (see Section~\ref{sec:SemanticsRule.LEUndefIdentV0});
\item SemanticsRule.LEUndefIdentV1 (see Section~\ref{sec:SemanticsRule.LEUndefIdentV1});
\item SemanticsRule.LESlice (see Section~\ref{sec:SemanticsRule.LESlice});
\item SemanticsRule.LESetArray (see Section~\ref{sec:SemanticsRule.LESetArray});
\item SemanticsRule.LESetField (see Section~\ref{sec:SemanticsRule.LESetField});
\item SemanticsRule.LETuple (see Section~\ref{sec:SemanticsRule.LETuple}).
\end{itemize}

\section{SemanticsRule.LEIgnore \label{sec:SemanticsRule.LEIgnore}}

    \subsection{Prose}
    Evaluation of the left-hand-side expression \texttt{le} associated with a
value \texttt{v} under an environment \texttt{env} is \texttt{new\_env} and all
of the following apply:
    \begin{itemize}
    \item \texttt{le} can be ignored;
    \item \texttt{new\_env} is \texttt{env}.
    \end{itemize}

  \subsection{Example}
  The left-hand-side expression [-] within [- = 42;] evaluates to the value [42].

  \subsubsection{Code}
  \VerbatimInput[firstline=\LEIgnoreBegin, lastline=\LEIgnoreEnd]{../Interpreter.ml}

  \subsection{Formally: sequential case}
  \begin{align}
  \end{align} 

  \subsection{Formally: concurrent case}
  \begin{align}
  \end{align} 

  \subsection{Comments}

\section{SemanticsRule.LELocalVar \label{sec:SemanticsRule.LELocalVar}}

    \subsection{Prose}

    Evaluation of the left-hand-side expression \texttt{le} associated with a
value \texttt{v} under an environment \texttt{env} is \texttt{new\_env} and all
of the following apply:
    \begin{itemize}
    \item \texttt{le} denotes a variable \texttt{x};
    \item \texttt{x} is locally bound in \texttt{env};
    \item \texttt{new\_env} is \texttt{env} where \texttt{x} is bound to \texttt{v}.
    \end{itemize}

    \subsection{Example: SemanticsRule.LELocalVar.asl}
    In the program:
    \VerbatimInput{../tests/ASLSemanticsReference.t/SemanticsRule.LELocalVar.asl}
    the evaluation of the left-hand-side expression \texttt{x} within \texttt{x = 42;} uses SemanticsRule.LELocalVar.

  \subsubsection{Code}
  \VerbatimInput[firstline=\LELocalVarBegin, lastline=\LELocalVarEnd]{../Interpreter.ml}

  \subsection{Formally: sequential case}
  \begin{align}
  \end{align} 

  \subsection{Formally: concurrent case}
  \begin{align}
  \end{align} 

  \subsection{Comments}

\section{SemanticsRule.LEGlobalVar \label{sec:SemanticsRule.LEGlobalVar}}

    \subsection{Prose}
    Evaluation of the left-hand-side expression \texttt{le} associated with a
value \texttt{v} under an environment \texttt{env} is \texttt{new\_env} and all
of the following apply:
    \begin{itemize}
    \item \texttt{le} denotes a variable \texttt{x};
    \item \texttt{x} is globally bound in \texttt{env};
    \item \texttt{new\_env} is \texttt{env} where \texttt{x} is bound to \texttt{v}.
    \end{itemize}

    \subsection{Example: SemanticsRule.LEGlobalVar.asl}
    In the program:
    \VerbatimInput{../tests/ASLSemanticsReference.t/SemanticsRule.LEGlobalVar.asl}
    the evaluation of the left-hand-side expression \texttt{x} within \texttt{x = 42;} uses SemanticsRule.LEGlobalVar.

  \subsubsection{Code}
  \VerbatimInput[firstline=\LEGlobalVarBegin, lastline=\LEGlobalVarEnd]{../Interpreter.ml}

  \subsection{Formally: sequential case}
  \begin{align}
  \end{align} 

  \subsection{Formally: concurrent case}
  \begin{align}
  \end{align} 

    \subsection{Comments}

\section{SemanticsRule.LEUndefIdentV0 \label{sec:SemanticsRule.LEUndefIdentV0}}

    \subsection{Prose}
    Evaluation of the left-hand-side expression \texttt{le} associated with a
value \texttt{v} under an environment \texttt{env} is \texttt{new\_env} and all
of the following apply:
    \begin{itemize}
    \item \texttt{le} denotes a variable \texttt{x} which is not bound in \texttt{env};
    \item the ASL language version is \texttt{V0};
    \item \texttt{new\_env} is \texttt{env} where \texttt{x} has been declared
a local variable bound to the value \texttt{v}.
    \end{itemize}

  \subsection{Example: SemanticsRule.LEUndefIdentV0.asl}
  \VerbatimInput{../tests/ASLSemanticsReference.t/SemanticsRule.LEUndefIdentV0.asl}

  \subsubsection{Code}
  \VerbatimInput[firstline=\LEUndefIdentZeroBegin, lastline=\LEUndefIdentZeroEnd]{../Interpreter.ml}

  \subsection{Formally: sequential case}
  \begin{align}
  \end{align} 

  \subsection{Formally: concurrent case}
  \begin{align}
  \end{align} 

  \subsection{Comments}

\section{SemanticsRule.LEUndefIdentV1 \label{sec:SemanticsRule.LEUndefIdentV1}}

    \subsection{Prose}
    All of the following apply:
    \begin{itemize}
    \item \texttt{le} denotes a variable \texttt{x} which is not bound in \texttt{env};
    \item the ASL language version is \texttt{V1};
    \item an \texttt{UndefinedIdentifier} error is raised.
    \end{itemize}

    \subsection{Example: SemanticsRule.LEUndefIdentV1.asl}
    In the program:
    \VerbatimInput{../tests/ASLSemanticsReference.t/SemanticsRule.LEUndefIdentV1.asl}
    the evaluation of the left-hand-side expression \texttt{y} within \texttt{y
= 3;} raises an ``Undefined identifier'' error in the environment where
\texttt{x} is bound to \texttt{42}.

  \subsubsection{Code}
  \VerbatimInput[firstline=\LEUndefIdentOneBegin, lastline=\LEUndefIdentOneEnd]{../Interpreter.ml}

  \subsection{Formally: sequential case}
  \begin{align}
  \end{align} 

  \subsection{Formally: concurrent case}
  \begin{align}
  \end{align} 

    \subsection{Comments}

\section{SemanticsRule.LESlice \label{sec:SemanticsRule.LESlice}}

    \subsection{Prose}
    Evaluation of the left-hand-side expression \texttt{le} associated with a
value \texttt{v} under an environment \texttt{env} is \texttt{new\_env} and all
of the following apply:
    \begin{itemize}
    \item \texttt{le} denotes a left-hand-side expression sliced as per \texttt{slices};
    \item The right-hand-side expression corresponding to \texttt{le} is \texttt{re\_bv};
    \item The evaluation of \texttt{re\_bv} under \texttt{env} is \texttt{rv\_bv};
    \item The evaluation of \texttt{slices} under \texttt{env} is \texttt{positions};
    \item \texttt{new\_m\_bv} is \texttt{rv\_bv} where the positions \texttt{positions} have been updated to \texttt{v};
    \item \texttt{new\_env} is \texttt{env} where \texttt{le} is bound to \texttt{new\_m\_bv}.
    \end{itemize}

   \subsection{Example: SemanticsRule.LESlice.asl}
   In the program:
    \VerbatimInput{../tests/ASLSemanticsReference.t/SemanticsRule.LESlice.asl}
   \texttt{x[3:0] = '0000'} binds \texttt{x} to \texttt{'11110000'} in the environment where \texttt{x} is bound to \texttt{'11111111'}.

  \subsection{Code}
  \VerbatimInput[firstline=\LESliceBegin, lastline=\LESliceEnd]{../Interpreter.ml}

  \subsection{Formally: sequential case}
  \begin{align}
  \end{align} 

  \subsection{Formally: concurrent case}
  \begin{align}
  \end{align} 

   \subsection{Comments}

\section{SemanticsRule.LESetArray \label{sec:SemanticsRule.LESetArray}}

    \subsection{Prose}
    Evaluation of the left-hand-side expression \texttt{le} associated with a
value \texttt{v} under an environment \texttt{env} is \texttt{new\_env} and all
of the following apply:
    \begin{itemize}
    \item \texttt{le} denotes an array \texttt{le\_array} and an index \texttt{e\_index};
    \item The right-hand-side expression corresponding to \texttt{le\_array} is \texttt{re\_array};
    \item The evaluation of \texttt{re\_array} under \texttt{env} is \texttt{rv\_array};
    \item The evaluation of \texttt{e\_index} under \texttt{env} is \texttt{v\_index};
    \item \texttt{new\_v\_array} is \texttt{rv\_array} where the value at index \texttt{v\_index} has been
      updated to \texttt{v};
    \item \texttt{new\_env} is \texttt{env} where \texttt{le\_array} is bound to
      \texttt{new\_v\_array}.
    \end{itemize}

    \subsection{Example: SemanticsRule.LESetArray.asl}
    The program:
    \VerbatimInput{../tests/ASLSemanticsReference.t/SemanticsRule.LESetArray.asl}
    binds the third element of \texttt{my\_array} to the value \texttt{53}.

  \subsection{Code}
  \VerbatimInput[firstline=\LESetArrayBegin, lastline=\LESetArrayEnd]{../Interpreter.ml}

  \subsection{Formally: sequential case}
  \begin{align}
  \end{align} 

  \subsection{Formally: concurrent case}
  \begin{align}
  \end{align} 

    \subsection{Comments}

\section{SemanticsRule.LESetField \label{sec:SemanticsRule.LESetField}}

    \subsection{Prose}
    Evaluation of the left-hand-side expression \texttt{le} associated with a
value \texttt{v} under an environment \texttt{env} is \texttt{new\_env} and all
of the following apply:
    \begin{itemize}
    \item \texttt{le} denotes a field name \texttt{field\_name} in a record \texttt{le\_record};
    \item The right-hand-side expression corresponding to \texttt{le\_record} is
      \texttt{re\_record};
    \item The evaluation of \texttt{re\_record} under \texttt{env} is \texttt{rv\_record};
    \item \texttt{new\_v\_record} is \texttt{rv\_record} where the field [field\_name] has been
      updated to \texttt{v};
    \item \texttt{new\_env} is \texttt{env} where \texttt{le\_record} is bound to
      \texttt{new\_v\_record}.
    \end{itemize}

    \subsection{Example: SemanticsRule.LESetField.asl}
    In the program:
    \VerbatimInput{../tests/ASLSemanticsReference.t/SemanticsRule.LESetField.asl}
    \texttt{my\_record.a = 42;} binds \texttt{my\_record} to \texttt{\{a: 42, b: 42\}} in the environment where \texttt{my\_record} is bound to \texttt{\{a: 3, b: 42\}}.

  \subsection{Code}
  \VerbatimInput[firstline=\LESetFieldBegin, lastline=\LESetFieldEnd]{../Interpreter.ml}

  \subsection{Comments}

\section{SemanticsRule.LEDestructuring \label{sec:SemanticsRule.LEDestructuring}}

    \subsection{Prose}
    Evaluation of the left-hand-side expression \texttt{le} associated with a
value \texttt{v} under an environment \texttt{env} is \texttt{new\_env} and all
of the following apply:
    \begin{itemize}
    \item [le] denotes a list of left-hand-side expressions [le\_list];
    \item [new\_env] is [env] where each left-hand-side expression in [le\_list] has
      been assigned the value at the corresponding index in [v].
    \end{itemize}

    \subsection{Example: SemanticsRule.LEDestructuring.asl}
    In the program:
    \VerbatimInput{../tests/ASLSemanticsReference.t/SemanticsRule.LEDestructuring.asl}
    \texttt{(x, y) = (3, 42)} binds \texttt{x} to \texttt{3} and \texttt{y} to \texttt{42} in the environment where \texttt{x} is bound to \texttt{42} and \texttt{y} is bound to \texttt{3}.

\chapter{Evaluation of Slices \label{chap:eval_slices}}
\texttt{eval\_slices env slices} is the list of pair \texttt{(i\_n, l\_n)} that
corresponds to the start (included) and the length of each slice in
\texttt{slices}.


  \subsection{Example: SemanticsRule.Slice0.asl}
  In the program:
  \VerbatimInput{../tests/ASLSemanticsReference.t/SemanticsRule.Slice0.asl}
  \texttt{x[3:2]} evaluates to \texttt{'10'}.

  \subsection{Example: SemanticsRule.Slice1.asl}
  In the program:
  \VerbatimInput{../tests/ASLSemanticsReference.t/SemanticsRule.Slice1.asl}
  \texttt{x[3:2, 42+:5]} evaluates to \texttt{'1010101'}.

\chapter{Evaluation of Patterns \label{chap:eval_pattern}}
[eval\_pattern env pos v p] determines if [v] matches the pattern [p].

Evaluation of the pattern \texttt{p} under environment \texttt{env} is
\texttt{(res,new\_env)}, or an error, and one of the following applies:
\begin{itemize}
\item SemanticsRule.PAll (see Section~\ref{sec:SemanticsRule.PAll})
\item SemanticsRule.PAny (see Section~\ref{sec:SemanticsRule.PAny})
\item SemanticsRule.PGeq (see Section~\ref{sec:SemanticsRule.PGeq})
\item SemanticsRule.PLeq (see Section~\ref{sec:SemanticsRule.PLeq})
\item SemanticsRule.PNot (see Section~\ref{sec:SemanticsRule.PNot})
\item SemanticsRule.PRange (see Section~\ref{sec:SemanticsRule.PRange})
\item SemanticsRule.PSingle (see Section~\ref{sec:SemanticsRule.PSingle})
\item SemanticsRule.PMask (see Section~\ref{sec:SemanticsRule.PMask})
\item SemanticsRule.PTuple (see Section~\ref{sec:SemanticsRule.PTuple})
\end{itemize}

\section{SemanticsRule.PAll \label{sec:SemanticsRule.PAll}}

  \subsection{Prose}
  Evaluation of the pattern \texttt{p} under environment \texttt{env} is
  \texttt{(res,new\_env)} and all of the following apply:
  \begin{itemize}  
  \item \texttt{p} is the pattern which matches everything;
  \item \texttt{res} is a value \texttt{v};
  \item \texttt{v} is \texttt{true};
  \item \texttt{new\_env} is \texttt{env}.
  \end{itemize}

  \subsection{Example: SemanticsRule.PAll.asl}
  \VerbatimInput{../tests/ASLSemanticsReference.t/SemanticsRule.PAll.asl}

  \subsection{Code}
  \VerbatimInput[firstline=, lastline=]{../Interpreter.ml}

  \subsection{Formally: sequential case}
  \begin{align}
  \end{align} 

  \subsection{Formally: concurrent case}
  \begin{align}
  \end{align} 

  \subsection{Comments}

\section{SemanticsRule.PAny \label{sec:SemanticsRule.PAny}}

    \subsection{Prose}
  Evaluation of the pattern \texttt{p} under environment \texttt{env} is
  \texttt{(res,new\_env)} and all of the following apply:
    \begin{itemize}
    \item \texttt{p} is a list of patterns \texttt{li\_patterns};
    \item \texttt{res} is a value \texttt{v};
    \item \texttt{v} is a record of values \texttt{li\_values};
    \item at least one value in \texttt{li\_values} matches the corresponding
pattern in \texttt{li\_patterns} in \texttt{env};
    \item \texttt{new\_env} is \texttt{env}.
    \end{itemize}

    \subsection{Example: SemanticsRule.PAnyTRUE.asl}
    \VerbatimInput{../tests/ASLSemanticsReference.t/SemanticsRule.PAnyTRUE.asl}

    \subsection{Example: SemanticsRule.PAnyFALSE.asl}
    \VerbatimInput{../tests/ASLSemanticsReference.t/SemanticsRule.PAnyFALSE.asl}

    \subsection{Code}
    \VerbatimInput[firstline=, lastline=]{../Interpreter.ml}

    \subsection{Formally: sequential case}
    \begin{align}
    \end{align} 

    \subsection{Formally: concurrent case}
    \begin{align}
    \end{align} 

    \subsection{Comments}

\section{SemanticsRule.PGeq \label{sec:SemanticsRule.PGeq}}

    \subsection{Prose}
   Evaluation of the pattern \texttt{p} under environment \texttt{env} is
  \texttt{(res,new\_env)} and all of the following apply:
    \begin{itemize}
    \item [p] is a side-effect-free expression [e];
    \item \texttt{res} is a value \texttt{v};
    \item [v] is greater than or equal to the evaluation of [e] in [env];
    \item \texttt{new\_env} is \texttt{env}.
    \end{itemize}

    \subsection{Example: SemanticsRule.PGeqTRUE.asl}
    \VerbatimInput{../tests/ASLSemanticsReference.t/SemanticsRule.PGeqTRUE.asl}

    \subsection{Example: SemanticsRule.PGeqFALSE.asl}
    \VerbatimInput{../tests/ASLSemanticsReference.t/SemanticsRule.PGeqFALSE.asl}

  \subsection{Code}
  \VerbatimInput[firstline=, lastline=]{../Interpreter.ml}

  \subsection{Formally: sequential case}
  \begin{align}
  \end{align} 

  \subsection{Formally: concurrent case}
  \begin{align}
  \end{align} 

  \subsection{Comments}

\section{SemanticsRule.PLeq \label{sec:SemanticsRule.PLeq}}

    \subsection{Prose}

   Evaluation of the pattern \texttt{p} under environment \texttt{env} is
  \texttt{(res,new\_env)} and all of the following apply:
    \begin{itemize}
    \item \texttt{p} is a side-effect-free expression \texttt{e};
    \item \texttt{res} is a value \texttt{v};
    \item \texttt{v} is less than or equal to the evaluation of \texttt{e} in \texttt{env};
    \item \texttt{new\_env} is \texttt{env}.
    \end{itemize}

    \subsection{Example: SemanticsRule.PLeqTRUE.asl}
    \VerbatimInput{../tests/ASLSemanticsReference.t/SemanticsRule.PLeqTRUE.asl}

    \subsection{Example: SemanticsRule.PLeqFALSE.asl}
    \VerbatimInput{../tests/ASLSemanticsReference.t/SemanticsRule.PLeqFALSE.asl}

  \subsection{Code}
  \VerbatimInput[firstline=, lastline=]{../Interpreter.ml}

  \subsection{Formally: sequential case}
  \begin{align}
  \end{align} 

  \subsection{Formally: concurrent case}
  \begin{align}
  \end{align} 

    \subsection{Comments}

\section{SemanticsRule.PNot \label{sec:SemanticsRule.PNot}}

    \subsection{Prose}
   Evaluation of the pattern \texttt{p} under environment \texttt{env} is
  \texttt{(res,new\_env)} and all of the following apply:
    \begin{itemize}
    \item \texttt{p} is a list of patterns \texttt{li\_patterns};
    \item \texttt{res} is a value \texttt{v};
    \item \texttt{v} is a record of values \texttt{li\_values};
    \item each value in \texttt{li\_values} does not match the corresponding pattern in \texttt{li\_patterns} in \texttt{env};
    \item \texttt{new\_env} is \texttt{env}.
    \end{itemize}
 
    \subsection{Example: SemanticsRule.PNotTRUE.asl}
    \VerbatimInput{../tests/ASLSemanticsReference.t/SemanticsRule.PNotTRUE.asl}

    \subsection{Example: SemanticsRule.PNotFALSE.asl}
    \VerbatimInput{../tests/ASLSemanticsReference.t/SemanticsRule.PNotFALSE.asl}

  \subsection{Code}
  \VerbatimInput[firstline=, lastline=]{../Interpreter.ml}

  \subsection{Formally: sequential case}
  \begin{align}
  \end{align} 

  \subsection{Formally: concurrent case}
  \begin{align}
  \end{align} 

    \subsection{Comments}

\section{SemanticsRule.PRange \label{sec:SemanticsRule.PRange}}

    \subsection{Prose}
   Evaluation of the pattern \texttt{p} under environment \texttt{env} is
  \texttt{(res,new\_env)} and all of the following apply:
    \begin{itemize}
    \item \texttt{p} is two side-effect-free expressions \texttt{e1} and \texttt{e2};
    \item \texttt{res} is a value \texttt{v};
    \item \texttt{v} is greater than or equal to \texttt{e1} in \texttt{env};
    \item \texttt{v} is lesser than or equal to \texttt{e2} in \texttt{env};
    \item \texttt{new\_env} is \texttt{env}.
    \end{itemize}

    \subsection{Example: SemanticsRule.PRangeTRUE.asl}
    \VerbatimInput{../tests/ASLSemanticsReference.t/SemanticsRule.PRangeTRUE.asl}

    \subsection{Example: SemanticsRule.PRangeFALSE.asl}
    \VerbatimInput{../tests/ASLSemanticsReference.t/SemanticsRule.PRangeFALSE.asl}

  \subsection{Code}
  \VerbatimInput[firstline=, lastline=]{../Interpreter.ml}

  \subsection{Formally: sequential case}
  \begin{align}
  \end{align} 

  \subsection{Formally: concurrent case}
  \begin{align}
  \end{align} 

    \subsection{Comments}

\section{SemanticsRule.PSingle \label{sec:SemanticsRule.PSingle}}

    \subsection{Prose}
   Evaluation of the pattern \texttt{p} under environment \texttt{env} is
  \texttt{(res,new\_env)} and all of the following apply:
    \begin{itemize}
    \item \texttt{p} is a side-effect-free expression \texttt{e};
    \item \texttt{res} is a value \texttt{v};
    \item \texttt{v} is the result of the evaluation of \texttt{e} in \texttt{env};
    \item \texttt{new\_env} is \texttt{env}.
    \end{itemize}

    \subsection{Example: SemanticsRule.PSingleTRUE.asl}
    \VerbatimInput{../tests/ASLSemanticsReference.t/SemanticsRule.PSingleTRUE.asl}

    \subsection{Example: SemanticsRule.PSingleFALSE.asl}
    \VerbatimInput{../tests/ASLSemanticsReference.t/SemanticsRule.PSingleFALSE.asl}

  \subsection{Code}
  \VerbatimInput[firstline=, lastline=]{../Interpreter.ml}

  \subsection{Formally: sequential case}
  \begin{align}
  \end{align} 

  \subsection{Formally: concurrent case}
  \begin{align}
  \end{align} 

    \subsection{Comments}

\section{SemanticsRule.PMask \label{sec:SemanticsRule.PMask}}

    \subsection{Prose}
   Evaluation of the pattern \texttt{p} under environment \texttt{env} is
  \texttt{(res,new\_env)} and all of the following apply:
    \begin{itemize}
    \item \texttt{p} is a mask \texttt{m};
    \item \texttt{res} is a value \texttt{v};
    \item \texttt{v} is a bitvector that matches the mask \texttt{m};
    \item \texttt{new\_env} is \texttt{env}.
    \end{itemize}

    \subsection{Example: SemanticsRule.PMaskTRUE.asl}
    \VerbatimInput{../tests/ASLSemanticsReference.t/SemanticsRule.PMaskTRUE.asl}

    \subsection{Example: SemanticsRule.PMaskFALSE.asl}
    \VerbatimInput{../tests/ASLSemanticsReference.t/SemanticsRule.PMaskFALSE.asl}

  \subsection{Code}
  \VerbatimInput[firstline=, lastline=]{../Interpreter.ml}

  \subsection{Formally: sequential case}
  \begin{align}
  \end{align} 

  \subsection{Formally: concurrent case}
  \begin{align}
  \end{align} 

    \subsection{Comments}

\section{SemanticsRule.PTuple \label{sec:SemanticsRule.PTuple}}

    \subsection{Prose}
   Evaluation of the pattern \texttt{p} under environment \texttt{env} is
  \texttt{(res,new\_env)} and all of the following apply: 
    \begin{itemize}
    \item \texttt{p} gives a list of patterns \texttt{li};
    \item \texttt{res} is a value \texttt{v};
    \item each element in \texttt{v} matches the corresponding pattern in the list \texttt{li} in \texttt{env};
    \item \texttt{new\_env} is \texttt{env}.
    \end{itemize}

    \subsection{Example: SemanticsRule.PTupleTRUE.asl}
    \VerbatimInput{../tests/ASLSemanticsReference.t/SemanticsRule.PTupleTRUE.asl}

    \subsection{Example: SemanticsRule.PTupleFALSE.asl}
    \VerbatimInput{../tests/ASLSemanticsReference.t/SemanticsRule.PTupleFALSE.asl}

    \subsection{Comments}

\chapter{Evaluation of Local Declarations \label{chap:eval_local_decl}}

\texttt{eval\_local\_decl ldi env m\_init\_opt} declares local variables
\texttt{ldi} in \texttt{env} with an optional initialisation value
\texttt{m\_init\_opt}.  Evaluation of the local variables \texttt{ldi}
under an environment \texttt{env} is either \texttt{new\_env} or raises an
error and one of the following applies:
\begin{itemize}
\item SemanticsRule.LDIgnore (see Section~\ref{sec:SemanticsRule.LDIgnore},
\item SemanticsRule.LDVar (see Section~\ref{sec:SemanticsRule.LDVar},
\item SemanticsRule.LDTypedVar (see Section~\ref{sec:SemanticsRule.LDTypedVar},
\item SemanticsRule.LDTuple (see Section~\ref{sec:SemanticsRule.LDTuple},
\item SemanticsRule.LDTypedTuple (see Section~\ref{sec:SemanticsRule.LDTypedTuple},
\item SemanticsRule.LDUninitialisedTuple (see Section~\ref{sec:SemanticsRule.LDUninitialisedTuple}.
\end{itemize}

\section{SemanticsRule.LDIgnore \label{sec:SemanticsRule.LDIgnore}}

    \subsection{Prose}
Evaluation of the local variables \texttt{ldi} under the environment
\texttt{env} is \texttt{new\_env} and all of the following apply:
    \begin{itemize}
    \item \texttt{ldi} indicates that the initialisation value will be discarded;
    \item \texttt{new\_env} is \texttt{env}.
    \end{itemize}

    \subsection{Examples}
    [var - = 3;] evaluates [3].

    [var - : integer;] does not modify the environment.

  \subsection{Code}
  \VerbatimInput[firstline=, lastline=]{../Interpreter.ml}

  \subsection{Formally: sequential case}
  \begin{align}
  \end{align} 

  \subsection{Formally: concurrent case}
  \begin{align}
  \end{align} 

    \subsection{Comments}

\section{SemanticsRule.LDVar \label{sec:SemanticsRule.LDVar}}

    \subsection{Prose}
Evaluation of the local variables \texttt{ldi} under the environment
\texttt{env} is \texttt{new\_env} and all of the following apply:
    \begin{itemize}
    \item \texttt{ldi} is a variable \texttt{x};
    \item \texttt{m\_init\_opt} is a value \texttt{m};
    \item \texttt{new\_env} is \texttt{env} modified to declare \texttt{x} as a
local variable bound to value \texttt{m}.
    \end{itemize}

    \subsection{Example: SemanticsRule.LDVar0.asl}
    In the program:
    \VerbatimInput{../tests/ASLSemanticsReference.t/SemanticsRule.LDVar0.asl}
    \texttt{var x = 3;} binds \texttt{x} to the evaluation of \texttt{3} in \texttt{env}.

    \subsection{Example: SemanticsRule.LDVar1.asl}
    In the program:
    \VerbatimInput{../tests/ASLSemanticsReference.t/SemanticsRule.LDVar1.asl}
    \texttt{var x : integer = 3;} binds \texttt{x} to the evaluation of
\texttt{3} in \texttt{env}, without type consideration at runtime.

  \subsection{Code}
  \VerbatimInput[firstline=, lastline=]{../Interpreter.ml}

  \subsection{Formally: sequential case}
  \begin{align}
  \end{align} 

  \subsection{Formally: concurrent case}
  \begin{align}
  \end{align} 

    \subsection{Comments}

\section{SemanticsRule.LDTypedVar \label{sec:SemanticsRule.LDTypedVar}}

    \subsection{Prose}
Evaluation of the local variables \texttt{ldi} under the environment
\texttt{env} is \texttt{new\_env} and all of the following apply:
    \begin{itemize}
    \item \texttt{ldi} is a variable \texttt{x} of type \texttt{ty};
    \item \texttt{m\_init\_opt} is \texttt{None};
    \item \texttt{new\_env} is \texttt{env} modified to declare \texttt{x} as a local variable bound to
      the base value of \texttt{ty}.
    \end{itemize}

    \subsection{Examples}
    [var x : integer;] binds [x] in [env] to the base value of [integer].

  \subsection{Code}
  \VerbatimInput[firstline=, lastline=]{../Interpreter.ml}

  \subsection{Formally: sequential case}
  \begin{align}
  \end{align} 

  \subsection{Formally: concurrent case}
  \begin{align}
  \end{align} 

    \subsection{Comments}

\section{SemanticsRule.LDTuple \label{sec:SemanticsRule.LDTuple}}

    \subsection{Prose}
Evaluation of the local variables \texttt{ldi} under the environment
\texttt{env} is \texttt{new\_env} and all of the following apply:
    \begin{itemize}
    \item \texttt{ldi} gives a list of local variables \texttt{ldis};
    \item \texttt{m\_init\_opt} is a list of values \texttt{liv};
    \item \texttt{new\_env} is \texttt{env} modified to declare each element of \texttt{ldis} to be bound
      to the corresponding value in \texttt{liv}.
    \end{itemize}

    \subsection{Example: SemanticsRule.LDTuple.asl}
    [var (x,y,z) = (1,2,3);] binds [x] (resp. [y], [z]) to the evaluation of
    [1] (resp. [2], [3]) in [env].

  \subsection{Code}
  \VerbatimInput[firstline=, lastline=]{../Interpreter.ml}

  \subsection{Formally: sequential case}
  \begin{align}
  \end{align} 

  \subsection{Formally: concurrent case}
  \begin{align}
  \end{align} 

    \subsection{Comments}

\section{SemanticsRule.LDTypedTuple \label{sec:SemanticsRule.LDTypedTuple}}

    \subsection{Prose}
Evaluation of the local variables \texttt{ldi} under the environment
\texttt{env} is \texttt{new\_env} and all of the following apply:
    \begin{itemize}
    \item \texttt{ldi} gives a list of local variables \texttt{ldis} and a type \texttt{ty};
    \item \texttt{m\_init\_opt} is \texttt{None};
    \item \texttt{new\_env} is \texttt{env} modified to declare each element of \texttt{ldis} with type \texttt{ty}.
    \end{itemize}

    \subsection{Examples}
    [var (x,y,z) : integer;] binds [x], [y] and [z] in [env] to the base value
    of [integer].

  \subsection{Code}
  \VerbatimInput[firstline=, lastline=]{../Interpreter.ml}

  \subsection{Formally: sequential case}
  \begin{align}
  \end{align} 

  \subsection{Formally: concurrent case}
  \begin{align}
  \end{align} 

    \subsection{Comments}

\section{SemanticsRule.LDUninitialisedTuple \label{sec:SemanticsRule.LDUninitialisedTuple}}

    \subsection{Prose}
Evaluation of the local variables \texttt{ldi} under the environment
\texttt{env} is \texttt{new\_env} and all of the following apply:
    \begin{itemize}
    \item \texttt{ldi} gives a list of local variables \texttt{ldis};
    \item \texttt{new\_env} is \texttt{env} modified to declare each element of \texttt{ldis}.
    \end{itemize}

    \subsection{Examples}
    [var (x : integer, y : boolean);] binds [x] to the base value of [integer]
    and [y] to the base value of [boolean].

    \subsection{Comments}

\chapter{Evaluation of Statements \label{chap:eval_stmt}}
Evaluation \texttt{eval\_stmt env s} of a statement \texttt{s} under
environment \texttt{env} is either a \texttt{Throwing}, an interruption
\texttt{Returning vs} or a new environment \texttt{new\_env}. Formally, one of
the following applies:
\begin{itemize}
\item SemanticsRule.SPass (see Section~\ref{sec:SemanticsRule.SPass}), 
\item SemanticsRule.SAssign (see Section~\ref{sec:SemanticsRule.SAssign}), 
\item SemanticsRule.SAssignCall (see Section~\ref{sec:SemanticsRule.SAssignCall}), 
\item SemanticsRule.SAssignTuple (see Section~\ref{sec:SemanticsRule.SAssignTuple}), 
\item SemanticsRule.SReturnNone (see Section~\ref{sec:SemanticsRule.SReturnNone}), 
\item SemanticsRule.SReturnOne (see Section~\ref{sec:SemanticsRule.SReturnOne}), 
\item SemanticsRule.SReturnSome (see Section~\ref{sec:SemanticsRule.SReturnSome}), 
\item SemanticsRule.SSeq (see Section~\ref{sec:SemanticsRule.SSeq}), 
\item SemanticsRule.SCall (see Section~\ref{sec:SemanticsRule.SCall}), 
\item SemanticsRule.SCond (see Section~\ref{sec:SemanticsRule.SCond}), 
\item SemanticsRule.SCase (see Section~\ref{sec:SemanticsRule.SCase}), 
\item SemanticsRule.SAssert (see Section~\ref{sec:SemanticsRule.SAssert}), 
\item SemanticsRule.SWhile (see Section~\ref{sec:SemanticsRule.SWhile}), 
\item SemanticsRule.SRepeat (see Section~\ref{sec:SemanticsRule.SRepeat}), 
\item SemanticsRule.SFor (see Section~\ref{sec:SemanticsRule.SFor}), 
\item SemanticsRule.SThrowNone (see Section~\ref{sec:SemanticsRule.SThrowNone}), 
\item SemanticsRule.SThrowSomeTyped (see Section~\ref{sec:SemanticsRule.SThrowSomeTyped}), 
\item SemanticsRule.SThrowSome (see Section~\ref{sec:SemanticsRule.SThrowSome}), 
\item SemanticsRule.STry (see Section~\ref{sec:SemanticsRule.STry}), 
\item SemanticsRule.SDeclSome (see Section~\ref{sec:SemanticsRule.SDeclSome}), 
\item SemanticsRule.SDeclNone (see Section~\ref{sec:SemanticsRule.SDeclNone}). 
\end{itemize}

\section{SemanticsRule.SPass \label{sec:SemanticsRule.SPass}}

  \subsection{Prose}
  Evaluation of the statement \texttt{s} under environment \texttt{env} is
\texttt{new\_env} and all of the following apply:
  \begin{itemize}
  \item \texttt{s} is a \texttt{pass} statement;
  \item \texttt{new\_env} is \texttt{env}.
  \end{itemize}

  \subsection{Examples}
  [pass;] does nothing.

  \subsection{Code}
  \VerbatimInput[firstline=, lastline=]{../Interpreter.ml}

  \subsection{Formally: sequential case}
  \begin{align}
  \interp{\syntt{pass}} (\texttt{env}) & \triangleq \{ (\bot, \texttt{env}) \}
  \label{eq:sem-seq-spass}
  \end{align} 

  \subsection{Formally: concurrent case}
  \begin{align}
  \interp{\syntt{pass}} (\texttt{env}) & \triangleq \{ (\bot, \texttt{env}, \varnothing) \}
  \label{eq:sem-conc-spass}
  \end{align} 

  \subsection{Comments}

\section{SemanticsRule.SAssign \label{sec:SemanticsRule.SAssign}}

  \subsection{Prose}
  All of the following apply:
  \begin{itemize}
  \item [s] is an assignment [le = re];
  \item [v] is the evaluation of [re] under [env];
  \item [r\_env] is [env] modified after evaluation of [re] under [env];
  \item [new\_env] is [r\_env] modified after evaluation of [le] under [r\_env] with
    [v].
  \end{itemize}

  \subsection{Examples}
  [x = 3;] binds [x] to [3] in [x $mapsto$42] ([new\_env] is here [x $mapsto$3]).

  \subsection{Code}
  \VerbatimInput[firstline=, lastline=]{../Interpreter.ml}

  \subsection{Formally: sequential case}
  \begin{equation}
    \interp{x = e} (E) \triangleq
      \left\{ \left(\bot, E'[x \mapsto v ]\right) \st{} (v, E') \in \interp{e} (E) \right\}
    \label{eq:sem-seq-sassign}
  \end{equation}

  \subsection{Formally: concurrent case}
  \begin{equation}
    \interp{x = e} (E) \triangleq
      \left\{ \left(\bot, E'[x \mapsto v ], S \xrightarrow{\asldata} W(x)\right) \st{}
        (v, E', S) \in \interp{e} (E)
      \right\}
    \label{eq:sem-conc-sassign}
  \end{equation}

  \subsection{Comments}

\section{SemanticsRule.SAssignCall \label{sec:SemanticsRule.SAssignCall}}

    \subsection{Prose}
    <description>

    \subsection{Comments}

    \subsection{Examples}
    <example>

  \subsection{Code}
  \VerbatimInput[firstline=, lastline=]{../Interpreter.ml}

  \subsection{Formally: sequential case}
  \begin{align}
  \end{align} 

  \subsection{Formally: concurrent case}
  \begin{align}
  \end{align} 

\section{SemanticsRule.SAssignTuple \label{sec:SemanticsRule.SAssignTuple}}

    \subsection{Prose}
    <description>

    \subsection{Comments}

    \subsection{Examples}
    <example>

  \subsection{Code}
  \VerbatimInput[firstline=, lastline=]{../Interpreter.ml}

  \subsection{Formally: sequential case}
  \begin{align}
  \end{align} 

  \subsection{Formally: concurrent case}
  \begin{align}
  \end{align} 

\section{SemanticsRule.SReturnNone \label{sec:SemanticsRule.SReturnNone}}

    \subsection{Prose}
    All of the following apply:
    \begin{itemize}
    \item [s] is a return statement;
    \item [vs] is [[]];
    \item [new\_env] is [env].
    \end{itemize}

    \subsection{Comments}

    \subsection{Example: SReturnNoneReturn.asl}
    The program:
    \VerbatimInput{../tests/ASLSemanticsReference.t/SemanticsRule.SReturnNone.asl} 
    exits the current procedure.

  \subsection{Code}
  \VerbatimInput[firstline=, lastline=]{../Interpreter.ml}

  \subsection{Formally: sequential case}
  \begin{align}
  \end{align} 

  \subsection{Formally: concurrent case}
  \begin{align}
  \end{align} 

\section{SemanticsRule.SReturnOne \label{sec:SemanticsRule.SReturnOne}}

    \subsection{Prose}
    All of the following apply:
    \begin{itemize}
    \item [s] is a return statement;
    \item [s] gives an expression [e];
    \item [v] is the evaluation of [e] under [env];
    \item [vs] is [[v]];
    \item [new\_env] is [env] modified after evaluation of [e] under [env].
    \end{itemize}

    \subsection{Comments}

    \subsection{Examples}
    [return 3;] exits the current subprogram with value [3].

  \subsection{Code}
  \VerbatimInput[firstline=, lastline=]{../Interpreter.ml}

  \subsection{Formally: sequential case}
  \begin{align}
  \end{align} 

  \subsection{Formally: concurrent case}
  \begin{align}
  \end{align} 

\section{SemanticsRule.SReturnSome \label{sec:SemanticsRule.SReturnSome}}

    \subsection{Prose}
    All of the following apply:
    \begin{itemize}
    \item [s] is a return statement;
    \item [s] gives a list of expressions [es];
    \item [vs] is the piece-wise evaluation of the list [es] under [env];
    \item [new\_env] is [env] modified after evaluation of [e] under [env].
    \end{itemize}

    \subsection{Comments}

    \subsection{Examples}
    [return (3, 42);] exits the current subprogram with value [(3, 42)].

  \subsection{Code}
  \VerbatimInput[firstline=, lastline=]{../Interpreter.ml}

  \subsection{Formally: sequential case}
  \begin{align}
  \interp{\syntt{return}\ e} (E) & \triangleq \interp{e} (E)
  \label{eq:sem-seq-sreturn}
  \end{align} 

  \subsection{Formally: concurrent case}
  \begin{align}
  \interp{\syntt{return}\ e} (E) & \triangleq \interp{e} (E)
  \label{eq:sem-conc-sreturn}
  \end{align} 

\section{SemanticsRule.SSeq \label{sec:SemanticsRule.SSeq}}

  \subsection{Prose}
  All of the following apply:
  \begin{itemize}
  \item [s] is a then statement;
  \item [s] gives two statements [s1] and [s2];
  \item [env'] is [env] modified after evaluation of [s1];
  \item [new\_env] is [env'] modified after evaluation of [s2].
  \end{itemize}

  \subsection{Examples}
  [print(3); print(42);] prints ``342''.

  \subsection{Code}
  \VerbatimInput[firstline=, lastline=]{../Interpreter.ml}

  \subsection{Formally: sequential case}
  The semantics of~$s_1; s_2$ is the semantics of~$s_2$ applied to the results
  of the semantics of~$s_1$ if they do not perform an early return, in which
  case it is the semantics of~$s_1$.
  \begin{equation}
    \interp{s_1; s_2} (E) \triangleq
      \left\{ (v?, E'') \st{}
      \begin{aligned}
        & (v, E'') \in \interp{s_1} (E)
        \\ \text{or}\ &
        \left(
        \begin{aligned}
            & (\bot{}, E') \in \interp{s_1} (E)
            \\ \text{and}\ &
            (v?, E'') \in \interp{s_2} (E')
        \end{aligned}
        \right)
      \end{aligned}
      \right\}
    \label{eq:sem-seq-sseq}
  \end{equation}

  \subsection{Formally: concurrent case}
  The evaluation of two statements introduces an $\aslpo$ arrow between the two
graphs produced by their interpretations:
  \begin{equation}
    \interp{s_1; s_2} (E) \triangleq
      \left\{ (v, E'', S'') \st{}
      \begin{aligned}
        & (v, E'', S'') \in \interp{s_1} (E)
        \\ \text{or}\ &
        \left(
        \begin{aligned}
            & (\bot{}, E', S) \in \interp{s_1} (E)
            \\ \text{and}\ &
            (v, E'', S') \in \interp{s_2} (E')
            \\ \text{and}\ &
            S = S' \xrightarrow{\aslpo} S''
        \end{aligned}
        \right)
      \end{aligned}
      \right\}
    \label{eq:sem-conc-sseq}
  \end{equation}

  \subsection{Comments}

\section{SemanticsRule.SCall \label{sec:SemanticsRule.SCall}}

    \subsection{Prose}
    All of the following apply:
    \begin{itemize}
    \item [s] is a call statement;
    \item [s] gives a subprogram name [name], actual arguments [actual\_args] and
      parameters [params];
    \item [env'] is [env] modified after evaluation of the subprogram call;
    \item [new\_env] is [env'].
    \end{itemize}

    \subsection{Comments}

    \subsection{Examples}
    [Zeros(3)] evaluates to ['000'].

  \subsection{Code}
  \VerbatimInput[firstline=, lastline=]{../Interpreter.ml}

  \subsection{Formally: sequential case}
  \begin{align}
  \end{align} 

  \subsection{Formally: concurrent case}
  \begin{align}
  \end{align} 

\section{SemanticsRule.SCond \label{sec:SemanticsRule.SCond}}

    \subsection{Prose}
    All of the following apply:
    \begin{itemize}
    \item [s] is a conditional statement;
    \item [s] gives a condition [cond] and two statements [s1] and [s2];
    \item [v\_cond] is the evaluation of [cond];
    \item [new\_env] is [env] modified after evaluation of [s1] or [s2] depending on
      [v\_cond].
    \end{itemize}

    \subsection{Example: SemanticsRule.SCond.asl}
    The program:
    \VerbatimInput{../tests/ASLSemanticsReference.t/SemanticsRule.SCond.asl}
    prints "3".

  \subsection{Code}
  \VerbatimInput[firstline=, lastline=]{../Interpreter.ml}

  \subsection{Formally: sequential case}
  The semantics of a conditional statement $\ife{e}{s_1}{s_2}$ chooses between
  the semantics of~$s_1$ or $s_2$ depending on the evaluation of $e$:
  \begin{equation}
    \interp{\begin{array}{l}
      \syntt{if}\ e\\
      \syntt{then}\ s_1\\
      \syntt{else}\ s_2\ \syntt{end}
    \end{array}} (E) \triangleq
    % \interp{\ife{e}{s_1}{s_2}} (E) & \triangleq
      \left\{ (v?, E'') \st{} \ 
      \begin{aligned}
        & (b, E') \in \interp{e} (E)
        \\ \text{and}\ &
        \left( \begin{aligned}
          & \left(b = \syntt{true} \ \text{and} \ s' = s_1 \right)
          \\ \text{or}\ &
          \left( b = \syntt{false} \ \text{and}\ s' = s_2 \right)
        \end{aligned} \right)
        \\ \text{and}\ &
        (v?, E'') \in \interp{s'} (E')
      \end{aligned}
      \right\}
    \label{eq:sem-seq-scond}
  \end{equation}

  \subsection{Formally: concurrent case}
  A conditional statement introduces control dependencies $\aslctrl$ between its
  condition and its body:
  \begin{equation}
    \interp{\begin{array}{l}
      \syntt{if}\ e\\
      \syntt{then}\ s_1\\
      \syntt{else}\ s_2\ \syntt{end}
    \end{array}} (E) \triangleq
      \left\{ (v, E'', S'') \st{} \ 
      \begin{aligned}
        & (b, E', S) \in \interp{e} (E)
        \\ \text{and}\ &
        \left( \begin{aligned}
          & \left(b = \syntt{true}\ \text{and}\ s' = s_1 \right)
          \\ \text{or}\ &
          \left( b = \syntt{false}\ \text{and}\ s' = s_2 \right)
        \end{aligned} \right)
        \\ \text{and}\ &
        (v, E'', S') \in \interp{s'} (E')
        \\ \text{and}\ &
        S'' = S \xrightarrow{\aslctrl} S'
      \end{aligned}
      \right\}
    \label{eq:sem-conc-scond}
  \end{equation}

  \subsection{Comments}

\section{SemanticsRule.SCase \label{sec:SemanticsRule.SCase}}

    \subsection{Prose}
    <description>

    \subsection{Comments}

    \subsection{Example: SemanticsRule.SCase.asl}
    The program:
    \VerbatimInput{../tests/ASLSemanticsReference.t/SemanticsRule.SCase.asl} 
    prints ``less than 42.''

  \subsection{Code}
  \VerbatimInput[firstline=, lastline=]{../Interpreter.ml}

  \subsection{Formally: sequential case}
  \begin{align}
  \end{align} 

  \subsection{Formally: concurrent case}
  \begin{align}
  \end{align} 

\section{SemanticsRule.SAssert \label{sec:SemanticsRule.SAssert}}

    \subsection{Prose}
    All of the following apply:
    \begin{itemize}
    \item [s] is an assert statement;
    \item [s] gives an expression [e];
    \item [v] is the evaluation of [e];
    \item if [v] is true then [new\_env] is [env], otherwise an AssertionFailed
      error is raised.
    \end{itemize}

    \subsection{Comments}

    \subsection{Examples}
    [assert (42 != 3);] ensures that [3] is not equal to [42].
    [assert (42 == 3);] raises an [AssertionFailed] error.

  \subsection{Code}
  \VerbatimInput[firstline=, lastline=]{../Interpreter.ml}

  \subsection{Formally: sequential case}
  \begin{align}
  \end{align} 

  \subsection{Formally: concurrent case}
  \begin{align}
  \end{align} 

\section{SemanticsRule.SWhile \label{sec:SemanticsRule.SWhile}}

    \subsection{Prose}
    All of the following apply:
    \begin{itemize}
    \item [s] is a while statement;
    \item [s] gives an expression [e] and a loop body [body];
    \item [new\_env] is [env] modified after evaluation of the loop [(e,body)].
    \end{itemize}

    \subsection{Comments}

    \subsection{Example: SemanticsRule.SWhile.asl}
    The program:
    \VerbatimInput{../tests/ASLSemanticsReference.t/SemanticsRule.SWhile.asl}
    prints ``0123''.

  \subsection{Code}
  \VerbatimInput[firstline=, lastline=]{../Interpreter.ml}

  \subsection{Formally: sequential case}
  \begin{align}
  \end{align} 

  \subsection{Formally: concurrent case}
  \begin{align}
  \end{align} 

\section{SemanticsRule.SRepeat \label{sec:SemanticsRule.SRepeat}}

    \subsection{Prose}
    All of the following apply:
    \begin{itemize}
    \item [s] is a repeat statement;
    \item [s] gives an expression [e] and a loop body [body];
    \item [new\_env] is [env] modified after evaluation of the loop [(e,body)].
    \end{itemize}

    \subsection{Comments}

    \subsection{Example: SemanticsRule.SRepeat.asl}
    The program:
    \VerbatimInput{../tests/ASLSemanticsReference.t/SemanticsRule.SRepeat.asl}
    prints ``0123''.

  \subsection{Code}
  \VerbatimInput[firstline=, lastline=]{../Interpreter.ml}

  \subsection{Formally: sequential case}
  \begin{align}
  \end{align} 

  \subsection{Formally: concurrent case}
  \begin{align}
  \end{align} 

\section{SemanticsRule.SFor \label{sec:SemanticsRule.SFor}}

    \subsection{Prose}
    All of the following apply:
    \begin{itemize}
    \item [s] is a for statement;
    \item [s] gives [(id,e1,dir,e2,s)];
    \item <description>
    \end{itemize}

    \subsection{Comments}

    \subsection{Example: SemanticsRule.SFor.asl}
    The program:
    \VerbatimInput{../tests/ASLSemanticsReference.t/SemanticsRule.SFor.asl}
    prints ``0123''.

  \subsection{Code}
  \VerbatimInput[firstline=, lastline=]{../Interpreter.ml}

  \subsection{Formally: sequential case}
  \begin{align}
  \end{align} 

  \subsection{Formally: concurrent case}
  \begin{align}
  \end{align} 

\section{SemanticsRule.SThrowNone \label{sec:SemanticsRule.SThrowNone}}

    \subsection{Prose}
    All of the following apply:
    \begin{itemize}
    \item [s] is a throw statement;
    \end{itemize}

    \subsection{Comments}

    \subsection{Example: SemanticsRule.SThrowNone.asl}
    The program:
    \VerbatimInput{../tests/ASLSemanticsReference.t/SemanticsRule.SThrowNone.asl} 
    throws a [MyException] exception.

  \subsection{Code}
  \VerbatimInput[firstline=, lastline=]{../Interpreter.ml}

  \subsection{Formally: sequential case}
  \begin{align}
  \end{align} 

  \subsection{Formally: concurrent case}
  \begin{align}
  \end{align} 

\section{SemanticsRule.SThrowSomeTyped \label{sec:SemanticsRule.SThrowSomeTyped}}

    \subsection{Prose}
    All of the following apply:
    \begin{itemize}
    \item [s] is a throw statement;
    \end{itemize}

    \subsection{Comments}

    \subsection{Example: SemanticsRule.SThrowSomeTyped.asl}
    The program:
    \VerbatimInput{../tests/ASLSemanticsReference.t/SemanticsRule.SThrowSomeTyped.asl}
    throws a [MyException {a: 3, b: 42}] exception.

  \subsection{Code}
  \VerbatimInput[firstline=, lastline=]{../Interpreter.ml}

  \subsection{Formally: sequential case}
  \begin{align}
  \end{align} 

  \subsection{Formally: concurrent case}
  \begin{align}
  \end{align} 

\section{SemanticsRule.SThrowSome \label{sec:SemanticsRule.SThrowSome}}

    \subsection{Prose}
    All of the following apply:
    \begin{itemize}
    \item [s] is a throw statement;
    \end{itemize}

    \subsection{Comments}

    \subsection{Examples}
    <example>

  \subsection{Code}
  \VerbatimInput[firstline=, lastline=]{../Interpreter.ml}

  \subsection{Formally: sequential case}
  \begin{align}
  \end{align} 

  \subsection{Formally: concurrent case}
  \begin{align}
  \end{align} 

\section{SemanticsRule.STry \label{sec:SemanticsRule.STry}}

    \subsection{Prose}
    All of the following apply:
    \begin{itemize}
    \item [s] is a try statement;
    \item [s] gives [(s', catchers, otherwise\_opt)];
    \item [s\_m] is the evaluation of the block [s'] under [env];
    \item [new\_env] is [env] modified after evaluation of the catchers [(catchers
      otherwise\_opt s\_m)].
    \end{itemize}

    \subsection{Comments}

    \subsection{Example: SemanticsRule.STry.asl}
    The program:
    \VerbatimInput{../tests/ASLSemanticsReference.t/SemanticsRule.STry.asl}
    prints ``3''.

  \subsection{Code}
  \VerbatimInput[firstline=, lastline=]{../Interpreter.ml}

  \subsection{Formally: sequential case}
  \begin{align}
  \end{align} 

  \subsection{Formally: concurrent case}
  \begin{align}
  \end{align} 

\section{SemanticsRule.SDeclSome \label{sec:SemanticsRule.SDeclSome}}

    \subsection{Prose}
    All of the following apply:
    \begin{itemize}
    \item [s] is a declaration;
    \item [s] gives [(\_ldk, ldi, Some e)];
    \item [new\_env] is [env] modified after evaluation of the local declaration
      [ldi env (Some m)].
    \end{itemize}

    \subsection{Comments}

    \subsection{Examples}
    [let x = 3;] binds [x] to [3] in the empty environment.

  \subsection{Code}
  \VerbatimInput[firstline=, lastline=]{../Interpreter.ml}

  \subsection{Formally: sequential case}
  \begin{align}
  \end{align} 

  \subsection{Formally: concurrent case}
  \begin{align}
  \end{align} 

\section{SemanticsRule.SDeclNone \label{sec:SemanticsRule.SDeclNone}}

    \subsection{Prose}
    All of the following apply:
    \begin{itemize}
    \item [s] is a declaration;
    \item [s] gives [(\_dlk, ldi, None)];
    \item [new\_env] is [env] modified after evaluation of the local declaration
      [ldi env None]. 
    \end{itemize}

    \subsection{Comments}

    \subsection{Examples}
    [var x : integer;] binds [x] in [env] to the base value of [integer].

\chapter{Evaluation of Blocks \label{chap:eval_block}}
 
\section{SemanticsRule.Block \label{sec:SemanticsRule.Block}}

    \subsection{Prose}
    [eval\_block env stm] is [new\_env] and all of the following applies:
    \begin{itemize}
    \item [block\_env] is [env] such that the scope given by [env] is pushed onto the
      declaration stack;
    \item [block\_env'] is [env] modified after the evaluation of the statement [stm];
    \item [new\_env] is [block\_env'] after restoring the variable bindings of [env],
    with the updated values of [block\_env'].
    \end{itemize}
 
    \subsection{Comments}

    \subsection{Examples}
    <example>

\chapter{Evaluation of [while] and [repeat] loops \label{chap:eval_loop}}

\section{SemanticsRule.Loop \label{sec:SemanticsRule.Loop}}
[eval\_loop is\_while env e\_cond body] evaluates [body] in [env]: this is either
an interruption [Returning vs] or a new environment [new\_env].

    \subsection{Prose}
    Formally, [cond\_m] evaluates to [e\_cond] or [not e\_cond] as determined by
    [is\_while] and one of the following applies:

    \begin{itemize}
    \item All of the following apply:
      \begin{itemize}
      \item [cond\_m] evaluates to [false];
      \item [new\_env] is [env]---the loop is exited. 
      \end{itemize}
    \item All of the following apply:
      \begin{itemize}
      \item [cond\_m] evaluates to [true];
      \item [env1] is [env] modified after the evaluation of the statement
       [body]---this step might affect the value of [cond\_m] eventually leading
       to exiting the loop;
      \item [new\_env] is [env1] modified after the evaluation of [eval\_loop
        is\_while env e\_cond body].
      \end{itemize}
    \end{itemize}

    \subsection{Comments}

    \subsection{Example: SemanticsRule.Loop.asl}
    The program:
    \VerbatimInput{../tests/ASLSemanticsReference.t/SemanticsRule.Loop.asl}
    prints ``0123''.

\chapter{Evaluation of [for] loops \label{chap:eval_for}} 

\section{SemanticsRule.For \label{sec:SemanticsRule.For}}
[eval\_for undet env index\_name v\_start dir v\_end body] evaluates [body] in
[env]: this is either an interruption [Returning vs] or a new environment
[new\_env].

    \subsection{Prose}
    Formally, [cond\_m] evaluates to [leq v\_end v\_start] or [geq v\_end v\_start]
    as determined by [dir] and one of the following applies:

    \begin{itemize}
    \item All of the following apply:
      \begin{itemize}
      \item [cond\_m] evaluates to [true];
      \item [new\_env] is [env] as the loop is exited.
      \end{itemize}
    \item All of the following apply:
      \begin{itemize}
      \item [cond\_m] evaluates to [false];
      \item [env1] is [env] modified after the evaluation of the statement [body];
      \item [env2] is [env1] modified such that [index\_name] is bound to [v\_step];
      \item [v\_step] evaluates to [v\_start+1] or [v\_start-1] as determined by [dir];
      \item [new\_env] is [env2] modified after the evaluation of [eval\_for undet
        env index\_name v\_step dir v\_end body].
      \end{itemize}
    \end{itemize}

    \subsection{Comments}

    \subsection{Example: SemanticsRule.For.asl}
    The program:
    \VerbatimInput{../tests/ASLSemanticsReference.t/SemanticsRule.For.asl}
    prints ``3''.

\chapter{Evaluation of Catchers \label{chap:eval_catchers}}
[eval\_catchers env catchers otherwise\_opt s\_m], given the result [s\_m] of the
evaluation of a statement under environment [env] is [res] which is either a
[Throwing (v, v\_ty, env\_throw)], an interruption [Returning vs] or a new
environment [new\_env]. Formally, one of the following applies:
\begin{itemize}
\item SemanticsRule.Catch (see Section~\ref{sec:SemanticsRule.Catch}),
\item SemanticsRule.CatchNamed (see Section~\ref{sec:SemanticsRule.CatchNamed}),
\item SemanticsRule.CatchOtherwise (see Section~\ref{sec:SemanticsRule.CatchOtherwise}),
\item SemanticsRule.CatchNone (see Section~\ref{sec:SemanticsRule.CatchNone}),
\item SemanticsRule.CatchNoThrow (see Section~\ref{sec:SemanticsRule.CatchNoThrow}).
\end{itemize}

\section{SemanticsRule.Catch \label{sec:SemanticsRule.Catch}}

    \subsection{Prose}
    All of the following apply:
    \begin{itemize}
    \item [s\_m] is [Throwing (v, v\_ty, env\_throw)];
    \item [catcher] is the first catcher in [catchers] that matches [v\_ty];
    \item [catcher] does not declare a name;
    \item [catcher] gives a statement [s];
    \item One of the following applies:
      \begin{itemize}
      \item [env\_throw] and [env] have the same scope, and [env1] is [env\_throw];
      \item [env1] is the environment formed with the global part of [env\_throw]
        and the local part of [env];
      \end{itemize}
    \item One of the following applies:
      \begin{itemize}
      \item [Throwing (None, None, env\_throw1)] is the result of the evaluation of
        the block [s] in [env1], and [res] is [Throwing (v, v\_ty, env\_throw1)];
      \item [res] is the result of the evaluation of the block [s] in [env1].
      \end{itemize}
    \end{itemize}

    \subsection{Comments}

    \subsection{Example: SemanticsRule.Catch.asl}
    The program:
    \VerbatimInput{../tests/ASLSemanticsReference.t/SemanticsRule.Catch.asl}
    prints ``MyException''.

  \subsection{Code}
  \VerbatimInput[firstline=, lastline=]{../Interpreter.ml}

  \subsection{Formally: sequential case}
  \begin{align}
  \end{align} 

  \subsection{Formally: concurrent case}
  \begin{align}
  \end{align} 

\section{SemanticsRule.CatchNamed \label{sec:SemanticsRule.CatchNamed}}

    \subsection{Prose}
    All of the following apply:
    \begin{itemize}
    \item [s\_m] is [Throwing (v, v\_ty, env\_throw)];
    \item [catcher] is the first catcher in [catchers] that matches [v\_ty];
    \item [catcher] declares a name [name];
    \item [catcher] gives a statement [s];
    \item One of the following applies:
      \begin{itemize}
      \item [env\_throw] and [env] have the same scope, and [env1] is [env\_throw];
      \item [env1] is the environment formed with the global part of [env\_throw]
        and the local part of [env];
      \item [env2] is [env1] modified after binding locally [name] to the exception
        [v] raised by [s\_m];
      \end{itemize}
    \item One of the following applies:
      \begin{itemize}
      \item [Throwing (None, None, env\_throw1)] is the result of the evaluation of
        the block [s] in [env2], and [res] is [Throwing (v, v\_ty, env\_throw1)];
      \item [env3] is [env2] modified after the evaluation of the block [s] in
        [env2], and [new\_env] is [env3] modified after unbinding [name] from
        [env3].
      \item [res] is the result of the evaluation of the block [s] in [env2].
      \end{itemize}
    \end{itemize}

    \subsection{Comments}

    \subsection{Example: SemanticsRule.CatchNamed.asl}
    The program:
    \VerbatimInput{../tests/ASLSemanticsReference.t/SemanticsRule.CatchNamed.asl}
    prints ``My exception with my message''.

  \subsection{Code}
  \VerbatimInput[firstline=, lastline=]{../Interpreter.ml}

  \subsection{Formally: sequential case}
  \begin{align}
  \end{align} 

  \subsection{Formally: concurrent case}
  \begin{align}
  \end{align} 

\section{SemanticsRule.CatchOtherwise \label{sec:SemanticsRule.CatchOtherwise}}

    \subsection{Prose}
    All of the following apply:
    \begin{itemize}
    \item [s\_m] is [Throwing (v, v\_ty, env\_throw)];
    \item [otherwise\_opt] is [Some s];
    \item no catcher matches [v\_ty];
    \item One of the following applies:
      \begin{itemize}
      \item [env\_throw] and [env] have the same scope, and [env1] is [env\_throw];
      \item [env1] is the environment formed with the global part of [env\_throw]
        and the local part of [env];
      \end{itemize}
    \item One of the following applies:
      \begin{itemize}
      \item [Throwing (None, None, env\_throw1)] is the result of the evaluation of
        the block [s] in [env1], and [res] is [Throwing (v, v\_ty, env\_throw1)];
      \item [res] is the result of the evaluation of the block [s] in [env1].
      \end{itemize}
    \end{itemize}

    \subsection{Comments}

    \subsection{Example: SemanticsRule.CatchOtherwise.asl}
     The program:
     \VerbatimInput{../tests/ASLSemanticsReference.t/SemanticsRule.CatchOtherwise.asl}
     prints ``Another exception''.

  \subsection{Code}
  \VerbatimInput[firstline=, lastline=]{../Interpreter.ml}

  \subsection{Formally: sequential case}
  \begin{align}
  \end{align} 

  \subsection{Formally: concurrent case}
  \begin{align}
  \end{align} 

\section{SemanticsRule.CatchNone \label{sec:SemanticsRule.CatchNone}}

    \subsection{Prose}
    All of the following apply:
    \begin{itemize}
    \item [s\_m] is [Throwing (v, v\_ty, env\_throw)];
    \item [otherwise\_opt] is [None];
    \item no catcher matches [v\_ty];
    \item [new\_env] is [env].
    \end{itemize}

    \subsection{Comments}

    \subsection{Example: SemanticsRule.CatchNone.asl}
    The program:
    \VerbatimInput{../tests/ASLSemanticsReference.t/SemanticsRule.CatchNone.asl}
    does not print anything.

  \subsection{Code}
  \VerbatimInput[firstline=, lastline=]{../Interpreter.ml}

  \subsection{Formally: sequential case}
  \begin{align}
  \end{align} 

  \subsection{Formally: concurrent case}
  \begin{align}
  \end{align} 

\section{SemanticsRule.CatchNoThrow \label{sec:SemanticsRule.CatchNoThrow}}

    \subsection{Prose}
    All of the following apply:
    \begin{itemize}
    \item [s\_m] is not [Throwing];
    \item [res] is [s\_m].
    \end{itemize}

    \subsection{Comments}

    \subsection{Example: SemanticsRule.CatchNoThrow.asl}
    The program:
    \VerbatimInput{../tests/ASLSemanticsReference.t/SemanticsRule.CatchNoThrow.asl}
    prints ``No exception raised''.

\chapter{Evaluation of Functions \label{chap:eval_func}}
[eval\_func genv name pos actual\_args params] evaluates the subprogram named
[name] in the global environment [genv], with [actual\_args] the list of actual
arguments, and [params] the list of arguments deduced by type equality. This is
a new global environment [new\_genv] and a list of values [vs], or an error is
raised. Formally, one of the following applies:
\begin{itemize}
\item SemanticsRule.FUndefIdent (see Section~\ref{sec:SemanticsRule.FUndefIdent}),
\item SemanticsRule.FPrimitive (see Section~\ref{sec:SemanticsRule.FPrimitive}),
\item SemanticsRule.FBadArity (see Section~\ref{sec:SemanticsRule.FBadArity}),
\item SemanticsRule.FCall (see Section~\ref{sec:SemanticsRule.FCall}).
\end{itemize}

\section{SemanticsRule.FUndefIdent \label{sec:SemanticsRule.FUndefIdent}}

  \subsection{Prose}
  All of the following apply:
  \begin{itemize}
  \item [name] is undeclared in [genv];
  \item an [UndefinedIdentifier] error is raised.
  \end{itemize}

  \subsection{Comments}

  \subsection{Example: SemanticsRule.FUndefIdent.asl}
  The program:
  \VerbatimInput{../tests/ASLSemanticsReference.t/SemanticsRule.FUndefIdent.asl}
  raises an [UndefinedIdentifier "Foo"] error.

  \subsection{Code}
  \VerbatimInput[firstline=, lastline=]{../Interpreter.ml}

  \subsection{Formally: sequential case}
  \begin{align}
  \end{align} 

  \subsection{Formally: concurrent case}
  \begin{align}
  \end{align} 

\section{SemanticsRule.FPrimitive \label{sec:SemanticsRule.FPrimitive}}

  \subsection{Prose}
  All of the following apply:
  \begin{itemize}
  \item [name] is bound in [genv] to a primitive subprogram with a body [body];
  \item [new\_genv] is [genv];
  \item [vs] is the application of [body] on [actual\_args].
  \end{itemize}

  \subsection{Comments}

  \subsection{Examples}
  [print (2,3);] calls the primitive [print] on the evaluation of [(2,3)].

  \subsection{Code}
  \VerbatimInput[firstline=, lastline=]{../Interpreter.ml}

  \subsection{Formally: sequential case}
  \begin{align}
  \end{align} 

  \subsection{Formally: concurrent case}
  \begin{align}
  \end{align} 

\section{SemanticsRule.FBadArity \label{sec:SemanticsRule.FBadArity}}

  \subsection{Prose}
  All of the following apply:
  \begin{itemize}
  \item [name] is bound in [genv] to a subprogram with a list of formal arguments
    [arg\_decls];
  \item the lists [actual\_args] and [arg\_decls] do not have the same length;
  \item a "BadArity" error is raised.
  \end{itemize}

  \subsection{Comments}

  \subsection{Examples}
  [Zeros(3,4);] raises a [BadArity "Zeros" 1 2] error.

  \subsection{Code}
  \VerbatimInput[firstline=, lastline=]{../Interpreter.ml}

  \subsection{Formally: sequential case}
  \begin{align}
  \end{align} 

  \subsection{Formally: concurrent case}
  \begin{align}
  \end{align} 

\section{SemanticsRule.FCall \label{sec:SemanticsRule.FCall}}

  \subsection{Prose}
  All of the following apply:
  \begin{itemize}
  \item [name] is bound in [genv] to a subprogram with a list of formal arguments
    [arg\_decls] and a body statement [body];
  \item [env1] is the environment made of [genv] and the empty local environment,
  \item [env2] is [env1] modified so that each formal argument in [arg\_decls] is
    locally bound to the corresponding actual argument in [actual\_args];
  \item [env3] is [env2] modified so that each parameter in [params] is declared;
  \item [res] is the evaluation of [body] in [env3] and one of the following
    applies:
      \begin{itemize}
      \item [res] is an environment [env4] and [new\_genv] is the global environment
        given by [env4]---e.g. where the subprogram called is either a setter or
        a procedure;
      \item [res] is an interruption [Returning(xs,ret\_genv)] and [new\_genv] is
        [ret\_genv]---this is the general case.
      \end{itemize}
  \end{itemize}

  \subsection{Example: SemanticsRule.FCall.asl}
  The program:
  \VerbatimInput{../tests/ASLSemanticsReference.t/SemanticsRule.FCall.asl}

  \subsection{Code}
  \VerbatimInput[firstline=, lastline=]{../Interpreter.ml}

  \subsection{Formally: sequential case}
  The evaluation of a n-ary subprogram evaluates the arguments in order then
  calls the subprogram:
  \begin{align}
  \interp{f (e_1, \dotsc e_n)} (E_0) & \triangleq
    \left\{ (v, E') \st{}
    \begin{aligned}
      & \forall i \in \llbracket 1, n \rrbracket, (v_i, E_i) \in \interp{e_i} (E_{i - 1})
      \\ \text{and}\ &
      (v, E') \in \interp{\left\langle f, v_1, \dotsc v_n \right\rangle} \left(E_n\right)
    \end{aligned}
    \right\}
  \label{eq:sem-seq-fcall}
  \end{align} 

  \subsection{Formally: concurrent case}
  For $i$ ranging implicitely from 1 to $n$ included, a call to a subprogram $f$
  is interpreted as the interpretation of the subprogram call (see to $f$ after
  evaluating every argument in order:

  \begin{equation}
    \interp{f (e_i)} (E_0) \triangleq
      \left\{ (v, E', S) \st{}
      \begin{aligned}
        & % \forall i \in \llbracket 1, n \rrbracket,
          (v_i, E_i, S_i) \in \interp{e_i} (E_{i - 1})
        \\ \text{and}\ &
        (v, E', S) \in \interp{\left\langle f, (v_i,S_i) \right\rangle} \left(E_n\right)
      \end{aligned}
      \right\}
    \label{eq:sem-conc-fcall}
  \end{equation}

  \subsection{Comments}
  This aims to encompass LRM R\_DFWZ.

\end{document}
