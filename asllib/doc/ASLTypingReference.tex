\documentclass{book}
\usepackage{amsmath}  % Classic math package
\usepackage{amssymb}  % Classic math package
\usepackage{mathtools}  % Additional math package
\usepackage{graphicx}  % For figures
\usepackage{caption}  % For figure captions
\usepackage{subcaption}  % For subfigure captions
\usepackage{url}  % Automatically escapes urls
\usepackage{hyperref}  % Insert links inside pdfs
\usepackage[inline]{enumitem}  % For inline lists
\usepackage[export]{adjustbox}  % For centering too wide figures
\usepackage{mathpartir}  % For deduction rules and equations paragraphs
\usepackage{comment}
\usepackage{fancyvrb}
\input{ASLTypingLines}
\input{ASLTypeSatisfactionLines}

\usepackage{enumitem}
\renewlist{itemize}{itemize}{10}
\setlist[itemize,1]{label=\textbullet}
\setlist[itemize,2]{label=\textasteriskcentered}
\setlist[itemize,3]{label=\textendash}
\setlist[itemize,4]{label=$\triangleright$}
\setlist[itemize,5]{label=+}
\setlist[itemize,6]{label=\textbullet}
\setlist[itemize,7]{label=$\ast$}
\setlist[itemize,8]{label=\textendash}
\setlist[itemize,9]{label=$\triangleright$}
\setlist[itemize,10]{label=+}


%Macros
\newcommand\ie{i.\,e.}
\newcommand\eg{e.\,g.}
\newcommand\synor{\ |\ }
\newcommand\syntt[1]{\mathtt{#1}}
\newcommand\ife[3]{\text{if}\ #1\ \text{then}\ #2\ \text{else}\ #3\ \text{end}}
\newcommand\inenv[2]{\left\langle #1, #2 \right\rangle}
\newcommand\env[1]{\left\langle #1 \right\rangle}
\newcommand\reducesto{\ \to\ }
\newcommand\llbracket{[|}
\newcommand\rrbracket{|]}
\newcommand\interp[1]{\left\llbracket #1 \right\rrbracket}
\newcommand\st[0]{\ \middle|\ }
\newcommand\field[1]{.\text{#1}}
\newcommand\globals[0]{\field{globals}}
\newcommand\locals[0]{\field{locals}}
\newcommand\X[0]{\mathcal{X}}
\newcommand\N[0]{\mathbb{N}}
\newcommand\asldata[0]{\mathtt{asl\_data}}
\newcommand\aslctrl[0]{\mathtt{asl\_ctrl}}
\newcommand\aslpo[0]{\mathtt{asl\_po}}
\DeclareMathOperator{\dom}{dom}
\newcommand{\tests}{../tests/ASLTypingReference.t/}
\author{Arm Architecture Technology Group}
\title{ASL Typing Reference}
\begin{document}
\maketitle
\tableofcontents{}

\chapter{Copyright and Disclaimer}

The material in this document is copyright Arm Ltd 2023, all rights reserved.

This material covers both ASLv0 (viz, the existing ASL pseudocode language
which appears in the Arm Architecture Reference Manual) and ASLv1, a new,
experimental, and as yet unreleased version of ASL.

This material is work in progress, more precisely at pre-Alpha quality as
per Arm’s quality standards. In particular, this means that it would be
premature to base any production tool development on this material.

However, any feedback, question, query and feature request would be most
welcome; those can be sent to Arm’s Architecture Formal Team Lead Jade
Alglave \texttt{(jade.alglave@arm.com)} or by raising issues or PRs to the herdtools7
github repository.

\chapter{Preamble}

\section{Environments}

\section{Annotation}
Typing a program is typing its ``main'' subprogram.  Constructively, typing a
program requires following its Abstract Syntax Tree and typing each of its
components.

The types of a program are given by applying a set of
\texttt{annotate\_<object>} functions. Each \texttt{annotate\_<object>}
function describes how to annotate a specific object, as follows.
\begin{itemize}
\item \texttt{annotate\_expr} annotates expressions;
\item \texttt{annotate\_slices} annotates slices;
\item \texttt{annotate\_pattern} annotates pattern;
\item \texttt{annotate\_local\_decl\_item} annotates local declarations;
\item \texttt{annotate\_lexpr} annotates left-hand sides of assignments;
\item \texttt{annotate\_stmt} annotates statements;
\item \texttt{annotate\_block} annotates blocks;
\item \texttt{annotate\_catcher} annotates catchers;
\item \texttt{annotate\_call} annotates functions calls;
\item \texttt{annotate\_func} annotates functions.
\end{itemize}

This aims to encompass LRM 7.4.2. R\_VDPC.

\section{Reading guide}

The definition of each \texttt{annotation\_<object>} function is given by a number of
rules, which follow the possible shapes the \texttt{object} can have. For 
example, an expression can be a literal, or a binary operator, amongst other
things. Each of those has its own evaluation rule: TypingRule.Lit in
Section~\ref{sec:TypingRule.Lit}, Typing.Binop in
Section~\ref{sec:TypingRule.Binop} respectively.

Each rule is presented using the following template:
\begin{itemize}
\item a Prose paragraph gives the rule in English, and corresponds as much as possible to the code of the reference implementation ASLRef given at \url{~/herdtools7/asllib};
\item one or several Examples, which as much as possible are also given as regression tests in \url{~/herdtools7/asllib/tests/ASLTypingReference.t}
\item a Code paragraph which gives a verbatim of the corresponding implementation in the type-checker of ASLRef \url{~/herdtools7/asllib/Typing.ml};
\item Formal paragraphs which give formal definitions of the rule.
\end{itemize}

\chapter{Abstract Syntax}
In order to type both ASLv0 and ASLv1, we map the concrete syntax of both versions into an abstract syntax, which is a form of labelled tree. The abstract syntax tree abstracts away many syntactic elements that are only needed to enable parsing.

\newcommand\BNOT[0]{\texttt{"!"}}
\newcommand\NEG[0]{\texttt{"-"}}
\newcommand\NOT[0]{\texttt{"NOT"}}

\newcommand\unop[0]{\textrm{unop}}
\newcommand\binop[0]{\textrm{binop}}
\newcommand\binopin[0]{\textrm{binop\_in}}
\newcommand\binoppow[0]{\textrm{binop\_pow}}

% AST-only nodes
\newcommand\literal[0]{\texttt{literal}}
\newcommand\expr[0]{\texttt{expr}}

%The following node types directly correspond to their respective ASL grammar variables (non-terminals):  $\unop$, $\binop$, $\binopin$, $\binoppow$
The following node types directly correspond to their respective ASL grammar variables (non-terminals):  $\unop$, $\binop$, $\binopin$, $\binoppow$

\[
%\unop = \BNOT \;|\; \NEG \;|\; \NOT
\begin{array}{rcl}
\literal &=& \texttt{<int\_lit>} \;|\; \texttt{<boolean\_lit>} \;|\; \texttt{<real\_lit>} \;|\; \texttt{<bitvector\_lit>} \;|\; \texttt{<string\_lit>}\\
\expr &=& \texttt{E\_Literal}(literal) \;|\; \texttt{E\_Var}(\texttt{<identifier>})
\end{array}
\]

TODO: what about \texttt{hex\_lit}?

\chapter{Type Algebra}

\section{TypingRule.BuiltinSingularType \label{sec:TypingRule.BuiltinSingularType}}

    \subsection{Prose}
    The \emph{builtin singular types} are:
    \begin{itemize}
    \item integer;
    \item real;
    \item string;
    \item boolean;
    \item bits;
    \item enumeration.
    \end{itemize}

    \subsection{Example: TypingRule.BuiltinSingularTypes.asl}
\VerbatimInput[firstline=3,lastline=8]{\tests/TypingRule.BuiltinSingularTypes.asl}
Variables of  buitin types \texttt{integer}, \texttt{real}, \texttt{boolean}, \texttt{bits(4)} and \texttt{bits(2)} are defined.

    \subsection{Example: TypingRule.EnumerationType.asl}
\VerbatimInput{\tests/TypingRule.EnumerationType.asl}
The type \texttt{color} consists in two different constants \texttt{RED} and~\texttt{BLACK}.

    \subsection{Code}

    \subsection{Formally}

    \subsection{Comments}
    This aims to encompass LRM Section 3.1 D\_PQCK and D\_NZWT.

\section{TypingRule.BuiltinAggregateType \label{sec:TypingRule.BuiltinAggregateType}}

    \subsection{Prose}
    The builtin aggregate types are:
    \begin{itemize}
    \item tuple;
    \item array;
    \item record;
    \item exception.
    \end{itemize}

    \subsection{Example: TypingRule.BuiltinAggregateTypes.asl}
      \VerbatimInput{\tests/TypingRule.BuiltinAggregateTypes.asl}
      Type \texttt{pair} is the type of integer and booleans pairs. Notice that
      the syntax of types and expressions are similar.

Arrays are indexed either by integers from 0 to array size as specified in type
declaration, as illustrated by the type \texttt{T}, or by the elements of an
enumeration type, as illustrated by type~\texttt{pointCoord}.

The type~\texttt{pointRecord} is defined as a record type with  three fields
\texttt{x}, \texttt{y} and~\texttt{z}.

    \subsection{Example: TypingRule.BuiltinExceptionType.asl}
    \VerbatimInput{\tests/TypingRule.BuiltinExceptionType.asl}
    Two exception types are defined: exceptions \texttt{Not\_found} carry no
values, while exceptions \texttt{Error}  carry a messsage. Notice the
similarity with record types and that the empty field list \texttt{\{\}} can be
omitted in type declarations, as it is the case for \texttt{Not\_found}.

    \subsection{Code}

    \subsection{Formally}

    \subsection{Comments}
    This aims to encompass LRM Section 3.1 D\_PQCK and D\_KNBD.

\section{TypingRule.BuiltinSingularOrAggregate \label{sec:TypingRule.BuiltinSingularOrAggregate}}

    \subsection{Prose}
    \texttt{t} is a builtin type and one of the following applies:
    \begin{itemize}
    \item \texttt{t} is singular;
    \item \texttt{t} is aggregate.
    \end{itemize}
    
    \subsection{Example}

    \subsection{Code}

    \subsection{Formally}

     \[
     \begin{array}{l}
     t \in \texttt{ty} \;\;\;\; \texttt{is\_builtin\_singular}(t) \vdash \texttt{is\_builtin}(t)\\
     t \in \texttt{ty} \;\;\;\; \texttt{is\_builtin\_aggregate}(t) \vdash \texttt{is\_builtin}(t)\\
     \end{array}
     \]
   
\section{TypingRule.NamedType \label{sec:TypingRule.NamedType} } 

    \subsection{Prose}
    A named type is a type which is declared using the \texttt{type} syntax.

    \subsection{Example}

    \subsection{Code}

    \subsection{Formally}

    \subsection{Comments}
    This aims to encompass LRM Section 7.1.1 D\_VMZX.

\section{TypingRule.AnonymousType \label{sec:TypingRule.AnonymousType}}

    \subsection{Prose} 
    An anonymous type is a type which is not declared using the type syntax. 

    \subsection{Example}

    \subsection{Code}

    \subsection{Formally}

    \subsection{Comments}
    This aims to encompass LRM Section 7.1.1 D\_VMZX.

\section{TypingRule.SingularType}

    \subsection{Prose}
    A type \texttt{t} is singular if one of the following applies:
    \begin{itemize}
    \item \texttt{t} is a builtin singular type;
    \item All of the following apply:
      \begin{itemize}
      \item \texttt{t} is a named type;
      \item \texttt{t\_struct} is the structure of \texttt{t};
      \item \texttt{t\_struct} is a builtin singular.
      \end{itemize} 
    \end{itemize}

    \subsection{Example}

    \subsection{Code}

    \subsection{Formally}

    \subsection{Comments}
    This aims to encompass LRM Section 3.1 R\_GVZK.

\section{TypingRule.AggregateType}

    \subsection{Prose}
    A type \texttt{t} is aggregate if one of the following applies:
    \begin{itemize}
    \item \texttt{t} is a builtin aggregate type;
    \item All of the following apply:
      \begin{itemize}
      \item \texttt{t} is a named type;
      \item \texttt{t\_struct} is the structure of \texttt{t};
      \item \texttt{t\_struct} is a builtin aggregate. 
      \end{itemize}
    \end{itemize}

    \subsection{Example}

    \subsection{Code}

    \subsection{Formally}

    \subsection{Comments}
    This aims to encompass LRM Section 3.1 R\_GVZK.

\section{TypingRule.NonPrimitiveType}

    \subsection{Prose} 
    A type \texttt{t} is non-primitive if one of the following applies:
    \begin{itemize}
    \item \texttt{t} is a named type;
    \item All of the following apply:
      \begin{itemize}
      \item \texttt{t} is a tuple \texttt{li};
      \item there exists a non-primitive type in \texttt{li};
      \end{itemize}
    \item All of the following apply:
      \begin{itemize}
      \item \texttt{t} is an array of type \texttt{ty}
      \item \texttt{ty} is non-primitive; 
      \end{itemize}
    \item All of the following apply:
      \begin{itemize}
      \item \texttt{t} is a record with fields \texttt{fields};
      \item there exists a non-primitive type in \texttt{fields};
      \end{itemize}
    \item All of the following apply:
      \begin{itemize}
      \item \texttt{t} is an exception with fields \texttt{fields};
      \item there exists a non-primitive type in \texttt{fields};
      \end{itemize}
    \end{itemize}

    \subsection{Example}

    \subsection{Code}

    \subsection{Formally}

    \subsection{Comments}
    This aims to encompass LRM Section 7.1.1 D\_GWXK.

\section{TypingRule.PrimitiveType}

    \subsection{Prose} 
    A type \texttt{t} is primitive if it is not non-primitive.

    \subsection{Example}

    \subsection{Code}

    \subsection{Formally}

    \subsection{Comments}
    This aims to encompass LRM Section 7.1.1 D\_GWXK.

\section{TypingRule.Canonical}

    \subsection{Prose}

    \subsection{Example}

    \subsection{Code}

    \subsection{Formally}

    \subsection{Comments}

\section{TypingRule.Structure}

    \subsection{Prose}
    \texttt{ty} is a type and its structure is \texttt{t\_struct} and one of the following
    applies:
    \begin{itemize}
    \item All of the following apply:
      \begin{itemize}
      \item \texttt{ty} is a named type \texttt{x};
      \item One of the following applies:
        \begin{itemize}
        \item All of the following apply:
          \begin{itemize}
          \item \texttt{x} is not declared in the global environment; 
          \item an error "Undefined Identifier" is raised;
          \end{itemize}
        \item All of the following apply:
          \begin{itemize}
          \item \texttt{x} is declared in the global environment as some type \texttt{ty'};
          \item \texttt{t\_struct} is the structure of \texttt{ty'};
          \end{itemize}
       \end{itemize}
      \end{itemize}
    \item All of the following apply:
      \begin{itemize}
      \item \texttt{t} is a builtin singular type;
      \item \texttt{t\_struct} is \texttt{ty};
      \end{itemize}
    \item All of the following apply:
      \begin{itemize}
      \item \texttt{ty} is a tuple with \texttt{subtypes};
      \item \texttt{t\_struct} is a tuple with the structure of each element in \texttt{subtypes};
      \end{itemize}
    \item All of the following apply:
      \begin{itemize}
      \item \texttt{ty} is an array with \texttt{t};
      \item \texttt{t\_struct} is an array with the structure of \texttt{t};
      \end{itemize}
    \item All of the following apply:
      \begin{itemize}
      \item \texttt{ty} is a record with \texttt{fields};
      \item \texttt{t\_struct} is a record with ;
      \end{itemize}
    \item All of the following apply:
      \begin{itemize}
      \item \texttt{ty} is an exception with \texttt{fields};
      \item \texttt{t\_struct} is a record with ;
      \end{itemize}
    \end{itemize}
      
    \subsection{Examples}
    In this example:
    \texttt{type T1 of integer;} is the named type \texttt{T1}
whose structure is \texttt{integer}.

    In this example:
    \texttt{type T2 of (integer, T1);}
    is the named type \texttt{T2} whose structure is (integer, integer). In this
    example, \texttt{(integer, T1)} is non-primitive since it uses \texttt{T1}, which is builtin aggregate.

    In this example:
    \texttt{var x: T1;}
    the type of \texttt{x} is the named (hence non-primitive) type \texttt{T1}, whose structure
    is \texttt{integer}.

    In this example:
    \texttt{var y: integer;}
    the type of \texttt{y} is the anonymous primitive type \texttt{integer}.

    In this example:
    \texttt{var z: (integer, T1);}
    the type of \texttt{z} is the anonymous non-primitive type
\texttt{(integer, T1)} whose structure is \texttt{(integer, integer)}.

    \subsection{Code}

    \subsection{Formally}

    \subsection{Comments}
    The structure of a type is the primitive type it is equivalent to such that
    it can hold the same values. 

    This aims to encompass LRM Section 7.1.2 D\_FXQV.

\section{Domain of Values for Types}

\subsection{Prose}
  The domain of a type is the set of values which storage elements of that type
may hold. 

\subsection{Examples}
  The domain of \texttt{integer} is the infinite set of all integers.

  The domain of \texttt{bits(1)} is the set \texttt{{‘1’, ‘0’}}.

  The domain of \texttt{integer {2,16}} is the set containing the integers \texttt{2} and \texttt{16}.

  The domain of \texttt{bits({2,16})} is the set containing all two bit and all sixteen bit values.

  \subsection{Code}

  \subsection{Formally}

  \subsection{Comments}
  This aims to encompass LRM Section 7.1.3 D\_BMGM.

\section{Constrained Types}

\subsection{Prose}
  A constrained type is a type whose value is limited to a finite
  set. 

  A type which is not constrained is unconstrained.

  A constrained type with a non-empty constraint is well-constrained.

\subsection{Examples}
  Bitvector storage element’s widths are constrained integers.  

  \subsection{Code}

  \subsection{Formally}

\subsection{Comments}
  This aims to encompass LRM Section 3.4.1 D\_ZTPP, R\_WJYH, R\_HJPN, R\_CZTX, R\_TPHR

\chapter{Type satisfaction and related notions}

\section{TypingRule.Subtype} 

\subsection{Prose}
  The subtype relation is a partial order.

\subsection{Examples}

  \subsection{Code}

  \subsection{Formally}

\subsection{Comments}
  Since the subtype relation is a partial order, it is reflexive, viz, 
  every type is also a subtype of itself.

  Since the subtype relation is a partial order, it is transitive, viz, if A is
  a subtype of B and B is a subtype of C then A is a subtype of C.
 
  As a consequence, there is no need to declare the reflexive and transitive
  subtype relations explicitly. All other subtype relations must be explicitly
  declared.
 
  Since the subtype relation is a partial order, it is antisymmetric. Therefore
  it is an error if all of the following apply:
  \begin{itemize}
  \item \texttt{id1} is a subtype of \texttt{id2};
  \item \texttt{id2} is a subtype of \texttt{id1}.
  \end{itemize}

  This aims to encompass LRM Section 4.3.1 R\_NXRX, I\_KGKS, I\_MTML, I\_JVRM, I\_CHMP.

\section{TypingRule.Supertype}

  \subsection{Prose}
  \texttt{T} is a supertype of \texttt{S} if and only if \texttt{S} is a subtype of \texttt{T}.

  \subsection{Examples}

  \subsection{Code}

  \subsection{Formally}

  \subsection{Comments}
  Since the subtype relation is a partial order, it is reflexive. Therefore the
  supertype relation also is reflexive, viz, every type is also a supertype of
  itself.

  This aims to encompass LRM Section 4.3.1 I\_KGKS.

\section{TypingRule.StructuralSubtypeSatisfaction}

\subsection{Prose}
  \texttt{T} structural-subtype-satisfies \texttt{S} if one of the following applies:
  \begin{itemize}
  \item All of the following apply:
    \begin{itemize}
    \item \texttt{S} has the structure of an integer type;
    \item \texttt{T} has the structure of an integer type.
    \end{itemize}

  \item All of the following apply:
    \begin{itemize}
    \item \texttt{S} has the structure of a real type;
    \item \texttt{T} has the structure of a real type.
    \end{itemize} 

  \item All of the following apply:
    \begin{itemize}
    \item \texttt{S} has the structure of a string type;
    \item \texttt{T} has the structure of a string type.
    \end{itemize}

  \item All of the following apply:
    \begin{itemize}
    \item \texttt{S} has the structure of a boolean type;
    \item \texttt{T} has the structure of a boolean type.
    \end{itemize}

  \item All of the following apply:
    \begin{itemize}
    \item \texttt{S} has the structure of an enumeration type; 
    \item \texttt{T} has the structure of an enumeration type;
    \item \texttt{S} and \texttt{T} have the same enumeration literals.
    \end{itemize}

  \item All of the following apply:
    \begin{itemize}
    \item \texttt{S} has the structure of a bitvector type with determined width \texttt{w};
    \item One of the following applies:
      \begin{itemize}
      \item \texttt{T} has the structure of a bitvector type of determined width \texttt{w};
      \item \texttt{T} has the structure of a bitvector type of undetermined width.
      \end{itemize}
    \end{itemize}

  \item All of the following apply:
    \begin{itemize}
    \item \texttt{S} has the structure of a bitvector type with undetermined width;
    \item \texttt{T} has the structure of a bitvector type. 
    \end{itemize}

  \item All of the following apply:
    \begin{itemize}
    \item \texttt{S} has the structure of a bitvector type with bitfields \texttt{bitfields} and width \texttt{width};
    \item \texttt{T} has the structure of a bitvector type with width \texttt{width};
    \item for every bitfield \texttt{f} in \texttt{bitfields} there is a bitfield \texttt{f'} in \texttt{T} and
      all of the following apply:
      \begin{itemize}
      \item \texttt{f'} has the same name, width and offset as \texttt{f};
      \item \texttt{f'} type-satisfies \texttt{f}.
      \end{itemize}
    \end{itemize}

  \item All of the following apply:
    \begin{itemize}
    \item \texttt{S} has the structure of an array type with elements of type \texttt{E};
    \item \texttt{T} has the structure of an array type with elements of type \texttt{E};
    \item \texttt{T} has the same element indices as \texttt{S}.
    \end{itemize}

  \item All of the following apply:
    \begin{itemize}
    \item \texttt{S} has the structure of a tuple type;
    \item \texttt{T} has the structure of a tuple type;
    \item \texttt{T} has the same number of elements as \texttt{S};
    \item for each element \texttt{e} in \texttt{S} there is an element \texttt{e'} in \texttt{T} and \texttt{e'}
      type-satisfies \texttt{e}.
    \end{itemize}  

  \item All of the following apply:
    \begin{itemize}
    \item \texttt{S} has the structure of a record type;
    \item \texttt{T} has the structure of a record type;
    \item for each field \texttt{f} in \texttt{S} there is an element \texttt{f'} in \texttt{T} and \texttt{f'} has
      the same type as \texttt{f}.
    \end{itemize}

  \item All of the following apply:
    \begin{itemize}
    \item \texttt{S} has the structure of an exception type;
    \item \texttt{T} has the structure of an exception type;
    \item for each field \texttt{f} in \texttt{S} there is an element \texttt{f'} in \texttt{T} and \texttt{f'} has
      the same type as \texttt{f}.
    \end{itemize}
  \end{itemize}

  \subsection{Examples}

  \subsection{Code}
    \VerbatimInput[firstline=\StructuralSubtypeSatisfactionBegin, lastline=\StructuralSubtypeSatisfactionEnd]{../types.ml}

  \subsection{Formally}

\subsection{Comments}
  This aims to encompass LRM Section 7.3.1. D\_TRVR.

\section{TypingRule.DomainSubtypeSatisfaction}

\subsection{Prose}
 \texttt{T} domain-subtype-satisfies \texttt{S} if one of the following applies:
 \begin{itemize}
 \item All of the following apply:
    \begin{itemize}
    \item \texttt{S} does not have the structure of an aggregate type or bitvector type;
    \item the domain of \texttt{T} is a subset of the domain of \texttt{S}.
    \end{itemize}

  \item All of the following apply:
    \item One of the following applies:
      \begin{itemize}
      \item \texttt{S} has the structure of a bitvector type with undetermined width;
      \item \texttt{T} has the structure of a bitvector type with undetermined width;
      \end{itemize}
   \item the domain of \texttt{T} is a subset of the domain of \texttt{S}.
  \end{itemize}

  \subsection{Examples}

  \subsection{Code}
    \VerbatimInput[firstline=\DomainSubtypeSatisfactionBegin, lastline=\DomainSubtypeSatisfactionEnd]{../types.ml}

  \subsection{Formally}

  \subsection{Comments}
    This aims to encompass LRM Section 7.3.1. D\_TRVR.

\section{TypingRule.SubtypeSatisfaction}

  \subsection{Prose}
    \texttt{T} subtype-satisfies \texttt{S} if all of the following apply:
    \begin{itemize}
    \item \texttt{T} structural-subtype-satisfies \texttt{S};
    \item \texttt{T} domain-subtype-satisfies \texttt{S}.
    \end{itemize} 

  \subsection{Examples}

  \subsection{Code}
    \VerbatimInput[firstline=\SubtypeSatisfactionBegin, lastline=\SubtypeSatisfactionEnd]{../types.ml}

  \subsection{Formally}

  \subsection{Comments}
    This aims to encompass LRM Section 7.3.1. D\_TRVR.

\section{TypingRule.TypeSatisfaction \label{sec:TypingRule.TypeSatisfaction}}

\subsection{Prose}
 
\texttt{T} type-satisfies \texttt{S} if one of the following applies:
 \begin{itemize}
 \item \texttt{T} is a subtype of \texttt{S};
 \item All of the following apply:
    \begin{itemize}
    \item \texttt{T} subtype-satisfies \texttt{S};
    \item One of the following applies:
      \begin{itemize}
      \item \texttt{S} is an anonymous type;
      \item \texttt{T} is an anonymous type;
      \end{itemize}
    \end{itemize}
  \item All of the following apply:
    \begin{itemize}
    \item \texttt{T} is an anonymous bitvector with no bitfields;
    \item \texttt{S} has the structure of a bitvector (with or without bitfields);
    \item \texttt{S} has the same width as \texttt{T}.
    \end{itemize}
  \end{itemize}

\subsection{Example: TypingRule.TypeSatisfaction1.asl}
    In the program:
    \VerbatimInput{../tests/ASLTypingReference.t/TypingRule.TypeSatisfaction1.asl}
    \texttt{var pair: pairT = (1, dataT1)} is legal since the right-hand-side has
    anonymous, non-primitive type \texttt{(integer, T1)}.

\subsection{Example: TypingRule.TypeSatisfaction2.asl}
    In the program:
    \VerbatimInput{../tests/ASLTypingReference.t/TypingRule.TypeSatisfaction2.asl}
    \texttt{pair = (1, dataAsInt);} is legal since the right-hand-side has anonymous,
    primitive type \texttt{(integer, integer)}.
 
\subsection{Example: TypingRule.TypeSatisfaction3.asl}
    In the program:
    \VerbatimInput{../tests/ASLTypingReference.t/TypingRule.TypeSatisfaction3.asl}
    \texttt{pair = (1, dataT2);} is illegal since the right-hand-side has anonymous,
    non-primitive type \texttt{(integer, T2)} which does not subtype-satisfy named
    type \texttt{pairT}.

  \subsection{Code}
    \VerbatimInput[firstline=\TypeSatisfactionBegin, lastline=\TypeSatisfactionEnd]{../types.ml}

  \subsection{Formally}

  \subsection{Comments}
  Since the subtype relation is a partial order, it is reflexive. Therefore
  every type \texttt{T} is a subtype of itself, and as a consequence, every type \texttt{T}
  type-satisfies itself.
  
  This aims to encompass LRM Section 7.3.2 R\_FMXK and I\_NLFD.

\section{TypingRule.CanAssignTo \label{sec:TypingRule.CanAssignTo}}

  \subsubsection{Prose}
  \texttt{S} can be assigned to \texttt{T} if and only if all of the following apply:
  \begin{itemize}
  \item neither \texttt{S} nor \texttt{T} has the structure of the underconstrained integer type;
  \item \texttt{T} type-satisfies \texttt{S}.
  \end{itemize} 
 
  \subsection{Examples}

  \subsection{Code}
    \VerbatimInput[firstline=\CanAssignToBegin, lastline=\CanAssignToEnd]{../Typing.ml}

  \subsection{Formally}
  
  \subsection{Comments}
  This aims to encompass LRM R\_GNTS and R\_LXQZ.

\section{TypingRule.TypeClash}

  \subsection{Prose}
  \texttt{T} type-clashes with \texttt{S} if one of the following applies:
  \begin{itemize}
  \item \texttt{S} and \texttt{T} both have the structure of integers;
  \item \texttt{S} and \texttt{T} both have the structure of reals;
  \item \texttt{S} and \texttt{T} both have the structure of strings;
  \item \texttt{S} and \texttt{T} both have the structure of enumeration types with the same enumeration literals;
  \item \texttt{S} and \texttt{T} both have the structure of bitvectors;
  \item \texttt{S} and \texttt{T} both have the structure of arrays whose element types type-clash;
  \item \texttt{S} and \texttt{T} both have the structure of tuples of the same length whose
    corresponding element types type-clash;
  \item \texttt{S} is a subtype of \texttt{T};
  \item \texttt{S} is a supertype of \texttt{T}.
  \end{itemize}

  \subsection{Examples}

  \subsection{Code}
    \VerbatimInput[firstline=\TypeClashBegin, lastline=\TypeClashEnd]{../types.ml}

  \subsection{Formally}

  \subsection{Comments}
  Note that if \texttt{T} subtype-satisfies \texttt{S} then \texttt{T} and \texttt{S} type-clash, but not the other
  way around.

  Note that type-clashing is an equivalence relation. Therefore if \texttt{T}
  type-clashes with \texttt{A} and \texttt{B} then it is also the case that \texttt{A} and \texttt{B} type-clash.

  This aims to encompass LRM Section 7.3.3. D\_VPZZ, I\_PQCT and I\_WZKM.

\section{TypingRule.LowestCommonAncestor \label{sec:TypingRule.LowestCommonAncestor}}

\subsection{Prose}
  The lowest common ancestor of types \texttt{S} and \texttt{T} is \texttt{ty} and one of the following applies:
  \begin{itemize}
  \item All of the following apply:
    \begin{itemize}
    \item \texttt{S} and \texttt{T} are the same type;
    \item \texttt{ty} is \texttt{S}.
    \end{itemize}

  \item All of the following apply:
    \begin{itemize}
    \item \texttt{S} and \texttt{T} are both named types;
    \item \texttt{ty} is a common supertype of \texttt{S} and \texttt{T};
    \item \texttt{ty} is a subtype of all other common supertypes of \texttt{S} and \texttt{T}.
    \end{itemize}

  \item All of the following apply:
    \begin{itemize}
    \item \texttt{S} and \texttt{T} both have the structure of array types with the same index type
      and the same element types;

    \item One of the following applies:
      \begin{itemize}
      \item All of the following apply:
        \begin{itemize}
        \item \texttt{S} is a named type;
        \item \texttt{T} is an anonymous type;
        \item \texttt{ty} is \texttt{S}.
        \end{itemize}

      \item All of the following apply:
        \begin{itemize}
        \item \texttt{S} is an anonymous type;
        \item \texttt{T} is a named type;
        \item \texttt{ty} is \texttt{T}.
        \end{itemize}
      \end{itemize}
    \end{itemize}

  \item All of the following apply:
    \begin{itemize}
    \item \texttt{S} and \texttt{T} both have the structure of tuple types with the same number of elements;
    \item The types of the elements of \texttt{S} type-satisfy the types of the elements of \texttt{T};
    \item The types of the elements of \texttt{T} type-satisfy the types of the elements of \texttt{S};
    \item One of the following applies:

      \item All of the following apply:
        \begin{itemize}
        \item \texttt{S} is a named type;
        \item \texttt{T} is an anonymous type;
        \item \texttt{ty} is \texttt{S}.
        \end{itemize}

      \item All of the following apply:
        \begin{itemize}
        \item \texttt{S} is an anonymous type;
        \item \texttt{T} is a named type;
        \item \texttt{ty} is \texttt{T}.
        \end{itemize}

     \item All of the following apply:
        \begin{itemize}
        \item \texttt{S} is an anonymous type;
        \item \texttt{T} is an anonymous type;
	\item \texttt{ty} is the tuple type where the type of each element is the lowest common
	  ancestor of the types of the corresponding elements of \texttt{S} and \texttt{T}. 
        \end{itemize}
    \end{itemize}

  \item All of the following apply:
    \begin{itemize}
    \item \texttt{S} and \texttt{T} both have the structure of well-constrained integer types;
    \item One of the following applies:
      \begin{itemize}
      \item All of the following apply:
        \begin{itemize}
        \item \texttt{S} is a named type;
        \item \texttt{T} is an anonymous type;
        \item \texttt{ty} is \texttt{S}.
        \end{itemize}

      \item All of the following apply:
        \begin{itemize}
        \item \texttt{S} is an anonymous type;
        \item \texttt{T} is a named type;
        \item \texttt{ty} is \texttt{T}.
        \end{itemize}

      \item All of the following apply:
        \begin{itemize}
        \item \texttt{S} is an anonymous type;
        \item \texttt{T} is an anonymous type;
	\item \texttt{ty} is the well-constrained integer type whose domain is the union of the
	  domains of \texttt{S} and \texttt{T}.      
        \end{itemize}
      \end{itemize}
    \end{itemize}

  \item All of the following apply:
    \begin{itemize}
    \item Either \texttt{S} or \texttt{T} have the structure of an unconstrained integer type;
    \item One of the following applies:

      \item All of the following apply:
        \begin{itemize}
        \item \texttt{S} is a named type;
        \item \texttt{S} has the structure of an unconstrained integer type;
        \item \texttt{T} is an anonymous type;
        \item \texttt{ty} is \texttt{S}.
        \end{itemize}

      \item All of the following apply:
        \begin{itemize}
        \item S is an anonymous type;
        \item T is a named type;
        \item T has the structure of an unconstrained integer type;
        \item ty is T.
        \end{itemize}

      \item All of the following apply:
        \begin{itemize}
        \item \texttt{S} is an anonymous type;
        \item \texttt{T} is an anonymous type;
	\item \texttt{ty} is the unconstrained integer type. 
        \end{itemize}
    \end{itemize}

  \item All of the following apply:
    \begin{itemize}
    \item Either \texttt{S} or \texttt{T} have the structure of an under-constrained integer type;
    \item \texttt{ty} is the under-constrained integer type. 
    \end{itemize}

  \item \texttt{ty} is undefined.
  \end{itemize}

  \subsection{Examples}

  \subsection{Code}
    \VerbatimInput[firstline=\LowestCommonAncestorBegin, lastline=\LowestCommonAncestorEnd]{../types.ml}

  \subsection{Formally}

  \subsection{Comments}
    This aims to encompass LRM Section 7.6.1. R\_YZHM.

\section{TypingRule.CheckUnop \label{sec:TypingRule.CheckUnop}}

\subsection{Goal}
  Checking compatibility of an unary operator with the type of its argument.

\subsection{Prose}
  \texttt{t} is the result of checking compatibility of a unary operator \texttt{op} with
  type \texttt{t1} and one of the following applies:
  \begin{itemize}
  \item All of the following apply:
    \begin{itemize}
    \item \texttt{op} is \texttt{BNOT};
    \item \texttt{t1} type-satisfies \texttt{boolean};
    \item \texttt{t} is \texttt{boolean};
    \end{itemize}

  \item All of the following apply:
    \begin{itemize}
    \item \texttt{op} is \texttt{NEG};
    \item One of the following applies:
      \begin{itemize}
      \item \texttt{t1} type-satisfies \texttt{integer}; 
      \item \texttt{t1} type-satisfies \texttt{real};
      \end{itemize}
     \item One of the following applies:
       \begin{itemize}
       \item All of the following apply:
         \begin{itemize}
         \item \texttt{t1} has the structure of an unconstrained integer;
         \item \texttt{t} is an unconstrained integer;
         \end{itemize}
       \item All of the following apply:
         \begin{itemize}
         \item \texttt{t1} has the structure of a constrained integer;
         \item \texttt{t} is a constrained integer whose constraint is ;
         \end{itemize}
       \end{itemize}
    \end{itemize}  

  \item All of the following apply:
    \begin{itemize}
    \item \texttt{op} is \texttt{NOT};
    \item \texttt{t1} has the structure of a bitvector;
    \item \texttt{t} is \texttt{t1}.
    \end{itemize}
  \end{itemize}

  \subsection{Examples}

  \subsection{Code}
    \VerbatimInput[firstline=\CheckUnopBegin, lastline=\CheckUnopEnd]{../Typing.ml}

  \subsection{Formally}

  \subsection{Comments}

\section{TypingRule.CheckBinop \label{sec:TypingRule.CheckBinop}}

\subsection{Goal}
  Checking compatibility of a binary operator with the types of its arguments.

\subsection{Prose}
  \texttt{t} is the result of checking compatibility of a binary operator \texttt{op} with
  types \texttt{t1} and \texttt{t2} and one of the following applies:
\begin{itemize}
  \item All of the following apply:
    \begin{itemize}
    \item \texttt{op} is \texttt{AND}, \texttt{OR}, \texttt{EQ} or \texttt{IMPL}; 
    \item \texttt{t1} type-satisfies \texttt{boolean};
    \item \texttt{t2} type-satisfies \texttt{boolean};
    \item \texttt{t} is \texttt{boolean}.
    \end{itemize}

  \item All of the following apply:
    \begin{itemize}
    \item \texttt{op} is \texttt{AND}, \texttt{OR}, or \texttt{EOR};
    \item \texttt{t1} has the structure of a bitvector;
    \item \texttt{t2} has the structure of a bitvector;
    \item \texttt{t1} and \texttt{t2} have the same bitvector width \texttt{w};
    \item \texttt{t} is the bitvector type of width \texttt{w}.
    \end{itemize}

  \item All of the following apply:
    \begin{itemize}
    \item \texttt{op} is \texttt{PLUS} or \texttt{MINUS};
    \item \texttt{t1} has the structure of a bitvector;
    \item \texttt{t2} has the structure of a bitvector;
    \item \texttt{t1} and \texttt{t2} have the same bitvector width \texttt{w};
    \item \texttt{t2} type-satisfies \texttt{integer};
    \item \texttt{t} is the bitvector type of width \texttt{w}.
    \end{itemize}

  \item All of the following apply:
    \begin{itemize}
    \item \texttt{op} is \texttt{EQ\_OP} or \texttt{NEQ};
    \item One of the following applies:
      \begin{itemize}
      \item \texttt{t1} is equal to \texttt{t2};
      \item All of the following apply:
        \begin{itemize}
        \item \texttt{t1} type-satisfies \texttt{integer};
        \item \texttt{t2} type-satisfies \texttt{integer}; 
        \end{itemize}
      \item All of the following apply:
        \begin{itemize}
        \item \texttt{t1} has the structure of a bitvector;
        \item \texttt{t2} has the structure of a bitvector;
        \item \texttt{t1} and \texttt{t2} have the same bitvector width;
        \end{itemize}
      \item All of the following apply:
        \begin{itemize}
        \item \texttt{t1} type-satisfies \texttt{boolean};
        \item \texttt{t2} type-satisfies \texttt{boolean};
        \end{itemize}
      \item All of the following apply:
        \begin{itemize}
        \item \texttt{t1} enumerates local declarations \texttt{li1}; 
        \item \texttt{t2} enumerates local declarations \texttt{li2};
        \item \texttt{li1} equals \texttt{li2};
        \end{itemize}
      \end{itemize}
    \item \texttt{t} is \texttt{boolean}.
    \end{itemize}
   
  \item All of the following apply:
    \begin{itemize}
    \item \texttt{op} is \texttt{LEQ}, \texttt{GEQ}, \texttt{GT} or \texttt{LT};
    \item One of the following applies:
      \begin{itemize}
      \item All of the following apply:
        \begin{itemize}
        \item \texttt{t1} type-satisfies \texttt{integer};
        \item \texttt{t2} type-satisfies \texttt{integer};
        \end{itemize}
      \item All of the following apply:
        \begin{itemize}
        \item \texttt{t1} type-satisfies \texttt{real};
        \item \texttt{t2} type-satisfies \texttt{real};
        \end{itemize}
      \end{itemize}
    \item \texttt{t} is boolean.
    \end{itemize}

  \item All of the following apply:
    \begin{itemize}
    \item \texttt{op} is \texttt{MUL}, \texttt{DIV}, \texttt{DIVRM}, \texttt{MOD}, \texttt{SHL}, \texttt{SHR}, \texttt{POW}, \texttt{PLUS} or \texttt{MINUS};
    \item \texttt{struct1} is the structure of \texttt{t1};
    \item \texttt{struct2} is the structure of \texttt{t2};
    \item One of the following applies:
      \begin{itemize}
      \item All of the following apply:
        \begin{itemize}
        \item \texttt{t1} has the structure of an unconstrained integer;
        \item \texttt{t2} has the structure of an integer;
        \item \texttt{t} is an unconstrained integer;
        \end{itemize}
      \item All of the following apply:
        \begin{itemize}
        \item \texttt{t1} has the structure of an integer;
        \item \texttt{t2} has the structure of an unconstrained integer;
        \item \texttt{t} is an unconstrained integer;
        \end{itemize}
      \item One of the following applies:
       \begin{itemize} 
       \item All of the following apply:
          \begin{itemize}
          \item \texttt{t1} has the structure of an under-constrained integer;
          \item \texttt{t2} has the structure of a constrained integer;
          \item \texttt{t} is an under-constrained integer;
          \end{itemize}
        \item All of the following apply:
          \begin{itemize}
          \item \texttt{t1} has the structure of a constrained integer;
          \item \texttt{t2} has the structure of an under-constrained integer;
          \item \texttt{t} is an under-constrained integer;
          \end{itemize}
       \end{itemize}
      \item One of the following applies:
         \begin{itemize}
         \item All of the following apply:
           \begin{itemize}
           \item \texttt{t1} has the structure of a well-constrained integer;
           \item \texttt{t2} has the structure of a well-constrained integer;
	   \item \texttt{t} is a constrained integer whose constraint is calculated by
	     applying the operation to all possible value pairs;
           \end{itemize}
         \item All of the following apply:
           \begin{itemize}
           \item \texttt{t1} has the structure of a well-constrained integer;
           \item \texttt{t2} has the structure of an well-constrained integer;
	   \item \texttt{t} is a constrained integer whose constraint is calculated by
	     applying the operation to all possible value pairs;
           \end{itemize}
         \end{itemize}
      \item All of the following apply:
        \begin{itemize}
        \item \texttt{t1} has the structure of \texttt{real};
        \item \texttt{t2} has the structure of \texttt{real};
        \item \texttt{op} is \texttt{PLUS}, \texttt{MINUS} or \texttt{MUL};
        \item \texttt{t} is \texttt{real};
        \end{itemize}
     \item All of the following apply:
       \begin{itemize}
       \item \texttt{t1} has the structure of \texttt{real};
       \item \texttt{t2} has the structure of \texttt{integer};
       \item \texttt{op} is \texttt{POW};
       \item \texttt{t} is \texttt{real};
       \end{itemize}
     \end{itemize}
   \end{itemize} 

  \item All of the following apply:
    \begin{itemize}
    \item \texttt{op} is \texttt{RDIV};
    \item \texttt{t1} type-satisfies \texttt{real};
    \item \texttt{t} is \texttt{real}.
    \end{itemize}
\end{itemize}

  \subsection{Examples} 

  \subsection{Code}
    \VerbatimInput[firstline=\CheckUnopBegin, lastline=\CheckUnopEnd]{../Typing.ml}

  \subsection{Formally}

\subsection{Comments}
  This aims to encompass LRM Section 7.5.3 R\_BKNT, Section 7.5.5 R\_ZYWY, R\_BZKW,
  R\_KFYS, Section 7.5.6 R\_KXMR, Section 7.7 R\_SQXN, R\_MRHT.

\chapter{Typing of Expressions}

\texttt{annotate\_expr} specifies how to annotate an expression \texttt{e} in
an environment \texttt{env}.  Formally, the result of annotating the expression
\texttt{e} in \texttt{env} is \texttt{t,new\_env} where \texttt{t} is a type and
\texttt{new\_env} is an environment, and one of the following applies:
\begin{itemize}
\item TypingRule.Lit (see Section~\ref{sec:TypingRule.Lit});
\item TypingRule.ELocalVarConstant (see Section~\ref{sec:TypingRule.ELocalVarConstant})
\item TypingRule.ELocalVar (see Section~\ref{sec:TypingRule.ELocalVar})
\item TypingRule.EGlobalVarConstant (see Section~\ref{sec:TypingRule.EGlobalVarConstant})
\item TypingRule.EGlobalVar (see Section~\ref{sec:TypingRule.EGlobalVar})
\item TypingRule.EUndefIdent (see Section~\ref{sec:TypingRule.EUndefIdent})
\item TypingRule.Binop (see Section~\ref{sec:TypingRule.Binop})
\item TypingRule.Unop (see Section~\ref{sec:TypingRule.Unop})
\item TypingRule.ECond (see Section~\ref{sec:TypingRule.ECond})
\item TypingRule.ESlice (see Section~\ref{sec:TypingRule.ESlice})
\item TypingRule.ECall (see Section~\ref{sec:TypingRule.ECall})
\item TypingRule.EGetArray (see Section~\ref{sec:TypingRule.EGetArray})
\item TypingRule.EStructuredNotStructured (see Section~\ref{sec:TypingRule.EStructuredNotStructured})
\item TypingRule.EStructuredMissingField (see Section~\ref{sec:TypingRule.EStructuredMissingField})
\item TypingRule.ERecord (see Section~\ref{sec:TypingRule.ERecord})
\item TypingRule.EGetRecordField (see Section~\ref{sec:TypingRule.EGetRecordField})
\item TypingRule.EGetBadRecordField (see Section~\ref{sec:TypingRule.EGetBadRecordField})
\item TypingRule.EGetBadBitField (see Section~\ref{sec:TypingRule.EGetBadBitField})
\item TypingRule.EGetBadField (see Section~\ref{sec:TypingRule.EGetBadField})
\item TypingRule.EGetBitField (see Section~\ref{sec:TypingRule.EGetBitField})
\item TypingRule.EGetBitFieldNested (see Section~\ref{sec:TypingRule.EGetBitFieldNested})
\item TypingRule.EGetBitFieldTyped (see Section~\ref{sec:TypingRule.EGetBitFieldTyped})
\item TypingRule.EConcatEmpty (see Section~\ref{sec:TypingRule.EConcatEmpty})
\item TypingRule.EConcat (see Section~\ref{sec:TypingRule.EConcat})
\item TypingRule.ETuple (see Section~\ref{sec:TypingRule.ETuple})
\item TypingRule.EUnknown (see Section~\ref{sec:TypingRule.EUnknown})
\item TypingRule.EPattern (see Section~\ref{sec:TypingRule.EPattern})
\item TypingRule.CTC (see Section~\ref{sec:TypingRule.CTC})
\end{itemize}

\section{TypingRule.Lit \label{sec:TypingRule.Lit}}

  \subsection{Prose}
  The result of annotating the expression \texttt{e} in \texttt{env} is
\texttt{t,new\_env} and all of the following apply:
  \begin{itemize}
  \item \texttt{e} is a Literal \texttt{v};
  \item \texttt{t} is the type of \texttt{v};
  \item \texttt{new\_env} is \texttt{e}.
  \end{itemize}

  \subsection{Examples}

  \subsection{Code}
  \VerbatimInput[firstline=\LitBegin, lastline=\LitEnd]{../Typing.ml}  
 
  \subsection{Formally}

  \subsection{Comments}

\section{TypingRule.ELocalVarConstant \label{sec:TypingRule.ELocalVarConstant}}

  \subsection{Prose}
  The result of annotating the expression \texttt{e} in \texttt{env} is
\texttt{t,new\_env} and all of the following apply:  
  \begin{itemize}
  \item \texttt{e} denotes a variable \texttt{x};
  \item \texttt{x} is bound to a local constant~\texttt{v} of type \texttt{ty} in the local environment given by \texttt{env};
  \item \texttt{t} is \texttt{ty};
  \item \texttt{new\_env} is the Literal \texttt{v}.
  \end{itemize}

  \subsection{Examples}

  \subsection{Code}
  \VerbatimInput[firstline=\ELocalVarConstantBegin, lastline=\ELocalVarConstantEnd]{../Typing.ml}  

  \subsection{Formally}

  \subsection{Comments}

\section{TypingRule.ELocalVar \label{sec:TypingRule.ELocalVar}}

  \subsection{Prose}
  The result of annotating the expression \texttt{e} in \texttt{env} is
\texttt{t,new\_env} and all of the following apply: 
  \begin{itemize}
  \item \texttt{e} denotes a variable \texttt{x};
  \item \texttt{x} is not bound to a local constant; 
  \item \texttt{x} has type \texttt{ty} in the local environment given by \texttt{env};
  \item \texttt{t} is \texttt{ty};
  \item \texttt{new\_env} is \texttt{e}.
  \end{itemize}

  \subsection{Examples}

  \subsection{Code}
  \VerbatimInput[firstline=\ELocalVarBegin, lastline=\ELocalVarEnd]{../Typing.ml}  

  \subsection{Formally}

  \subsection{Comments}

\section{TypingRule.EGlobalVarConstantVal \label{sec:TypingRule.EGlobalVarConstant}}

  \subsection{Prose}
  The result of annotating the expression \texttt{e} in \texttt{env} is
\texttt{t,new\_env} and all of the following apply: 
  \begin{itemize}
  \item \texttt{e} denotes a variable \texttt{x};
  \item \texttt{x} is bound to a global constant~\texttt{v} of type \texttt{ty} in the global environment given by \texttt{env};
  \item \texttt{t} is \texttt{ty};
  \item \texttt{new\_env} is the Literal \texttt{v}.
  \end{itemize}

  \subsection{Examples}

  \subsection{Code}
  \VerbatimInput[firstline=\EGlobalVarConstantBegin, lastline=\EGlobalVarConstantEnd]{../Typing.ml}

  \subsection{Formally}

  \subsection{Comments}

\section{TypingRule.EGlobalVar \label{sec:TypingRule.EGlobalVar}}

  \subsection{Prose}
  The result of annotating the expression \texttt{e} in \texttt{env} is
\texttt{t,new\_env} and all of the following apply:
  \begin{itemize}
  \item \texttt{e} denotes a variable \texttt{x};
  \item \texttt{x} is not bound to a global constant;
  \item \texttt{x} has type \texttt{ty} in the global environment given by \texttt{env};
  \item \texttt{t} is \texttt{ty};
  \item \texttt{new\_env} is \texttt{e}.
  \end{itemize}

  \subsection{Examples}

  \subsection{Code}
  \VerbatimInput[firstline=\EGlobalVarBegin, lastline=\EGlobalVarEnd]{../Typing.ml}

  \subsection{Formally}

  \subsection{Comments}

\section{TypingRule.EUndefIdent \label{sec:TypingRule.EUndefIdent}}

  \subsection{Prose}
  The result of annotating the expression \texttt{e} in \texttt{env} is
\texttt{t,new\_env} and all of the following apply:
  \begin{itemize}
  \item \texttt{e} is a variable \texttt{x};
  \item \texttt{x} is not bound in \texttt{env};
  \item an error "Undefined Identifier" is raised.
  \end{itemize}

  \subsection{Examples}

  \subsection{Code}

  \subsection{Formally}

  \subsection{Comments}

\section{TypingRule.Binop \label{sec:TypingRule.Binop}}

  \subsection{Prose}
  The result of annotating the expression \texttt{e} in \texttt{env} is
\texttt{t,new\_env} and all of the following apply:
  \begin{itemize}
  \item \texttt{e} denotes a binary operation \texttt{op} over two expressions \texttt{e1} and \texttt{e2};
  \item \texttt{t1,e1'} is the result of annotating \texttt{e1} in \texttt{env};
  \item \texttt{t2,e2'} is the result of annotating \texttt{e2} in \texttt{env};
  \item \texttt{t} is the result of checking compatibility of \texttt{op} with \texttt{t1} and \texttt{t2} as per Section~\ref{sec:TypingRule.CheckBinop};
  \item \texttt{new\_env} denotes \texttt{op} over \texttt{e1'} and \texttt{e2'}.
  \end{itemize}

  \subsection{Examples}

  \subsection{Code}
    \VerbatimInput[firstline=\BinopBegin, lastline=\BinopEnd]{../Typing.ml}

  \subsection{Formally}

  \subsection{Comments}

\section{TypingRule.Unop \label{sec:TypingRule.Unop}}

  \subsection{Prose}
  The result of annotating the expression \texttt{e} in \texttt{env} is
\texttt{t,new\_env} and all of the following apply:
  \begin{itemize}
  \item \texttt{e} denotes a unary operation \texttt{op} over an expression \texttt{e'};
  \item \texttt{t'',e''} is the result of annotating \texttt{e'} in \texttt{env};
  \item \texttt{t} is the result of checking compatibility of \texttt{op} with \texttt{t''} as per Section~\ref{sec:TypingRule.CheckUnop};
  \item \texttt{new\_env} denotes \texttt{op} over \texttt{e''}.
  \end{itemize}

  \subsection{Examples}

  \subsection{Code}
    \VerbatimInput[firstline=\UnopBegin, lastline=\UnopEnd]{../Typing.ml}

  \subsection{Formally}

  \subsection{Comments}

\section{TypingRule.ECond \label{sec:TypingRule.ECond}}

  \subsection{Prose}
  The result of annotating the expression \texttt{e} in \texttt{env} is
\texttt{t,new\_env} and all of the following apply:
  \begin{itemize}
  \item \texttt{e} denotes a conditional expression with condition \texttt{e\_cond} with two options \texttt{e\_true} and \texttt{e\_false};
  \item \texttt{t\_cond, e'\_cond} is the result of annotating \texttt{e\_cond} in \texttt{env};
  \item \texttt{t\_true, e'\_true} is the result of annotating \texttt{e\_true} in \texttt{env};
  \item \texttt{t\_false, e'\_false} is the result of annotating \texttt{e\_false} in \texttt{env};
  \item One of the following applies:
    \begin{itemize}
    \item All of the following apply:
      \begin{itemize}
      \item \texttt{t} is the lowest common ancestor of \texttt{t\_true} and \texttt{t\_false};
      \item \texttt{new\_env} is the condition \texttt{e'\_cond} with two options \texttt{e'\_true} and \texttt{e'\_false}.
      \end{itemize}
    \item All of the following apply:
      \begin{itemize}
      \item there is no lowest common ancestor of \texttt{t\_true} and \texttt{t\_false};
      \item an error ``\texttt{Unreconciliable Types}'' is raised.
      \end{itemize}
    \end{itemize}
  \end{itemize}

  \subsection{Examples}

  \subsection{Code}
    \VerbatimInput[firstline=\ECondBegin, lastline=\ECondEnd]{../Typing.ml}

  \subsection{Formally}

  \subsection{Comments}
  This aims to encompass LRM Section 7.6 R\_XZVT.

\section{TypingRule.ESlice \label{sec:TypingRule.ESlice}}

  \subsection{Prose}
  The result of annotating the expression \texttt{e} in \texttt{env} is
\texttt{t,new\_env} and all of the following apply:
  \begin{itemize}
  \item \texttt{e} denotes the slicing of expression \texttt{e'} by the slices \texttt{slices};
  \item \texttt{t\_e',e'} is the result of annotating the expression \texttt{e'} in \texttt{env};
  \item \texttt{t\_e'} has the structure of an integer or a bitvector;
  \item \texttt{w} is the width of \texttt{slices};
  \item \texttt{slices'} is the result of annotating \texttt{slices} in \texttt{env};
  \item \texttt{t} is the bitvector type of width \texttt{w};
  \item \texttt{new\_env} is the slicing of expression \texttt{e'} by the slices \texttt{slices'}.
  \end{itemize}

  \subsection{Examples}

  \subsection{Code}
    \VerbatimInput[firstline=\ESliceBegin, lastline=\ESliceEnd]{../Typing.ml}

  \subsection{Formally}

  \subsection{Comments}
    The width of \texttt{slices} might be a symbolic expression if one of the
widths references a \texttt{let} identifier with a non-compile-time-constant initialiser
expression.

\section{TypingRule.ECall \label{sec:TypingRule.ECall}}

  \subsection{Prose}
  The result of annotating the expression \texttt{e} in \texttt{env} is
\texttt{t,new\_env} and all of the following apply:
  \begin{itemize}
  \item \texttt{e} denotes a call to a subprogram named \texttt{name} with arguments \texttt{args} and
    parameters \texttt{eqs};
  \item \texttt{name', args', eqs', ty} is the result of annotating the call of
    that subprogram in \texttt{env};
  \item \texttt{t} is \texttt{ty};
  \item \texttt{new\_env} is the call to the subprogram named \texttt{name'} with arguments \texttt{args'}
    and parameters \texttt{eqs'}.
  \end{itemize}

  \subsection{Examples}

  \subsection{Code}
    \VerbatimInput[firstline=\ECallBegin, lastline=\ECallEnd]{../Typing.ml}

  \subsection{Formally}

  \subsection{Comments}

\section{TypingRule.EGetArray \label{sec:TypingRule.EGetArray}}

  \subsection{Prose}
  The result of annotating the expression \texttt{e} in \texttt{env} is
\texttt{t,new\_env} and all of the following apply:
  \begin{itemize}
  \item \texttt{e} denotes the slicing of expression \texttt{e'} by the slices \texttt{slices};
  \item \texttt{t\_e',e'} is the result of annotating the expression \texttt{e'} in \texttt{env};
  \item \texttt{t\_e'} has the structure of an array with index type \texttt{wanted\_t\_index} and element type \texttt{t};
  \item \texttt{slices} is a single expression \texttt{e\_index};
  \item \texttt{t\_index', e\_index'} is the result of annotating \texttt{e\_index} in \texttt{env};
  \item \texttt{t\_index'} type-satisfies \texttt{wanted\_t\_index} as per Section~\ref{sec:TypingRule.TypeSatisfaction};
  \item \texttt{new\_e} is an access to array \texttt{e'} at index \texttt{e\_index'}.
  \end{itemize}

  \subsection{Examples}

  \subsection{Code}
    \VerbatimInput[firstline=\EGetArrayBegin, lastline=\EGetArrayEnd]{../Typing.ml}

  \subsection{Formally}

  \subsection{Comments}

\section{TypingRule.EStructuredNotStructured \label{sec:TypingRule.EStructuredNotStructured}}

  \subsection{Prose}
  The result of annotating the expression \texttt{e} in \texttt{env} is
\texttt{t,new\_env} and all of the following apply:
  \begin{itemize}
  \item \texttt{e} denotes the record expression or an exception expression of type \texttt{ty} with fields \texttt{fields};
  \item \texttt{ty} is neither a record nor an exception type;
  \item an error ``\texttt{Conflicting Types}'' is raised.
  \end{itemize}

  \subsection{Examples}

  \subsection{Code}
    \VerbatimInput[firstline=\EStructuredNotStructuredBegin, lastline=\EStructuredNotStructuredEnd]{../Typing.ml}

  \subsection{Formally}

  \subsection{Comments}

  This aims to encompass LRM Section 5.5 R\_WBCQ.

\section{TypingRule.EStructuredMissingField \label{sec:TypingRule.EStructuredMissingField}}

  \subsection{Prose}
  The result of annotating the expression \texttt{e} in \texttt{env} is
\texttt{t,new\_env} and all of the following apply:
  \begin{itemize}
  \item \texttt{e} denotes the record expression or an exception expression of type \texttt{ty} with fields \texttt{fields};
  \item \texttt{ty} is the name of a record or exception type with fields \texttt{field\_types};
  \item one field in \texttt{field\_types} is not initialised by \texttt{fields};
  \item an error ``\texttt{Missing Field}'' is raised.
  \end{itemize}

  \subsection{Examples}

  \subsection{Code}
    \VerbatimInput[firstline=\EStructuredMissingFieldBegin, lastline=\EStructuredMissingFieldEnd]{../Typing.ml}

  \subsection{Formally}

  \subsection{Comments}
  This aims to encompass LRM Section 5.5 R\_WBCQ.

\section{TypingRule.ERecord \label{sec:TypingRule.ERecord}}

  \subsection{Prose}
  The result of annotating the expression \texttt{e} in \texttt{env} is
\texttt{t,new\_env} and all of the following apply:
  \begin{itemize}
  \item \texttt{e} denotes the record expression of type \texttt{ty} with fields \texttt{fields};
  \item \texttt{ty} is the name of a record type with fields \texttt{field\_types};
  \item For each field named \texttt{name} associated with the expression \texttt{e'} in
    \texttt{field\_types}, all of the following apply:
    \begin{itemize}
    \item \texttt{t',e''} is the result of annotating \texttt{e'} in \texttt{env};
    \item \texttt{t\_spec'} is the type associated to \texttt{name} in \texttt{field\_types};
    \item \texttt{t'} type-satisfies \texttt{t\_spec'} as per Section~\ref{sec:TypingRule.TypeSatisfaction};
    \item \texttt{fields'} associates \texttt{name} to \texttt{e''};
    \end{itemize}
  \item \texttt{t} is \texttt{ty};
  \item \texttt{new\_env} is the record expression of type \texttt{ty} with fields \texttt{fields'}.
  \end{itemize}

  \subsection{Examples}

  \subsection{Code}
    \VerbatimInput[firstline=\ERecordBegin, lastline=\ERecordEnd]{../Typing.ml}

  \subsection{Formally}

  \subsection{Comments}
  This aims to encompass LRM Section 5.5 R\_WBCQ.

 
\section{TypingRule.EGetRecordField \label{sec:TypingRule.EGetRecordField}}

  \subsection{Prose}
  The result of annotating the expression \texttt{e} in \texttt{env} is
\texttt{t,new\_env} and all of the following apply:
  \begin{itemize}
  \item \texttt{e} denotes the access of field \texttt{field\_name} on expression \texttt{e1};
  \item \texttt{t\_e1, e2} is the result of annotating \texttt{e1} in \texttt{env};
  \item \texttt{t\_e1} has the structure of an exception or record type with fields \texttt{fields};
  \item \texttt{t\_e2} has the structure of an exception or record type with fields \texttt{fields};
  \item \texttt{field\_name} is declared in \texttt{fields};
  \item \texttt{t} is the type corresponding to \texttt{field\_name} in \texttt{fields};
  \item \texttt{new\_env} is the access of field \texttt{field\_name} on expression \texttt{e2}.
  \end{itemize}

  \subsection{Examples}

  \subsection{Code}
    \VerbatimInput[firstline=\EGetRecordFieldBegin, lastline=\EGetRecordFieldEnd]{../Typing.ml}

  \subsection{Formally}

  \subsection{Comments}

\section{TypingRule.EGetBadRecordField \label{sec:TypingRule.EGetBadRecordField}}

  \subsection{Prose}
  The result of annotating the expression \texttt{e} in \texttt{env} is
\texttt{t,new\_env} and all of the following apply:
  \begin{itemize}
  \item \texttt{e} denotes the access of field \texttt{field\_name} on expression \texttt{e1};
  \item \texttt{t\_e1, e2} is the result of annotating \texttt{e1} in \texttt{env};
  \item \texttt{t\_e1} has the structure of an exception or record type with fields \texttt{fields};
  \item \texttt{t\_e2} has the structure of an exception or record type with fields \texttt{fields};
  \item \texttt{t\_e2} is an Exception or a Record type with fields \texttt{fields};
  \item \texttt{field\_name} is not declared in \texttt{fields};
  \item an error ``\texttt{Bad Field}'' is raised.
  \end{itemize}

  \subsection{Examples}

  \subsection{Code}
    \VerbatimInput[firstline=\EGetBadRecordFieldBegin, lastline=\EGetBadRecordFieldEnd]{../Typing.ml}

  \subsection{Formally}

  \subsection{Comments}

\section{TypingRule.EGetBadBitField \label{sec:TypingRule.EGetBadBitField}}

  \subsection{Prose}
  The result of annotating the expression \texttt{e} in \texttt{env} is
\texttt{t,new\_env} and all of the following apply:
  \begin{itemize}
  \item \texttt{e} denotes the access of field \texttt{field\_name} on expression \texttt{e1};
  \item \texttt{t\_e1} has the structure a bitvector type with bitfields \texttt{bitfields};
  \item \texttt{t\_e2} has the structure a bitvector type with bitfields \texttt{bitfields};
  \item \texttt{field\_name} is not declared in \texttt{bitfields};
  \item an error ``\texttt{Bad Field}'' is raised.
  \end{itemize}

  \subsection{Examples}

  \subsection{Code}
    \VerbatimInput[firstline=\EGetBadBitFieldBegin, lastline=\EGetBadBitFieldEnd]{../Typing.ml}

  \subsection{Formally}

  \subsection{Comments}

\section{TypingRule.EGetBadField \label{sec:TypingRule.EGetBadField}}

 \subsection{Prose}
  The result of annotating the expression \texttt{e} in \texttt{env} is
\texttt{t,new\_env} and all of the following apply:
   \begin{itemize}
   \item \texttt{e} denotes the access of field \texttt{field\_name} on expression \texttt{e1};
   \item \texttt{t\_e1, e2} is the result of annotating \texttt{e1} in \texttt{env};
   \item \texttt{t\_e1} does not have the structure of a record or an exception or a bitvector type;
   \item an error ``\texttt{Conflicting Types}'' is raised.
   \end{itemize}

 \subsection{Examples}

 \subsection{Code}
    \VerbatimInput[firstline=\EGetBadFieldBegin, lastline=\EGetBadFieldEnd]{../Typing.ml}

 \subsection{Formally}

 \subsection{Comments}

\section{TypingRule.EGetBitField \label{sec:TypingRule.EGetBitField}}

  \subsection{Prose}
  The result of annotating the expression \texttt{e} in \texttt{env} is
\texttt{t,new\_env} and all of the following apply:
  \begin{itemize}
  \item \texttt{e} denotes the access of field \texttt{field\_name} on expression \texttt{e1};
  \item \texttt{t\_e1, e2} is the result of annotating \texttt{e1} in \texttt{env};
  \item \texttt{t\_e1} has the structure of a bitvector type with bitfields \texttt{bitfields};
  \item \texttt{t\_e2} has the structure of a bitvector type with bitfields \texttt{bitfields};
  \item \texttt{field\_name} is declared in \texttt{bitfields};
  \item \texttt{slices} gives the slices corresponding to the bitfield \texttt{field\_name}
    in \texttt{bitfields};
  \item \texttt{e3} denotes the slicing of the expression \texttt{e2} by the slices \texttt{slices};
  \item \texttt{t,new\_env} is the result of annotating \texttt{e3}.
  \end{itemize}

  \subsection{Examples}

  \subsection{Code}
    \VerbatimInput[firstline=\EGetBitFieldBegin, lastline=\EGetBitFieldEnd]{../Typing.ml}

  \subsection{Formally}

  \subsection{Comments}

\section{TypingRule.EGetBitFieldNested \label{sec:TypingRule.EGetBitFieldNested}}

  \subsection{Prose}
  The result of annotating the expression \texttt{e} in \texttt{env} is
\texttt{t,new\_env} and all of the following apply:
  \begin{itemize}
  \item \texttt{e} denotes the access of field \texttt{field\_name} on expression \texttt{e1};
  \item \texttt{t\_e1, e2} is the result of annotating \texttt{e1} in \texttt{env};
  \item \texttt{t\_e1} has the structure of a bitvector type with bitfields \texttt{bitfields};
  \item \texttt{t\_e2} has the structure of a bitvector type with bitfields \texttt{bitfields};
  \item \texttt{field\_name} is declared in \texttt{bitfields};
  \item \texttt{slices} gives the slices corresponding to the bitfield \texttt{field\_name} in
    \texttt{bitfields};
  \item \texttt{e3} denotes the slicing of the expression \texttt{e2} by the slices \texttt{slices};
  \item \texttt{t4, e4} is the result of annotating \texttt{e3} in \texttt{env};
  \item \texttt{bitfields'} gives the bitfields corresponding to the bitfield \texttt{field\_name}
    in \texttt{bitfields};
  \item \texttt{t} is the bitvector type with the width of \texttt{t4} and the bitfields \texttt{bitfields'}
  \item \texttt{new\_env} is \texttt{e4}.
  \end{itemize}

  \subsection{Examples}

  \subsection{Code}
    \VerbatimInput[firstline=\EGetBitFieldNestedBegin, lastline=\EGetBitFieldNestedEnd]{../Typing.ml}

  \subsection{Formally}

  \subsection{Comments}

\section{TypingRule.EGetBitFieldTyped \label{sec:TypingRule.EGetBitFieldTyped}}

  \subsection{Prose}
  The result of annotating the expression \texttt{e} in \texttt{env} is
\texttt{t,new\_env} and all of the following apply:
  \begin{itemize}
  \item \texttt{e} denotes \texttt{e1, field\_name};
  \item \texttt{t\_e1, e2} is the result of annotating \texttt{e1} in \texttt{env};
  \item \texttt{t\_e1} has the structure of a bitvector type with bitfields \texttt{bitfields};
  \item \texttt{t\_e2} has the structure of a bitvector type with bitfields \texttt{bitfields};
  \item \texttt{field\_name} is declared in \texttt{bitfields};
  \item \texttt{slices} gives the slices corresponding to the bitfield \texttt{field\_name} in
    \texttt{bitfields};
  \item \texttt{t\_e3,e3} is the result of annotating \texttt{e2,slices} in \texttt{env};
  \item \texttt{t} gives the type corresponding to the bitfield \texttt{field\_name} in \texttt{bitfields};
  \item \texttt{t\_e3} type-satisfies \texttt{t} in \texttt{env};
  \item \texttt{new\_env} is \texttt{e3}.
  \end{itemize}

  \subsection{Examples}

  \subsection{Code}
    \VerbatimInput[firstline=\EGetBitFieldTypedBegin, lastline=\EGetBitFieldTypedEnd]{../Typing.ml}

  \subsection{Formally}

  \subsection{Comments}

\section{TypingRule.EConcatEmpty \label{sec:TypingRule.EConcatEmpty}}

  \subsection{Prose}
  The result of annotating the expression \texttt{e} in \texttt{env} is
\texttt{t,new\_env} and all of the following apply:
  \begin{itemize}
  \item \texttt{e} denotes the empty concatenation;
  \item \texttt{t} is \texttt{bits(0)};
  \item \texttt{new\_env} is \texttt{e}.
  \end{itemize}

  \subsection{Examples}

  \subsection{Code}
    \VerbatimInput[firstline=\EConcatEmptyBegin, lastline=\EConcatEmptyEnd]{../Typing.ml}

  \subsection{Formally}

  \subsection{Comments}

\section{TypingRule.EConcat \label{sec:TypingRule.EConcat}}

  \subsection{Prose}
  The result of annotating the expression \texttt{e} in \texttt{env} is
\texttt{t,new\_env} and all of the following apply:
  \begin{itemize}
  \item \texttt{e} denotes the concatenation of a non-empty list of expressions \texttt{li};
  \item \texttt{ts, es} is the result of annotating \texttt{li} in \texttt{env};
  \item all elements of \texttt{ts} have the structure of a bitvector type;
  \item \texttt{w} is the sum of the widths of the bitvector types \texttt{ts};
  \item \texttt{t} is \texttt{bits(w)};
  \item \texttt{new\_env} is \texttt{es}.
  \end{itemize}

  \subsection{Examples}

  \subsection{Code}
    \VerbatimInput[firstline=\EConcatBegin, lastline=\EConcatEnd]{../Typing.ml}

  \subsection{Formally}

  \subsection{Comments}
  This aims to encompass LRM Section 7.8 R\_NYNK and R\_KCZS.

  The sum of the widths of the bitvector types~\texttt{ts} might be a symbolic
expression that is unresolvable to an integer. For example:
    \VerbatimInput{../tests/ASLTypingReference.t/TypingRule.EConcatUnresolvableToInteger.asl}


\section{TypingRule.ETuple \label{sec:TypingRule.ETuple}}

  \subsection{Prose}
  The result of annotating the expression \texttt{e} in \texttt{env} is
\texttt{t,new\_env} and all of the following apply:
  \begin{itemize}
  \item \texttt{e} denotes a tuple \texttt{li};
  \item \texttt{ts, es} is the result of annotating in \texttt{env} each expression in \texttt{li};
  \item \texttt{t} is \texttt{ts};
  \item \texttt{new\_env} is \texttt{es}.
  \end{itemize}

  \subsection{Examples}

  \subsection{Code}
    \VerbatimInput[firstline=\ETupleBegin, lastline=\ETupleEnd]{../Typing.ml}

  \subsection{Formally}

  \subsection{Comments}

\section{TypingRule.EUnknown \label{sec:TypingRule.EUnknown}}

  \subsection{Prose}
  The result of annotating the expression \texttt{e} in \texttt{env} is
\texttt{t,new\_env} and all of the following apply:
  \begin{itemize}
  \item \texttt{e} denotes an expression \texttt{UNKNOWN} of type \texttt{ty};
  \item \texttt{ty'} is the structure of \texttt{ty} in \texttt{env};
  \item \texttt{t} is \texttt{ty};
  \item \texttt{new\_env} is an expression \texttt{UNKNOWN} of type \texttt{ty'}.
  \end{itemize}

  \subsection{Examples}

  \subsection{Code}
    \VerbatimInput[firstline=\EUnknownBegin, lastline=\EUnknownEnd]{../Typing.ml}

  \subsection{Formally}

  \subsection{Comments}

\section{TypingRule.EPattern \label{sec:TypingRule.EPattern}}

  \subsection{Prose}
  The result of annotating the expression \texttt{e} in \texttt{env} is
\texttt{t,new\_env} and all of the following apply:
  \begin{itemize}
  \item \texttt{e} denotes whether the expression \texttt{e'} matches \texttt{patterns};
  \item \texttt{t\_e', e''} is the result of annotating \texttt{e'} in \texttt{env};
  \item \texttt{patterns'} is the result of annotating \texttt{patterns, t\_e'} in \texttt{env};
  \item \texttt{t} is \texttt{boolean};
  \item \texttt{new\_env} denotes whether the expression \texttt{e''} matches \texttt{patterns'}.
  \end{itemize}

  \subsection{Examples}

  \subsection{Code}
    \VerbatimInput[firstline=\EPatternBegin, lastline=\EPatternEnd]{../Typing.ml}
        
  \subsection{Formally}
       
  \subsection{Comments}

\section{TypingRule.CTC \label{sec:TypingRule.CTC}}

  \subsection{Prose}
  The result of annotating the expression \texttt{e} in \texttt{env} is
\texttt{t,new\_env} and all of the following apply:
  \begin{itemize}
  \item \texttt{e} denotes an expression \texttt{e'} and a type \texttt{t'};
  \item \texttt{t'',new\_env} is the result of annotating \texttt{e'} in \texttt{env};
  \item One of the following applies:
    \begin{itemize}
    \item All of the following apply:
      \begin{itemize}
      \item \texttt{t''} is a structural subtype of \texttt{t} in \texttt{env};
      \item \texttt{t''} is a domain subtype of \texttt{t} in \texttt{env};
      \end{itemize}
    \item All of the following apply:
      \begin{itemize}
      \item \texttt{t''} is a structural subtype of \texttt{t'} in \texttt{env};
      \item \texttt{t''} is not a domain subtype of \texttt{t'} in \texttt{env};
      \item an execution-time check that the expression evaluates to a value in the
        domain of the required type is required.
      \end{itemize}
   \item All of the following apply:
     \begin{itemize}
     \item \texttt{t''} is not a structural subtype of \texttt{t'} in \texttt{env};
     \item a ``\texttt{ConflictingTypes}'' error is raised.
     \end{itemize}
   \end{itemize}
  \end{itemize}

  \subsection{Examples}

  \subsection{Code}
  \VerbatimInput[firstline=\CTCBegin, lastline=\CTCEnd]{../Typing.ml}   

  \subsection{Formally}

  \subsection{Comments}


\chapter{Typing of Left-Hand-Side Expressions}
Annotating~\texttt{le} in an environment~\texttt{env}, assuming \texttt{t\_e}
to be the type of the corresponding right-hand-side (\texttt{annotate\_lexpr
version env le t\_e}), results in \texttt{new\_le} and one of the following
applies:
\begin{itemize}
\item TypingRule.LEDiscard (see Section~\ref{sec:TypingRule.LEDiscard}),
\item TypingRule.LELocalVar (see Section~\ref{sec:TypingRule.LELocalVar}),
\item TypingRule.LEGlobalVar (see Section~\ref{sec:TypingRule.LEGlobalVar}),
\item TypingRule.LEDestructuring (see Section~\ref{sec:TypingRule.LEDestructuring}),
\item TypingRule.LESlice (see Section~\ref{sec:TypingRule.LESlice}),
\item TypingRule.LESetArray (see Section~\ref{sec:TypingRule.LESetArray}),
\item TypingRule.LESetBadRecordField (see Section~\ref{sec:TypingRule.LESetBadRecordField}),
\item TypingRule.LESetStructuredField (see Section~\ref{sec:TypingRule.LESetStructuredField}),
\item TypingRule.LESetBadBitField (see Section~\ref{sec:TypingRule.LESetBadBitField}),
\item TypingRule.LESetBitField (see Section~\ref{sec:TypingRule.LESetBitField}),
\item TypingRule.LESetBitFieldNested (see Section~\ref{sec:TypingRule.LESetBitFieldNested}),
\item TypingRule.LESetBitFieldTyped (see Section~\ref{sec:TypingRule.LESetBitFieldTyped}),
\item TypingRule.LESetBadField (see Section~\ref{sec:TypingRule.LESetBadField}),
\item TypingRule.LESetFields (see Section~\ref{sec:TypingRule.LESetFields}),
\item TypingRule.LEConcat (see Section~\ref{sec:TypingRule.LEConcat}).
\end{itemize}

\section{TypingRule.LEDiscard \label{sec:TypingRule.LEDiscard}}

  \subsection{Prose}
   Annotating~\texttt{le} in an environment~\texttt{env}, assuming
\texttt{t\_e} to be the type of the corresponding right-hand-side
(\texttt{annotate\_lexpr version env le t\_e}), results in \texttt{new\_le} and
all of the following apply:
   \begin{itemize}
   \item \texttt{le} denotes an expression which can be discarded;
   \item \texttt{new\_le} is \texttt{le}.
   \end{itemize}

  \subsection{Examples}

  \subsection{Code}
    \VerbatimInput[firstline=\LEDiscardBegin, lastline=\LEDiscardEnd]{../Typing.ml} 

  \subsection{Formally}

  \subsection{Comments}

\section{TypingRule.LELocalVar \label{sec:TypingRule.LELocalVar}}

   \subsection{Prose}
   Annotating~\texttt{le} in an environment~\texttt{env}, assuming
\texttt{t\_e} to be the type of the corresponding right-hand-side
(\texttt{annotate\_lexpr version env le t\_e}), results in \texttt{new\_le} and
all of the following apply:
   \begin{itemize}
   \item \texttt{le} denotes a local variable \texttt{x};
   \item \texttt{x} is locally declared as a variable of type \texttt{ty} in \texttt{env};
   \item \texttt{t\_e} can be assigned to \texttt{ty};
   \item \texttt{new\_le} is \texttt{le}.
   \end{itemize}

  \subsection{Examples}

  \subsection{Code}
    \VerbatimInput[firstline=\LELocalVarBegin, lastline=\LELocalVarEnd]{../Typing.ml} 

  \subsection{Formally}

  \subsection{Comments}

\section{TypingRule.LEGlobalVar \label{sec:TypingRule.LEGlobalVar}}

  \subsection{Prose}
   Annotating~\texttt{le} in an environment~\texttt{env}, assuming
\texttt{t\_e} to be the type of the corresponding right-hand-side
(\texttt{annotate\_lexpr version env le t\_e}), results in \texttt{new\_le} and
all of the following apply:
   \begin{itemize}
   \item \texttt{le} denotes a local variable \texttt{x};
   \item \texttt{x} is globally declared as a variable of type \texttt{ty} in \texttt{env};
   \item \texttt{t\_e} can be assigned to \texttt{ty};
   \item \texttt{new\_le} is \texttt{le}.
   \end{itemize}
 
  \subsection{Examples}

  \subsection{Code}
    \VerbatimInput[firstline=\LEGlobalVarBegin, lastline=\LEGlobalVarEnd]{../Typing.ml} 

  \subsection{Formally}

  \subsection{Comments}

\section{TypingRule.LEDestructuring \label{sec:TypingRule.LEDestructuring}}

  \subsection{Prose}
   Annotating~\texttt{le} in an environment~\texttt{env}, assuming
\texttt{t\_e} to be the type of the corresponding right-hand-side
(\texttt{annotate\_lexpr version env le t\_e}), results in \texttt{new\_le} and
all of the following apply:
   \begin{itemize}
   \item \texttt{le} denotes a tuple \texttt{les};
   \item \texttt{t\_e} has the structure of a tuple type \texttt{sub\_tys};
   \item the elements of \texttt{sub\_tys} can be assigned to the type of the elements of \texttt{les};
   \item One of the following applies:
     \begin{itemize}
     \item All of the following apply:
       \begin{itemize}
       \item \texttt{les} and \texttt{sub\_tys} have the same length;
       \item \texttt{new\_le} is the result of annotating \texttt{les} with \texttt{sub\_tys} in \texttt{env}
       \end{itemize}
     \item All of the following apply:
       \begin{itemize}
       \item \texttt{les} and \texttt{sub\_tys} do not have the same length;
       \item an error ``\texttt{Bad Arity LEDestructuring}'' is raised.
       \end{itemize}
     \end{itemize}
   \end{itemize}

  \subsection{Examples}

  \subsection{Code}
    \VerbatimInput[firstline=\LEDestructuringBegin, lastline=\LEDestructuringEnd]{../Typing.ml}

  \subsection{Formally}

  \subsection{Comments}

\section{TypingRule.LESlice \label{sec:TypingRule.LESlice}}

  \subsection{Prose}
   Annotating~\texttt{le} in an environment~\texttt{env}, assuming
\texttt{t\_e} to be the type of the corresponding right-hand-side
(\texttt{annotate\_lexpr version env le t\_e}), results in \texttt{new\_le} and
all of the following apply:
   \begin{itemize}
   \item \texttt{le} denotes the slicing of a left-hand-side expression \texttt{le1} by the slices \texttt{slices};
   \item \texttt{t\_le1} is the type result of annotating the right-hand-side expression corresponding to \texttt{le1} in \texttt{env};
   \item \texttt{t\_le1} has the structure of a bitvector type;
   \item \texttt{le2} is the result of annotating \texttt{le1} in \texttt{env};
   \item \texttt{width} is the width of the slices \texttt{slices} in \texttt{env};
   \item \texttt{t} is the bitvector type of width \texttt{width};
   \item \texttt{te} can be assigned to \texttt{t};
   \item \texttt{slices2} is the result of annotating \texttt{slices} in \texttt{env};
   \item \texttt{new\_le} is the slicing of \texttt{le2} by \texttt{slices2}.
   \end{itemize}
 
  \subsection{Examples}

  \subsection{Code}
    \VerbatimInput[firstline=\LESliceBegin, lastline=\LESliceEnd]{../Typing.ml}

  \subsection{Formally}

  \subsection{Comments}

\section{TypingRule.LESetArray \label{sec:TypingRule.LESetArray}}

  \subsection{Prose}
   Annotating~\texttt{le} in an environment~\texttt{env}, assuming
\texttt{t\_e} to be the type of the corresponding right-hand-side
(\texttt{annotate\_lexpr version env le t\_e}), results in \texttt{new\_le} and
all of the following apply:
   \begin{itemize}
   \item \texttt{le} denotes the slicing of a left-hand-side expression \texttt{le1} by the slices \texttt{slices};
   \item \texttt{t\_le1} is the type result of annotating the right-hand-side expression corresponding to \texttt{le1} in \texttt{env};
   \item \texttt{t\_le1} has the structure of an array type of size \texttt{size} and item type \texttt{t};
   \item \texttt{te} can be assigned to \texttt{t};
   \item \texttt{le2} is the result of annotating \texttt{le1} in \texttt{env};
   \item One of the following applies:
     \begin{itemize}
     \item \texttt{wanted\_t\_index} is an enumeration type of name \texttt{size};
     \item \texttt{wanted\_t\_index} is the type \texttt{integer {0..size-1}};
     \end{itemize}
   \item \texttt{slices} is a single expression \texttt{e\_index};
   \item \texttt{t\_index', e\_index'} is the result of annotating \texttt{e\_index} in \texttt{env};
   \item \texttt{wanted\_t\_index} can be assigned to \texttt{t\_index'};
   \item \texttt{new\_le} is an access to array \texttt{le2} at index \texttt{e\_index'}.  
   \end{itemize}

  \subsection{Examples}

  \subsection{Code}
    \VerbatimInput[firstline=\LESetArrayBegin, lastline=\LESetArrayEnd]{../Typing.ml}

  \subsection{Formally}

  \subsection{Comments}

\section{TypingRule.LESetBadStructuredField \label{sec:TypingRule.LESetBadStructuredField}}

  \subsection{Prose}
   Annotating~\texttt{le} in an environment~\texttt{env}, assuming
\texttt{t\_e} to be the type of the corresponding right-hand-side
(\texttt{annotate\_lexpr version env le t\_e}), results in \texttt{new\_le} and
all of the following apply:
   \begin{itemize}
   \item \texttt{le} denotes the access to the field named \texttt{field} in \texttt{le1};
   \item \texttt{t\_le1} is the type result of annotating the right-hand-side expression corresponding to \texttt{le1} in \texttt{env};
   \item \texttt{le2} is the result of annotating \texttt{le1} in \texttt{env};
   \item \texttt{t\_le1} has the structure of an exception or a record type with fields \texttt{fields};
   \item \texttt{field} is not declared in \texttt{fields};
   \item an error \texttt{``Bad Field''} is raised.
   \end{itemize}

  \subsection{Examples}

  \subsection{Code}
    \VerbatimInput[firstline=\LESetBadStructuredFieldBegin, lastline=\LESetBadStructuredFieldEnd]{../Typing.ml}

  \subsection{Formally}

  \subsection{Comments}

\section{TypingRule.LESetStructuredField \label{sec:TypingRule.LESetStructuredField}}

    \subsection{Prose}
   Annotating~\texttt{le} in an environment~\texttt{env}, assuming
\texttt{t\_e} to be the type of the corresponding right-hand-side
(\texttt{annotate\_lexpr version env le t\_e}), results in \texttt{new\_le} and
all of the following apply:
   \begin{itemize}
   \item \texttt{le} denotes the access to the field named \texttt{field} in \texttt{le1};
   \item \texttt{t\_le1} is the type result of annotating the right-hand-side expression corresponding to \texttt{le1} in \texttt{env};
   \item \texttt{le2} is the result of annotating \texttt{le1} in \texttt{env};
   \item \texttt{t\_le1} has the structure of an exception or a record type with fields \texttt{fields};
   \item \texttt{field} is bound to type \texttt{t} in \texttt{fields};
   \item \texttt{t} can be assigned to \texttt{t\_e}; 
   \item \texttt{new\_le} is the access to the field \texttt{field} in \texttt{le2}.
   \end{itemize}

  \subsection{Examples}

  \subsection{Code}
    \VerbatimInput[firstline=\LESetStructuredFieldBegin, lastline=\LESetStructuredFieldEnd]{../Typing.ml}

  \subsection{Formally}

  \subsection{Comments}

\section{TypingRule.LESetBadBitField \label{sec:TypingRule.LESetBadBitField}}

    \subsection{Prose}
   Annotating~\texttt{le} in an environment~\texttt{env}, assuming
\texttt{t\_e} to be the type of the corresponding right-hand-side
(\texttt{annotate\_lexpr version env le t\_e}), results in \texttt{new\_le} and
all of the following apply:
   \begin{itemize}
   \item \texttt{le} denotes the access to the field named \texttt{field} in \texttt{le1};
   \item \texttt{t\_le1} is the type result of annotating the right-hand-side expression corresponding to \texttt{le1} in \texttt{env};
   \item \texttt{le2} is the result of annotating \texttt{le1} in \texttt{env};
   \item \texttt{t\_le1} has the structure of a bitvector with bitfields \texttt{bitfields};
   \item \texttt{field} is not declared in \texttt{bitfields};
   \item an error ``\texttt{Bad Field}'' is raised.  
   \end{itemize}

  \subsection{Examples}

  \subsection{Code}
    \VerbatimInput[firstline=\LESetBadBitFieldBegin, lastline=\LESetBadBitFieldEnd]{../Typing.ml}

  \subsection{Formally}

  \subsection{Comments}

\section{TypingRule.LESetBitField \label{sec:TypingRule.LESetBitField}}

    \subsection{Prose}
   Annotating~\texttt{le} in an environment~\texttt{env}, assuming
\texttt{t\_e} to be the type of the corresponding right-hand-side
(\texttt{annotate\_lexpr version env le t\_e}), results in \texttt{new\_le} and
all of the following apply:
   \begin{itemize}
   \item \texttt{le} denotes the access to the field named \texttt{field} in \texttt{le1};
   \item \texttt{t\_le1} is the type result of annotating the right-hand-side expression corresponding to \texttt{le1} in \texttt{env};
   \item \texttt{le2} is the result of annotating \texttt{le1} in \texttt{env};
   \item \texttt{t\_le1} has the structure of a bitvector with bitfields \texttt{bitfields};
   \item \texttt{field} is declared in \texttt{bitfields};
   \item \texttt{slices} gives the slices corresponding to the bitfield \texttt{field} in
      \texttt{bitfields};
   \item \texttt{w} is the width of \texttt{slices};
   \item \texttt{t} is the bitvector type of width \texttt{w};
   \item \texttt{t} can be assigned to \texttt{t\_e};
   \item \texttt{le2} is the slicing of \texttt{le1} by \texttt{slices};
   \item \texttt{new\_le} is the result of annotating \texttt{le2} in \texttt{env}.
   \end{itemize}

  \subsection{Examples}

  \subsection{Code}
    \VerbatimInput[firstline=\LESetBitFieldBegin, lastline=\LESetBitFieldEnd]{../Typing.ml}

  \subsection{Formally}

  \subsection{Comments}

\section{TypingRule.LESetBitFieldNested \label{sec:TypingRule.LESetBitFieldNested}}

    \subsection{Prose}
   Annotating~\texttt{le} in an environment~\texttt{env}, assuming
\texttt{t\_e} to be the type of the corresponding right-hand-side
(\texttt{annotate\_lexpr version env le t\_e}), results in \texttt{new\_le} and
all of the following apply:
   \begin{itemize}
   \item \texttt{le} denotes the access to the field named \texttt{field} in \texttt{le1};
   \item \texttt{t\_le1} is the type result of annotating the right-hand-side expression corresponding to \texttt{le1} in \texttt{env};
   \item \texttt{le2} is the result of annotating \texttt{le1} in \texttt{env};
   \item \texttt{t\_le1} has the structure of a bitvector with bitfields \texttt{bitfields};
   \item \texttt{slices} gives the slices corresponding to the bitfield \texttt{field} in
      \texttt{bitfields};
   \item \texttt{w} is the width of \texttt{slices};
   \item \texttt{bitfields'} gives the bitfields corresponding to \texttt{field} in \texttt{bitfields};
   \item \texttt{t} is the bitvector type of width \texttt{w} and bitfields \texttt{bitfields'};
   \item \texttt{t} can be assigned to \texttt{t\_e};
   \item \texttt{le2} is the slicing of \texttt{le1} by \texttt{slices};
   \item \texttt{new\_le} is the result of annotating \texttt{le2} in \texttt{env}.
   \end{itemize}

  \subsection{Examples}

  \subsection{Code}
    \VerbatimInput[firstline=\LESetBitFieldNestedBegin, lastline=\LESetBitFieldNestedEnd]{../Typing.ml}

  \subsection{Formally}

  \subsection{Comments}

\section{TypingRule.LESetBitFieldTyped \label{sec:TypingRule.LESetBitFieldTyped}}

    \subsection{Prose}
   Annotating~\texttt{le} in an environment~\texttt{env}, assuming
\texttt{t\_e} to be the type of the corresponding right-hand-side
(\texttt{annotate\_lexpr version env le t\_e}), results in \texttt{new\_le} and
all of the following apply:
   \begin{itemize}
   \item \texttt{le} denotes the access to the field named \texttt{field} in \texttt{le1};
   \item \texttt{t\_le1} is the type result of annotating the right-hand-side expression corresponding to \texttt{le1} in \texttt{env};
   \item \texttt{le2} is the result of annotating \texttt{le1} in \texttt{env};
   \item \texttt{t\_le1} has the structure of a bitvector with bitfields \texttt{bitfields};
   \item \texttt{slices} gives the slices corresponding to the bitfield \texttt{field} in
      \texttt{bitfields};
   \item \texttt{w} is the width of \texttt{slices};
   \item \texttt{t'} is the bitvector type of width \texttt{w};
   \item \texttt{t} gives the type corresponding to the bitfield \texttt{field} in
      \texttt{bitfields};
   \item \texttt{t} can be assigned to \texttt{t'};
   \item \texttt{t} can be assigned to \texttt{t\_e};
   \item \texttt{le2} is the slicing of \texttt{le1} by \texttt{slices};
   \item \texttt{new\_le} is the result of annotating \texttt{le2} in \texttt{env}.
   \end{itemize}

  \subsection{Examples}

  \subsection{Code}
    \VerbatimInput[firstline=\LESetBitFieldTypedBegin, lastline=\LESetBitFieldTypedEnd]{../Typing.ml}

  \subsection{Formally}

  \subsection{Comments}

\section{TypingRule.LESetBadField \label{sec:TypingRule.LESetBadField}}

    \subsection{Prose}
   Annotating~\texttt{le} in an environment~\texttt{env}, assuming
\texttt{t\_e} to be the type of the corresponding right-hand-side
(\texttt{annotate\_lexpr version env le t\_e}), results in \texttt{new\_le} and
all of the following apply:
   \begin{itemize}
   \item \texttt{le} denotes the access to the field named \texttt{field} in \texttt{le1};
   \item \texttt{t\_le1} is the type result of annotating the right-hand-side expression corresponding to \texttt{le1} in \texttt{env};
   \item \texttt{le2} is the result of annotating \texttt{le1} in \texttt{env};
   \item \texttt{t\_le1} does not have the structure of a record, or an exception or a bitvector type;
   \item an error ``\texttt{Conflicting Types}'' is raised.
   \end{itemize}

  \subsection{Examples}

  \subsection{Code}
    \VerbatimInput[firstline=\LESetBadFieldBegin, lastline=\LESetBadFieldEnd]{../Typing.ml}

  \subsection{Formally}

  \subsection{Comments}

\section{TypingRule.LEConcat \label{sec:TypingRule.LEConcat}}

    \subsection{Prose}
   Annotating~\texttt{le} in an environment~\texttt{env}, assuming
\texttt{t\_e} to be the type of the corresponding right-hand-side
(\texttt{annotate\_lexpr version env le t\_e}), results in \texttt{new\_le} and
all of the following apply:
     
  \subsection{Examples}

  \subsection{Code}
    \VerbatimInput[firstline=\LEConcatBegin, lastline=\LEConcatEnd]{../Typing.ml}

  \subsection{Formally}

  \subsection{Comments}


\chapter{Typing of Slices}
\texttt{annotate\_slices env slices} is the pair \texttt{(offset, length)} and one of the
following applies:
\begin{itemize}
\item TypingRule.SliceSingle (see Section~\ref{sec:TypingRule.SliceSingle}),
\item TypingRule.SliceLength (see Section~\ref{sec:TypingRule.SliceLength}),
\item TypingRule.SliceRange (see Section~\ref{sec:TypingRule.SliceRange}),
\item TypingRule.SliceStar (see Section~\ref{sec:TypingRule.SliceStar}).
\end{itemize}

\section{TypingRule.SliceSingle \label{sec:TypingRule.SliceSingle}}

    \subsection{Prose}
    All of the following apply: 
   \begin{itemize}
   \item \texttt{slices} gives an index \texttt{i};
   \item \texttt{(offset, length)} is the result of applying TypingRule.SliceLength to \texttt{i, i+:1}.
   \end{itemize}

    \subsection{Examples}

  \subsection{Code}

  \subsection{Formally}

    \subsection{Comments}
    R\_GXKG: The notation b\texttt{i} is syntactic sugar for b\texttt{i +: 1}.

\section{TypingRule.SliceLength \label{sec:TypingRule.SliceLength}}

    \subsection{Prose}
    All of the following apply:
   \begin{itemize}
   \item \texttt{slices} gives \texttt{offset} and \texttt{length}; 
   \item \texttt{t\_offset, offset'} is the result of annotating \texttt{offset} in \texttt{env};
   \item \texttt{t\_length, length'} is the result of annotating \texttt{length} in \texttt{env};
   \item \texttt{t\_offset} has the structure of an integer type;
   \item \texttt{t\_length} has the structure of an integer type;
   \item \texttt{length} is pure.
   \end{itemize}

    \subsection{Examples}

  \subsection{Code}

  \subsection{Formally}

    \subsection{Comments}

\section{TypingRule.SliceRange \label{sec:TypingRule.SliceRange}}

    \subsection{Prose}
    All of the following apply:
   \begin{itemize}
   \item \texttt{slices} gives a range \texttt{(j, i)};
   \item \texttt{pre\_length} is \texttt{i +: j-i+1};
   \item \texttt{offset, length} is the result of applying TypingRule.SliceLength to \texttt{i,pre\_length}.
   \end{itemize}

    \subsection{Examples}

  \subsection{Code}

  \subsection{Formally}

    \subsection{Comments}
    R\_GXKG: The notation b\texttt{j:i} is syntactic sugar for b\texttt{i +: j-i+1}.

\section{TypingRule.SliceStar \label{sec:TypingRule.SliceStar}}

    \subsection{Prose}
    All of the following apply:
   \begin{itemize}
   \item \texttt{slices} gives \texttt{(factor, pre\_length)};
   \item \texttt{pre\_offset} is \texttt{factor * pre\_length};
   \item \texttt{offset, length} is the result of applying TypingRule.SliceLength to \texttt{(pre\_offset, pre\_length)}.
   \end{itemize}

    \subsection{Examples}

  \subsection{Code}

  \subsection{Formally}

    \subsection{Comments}
    R\_GXQG: The notation b\texttt{i *: n} is syntactic sugar for b\texttt{i*n +: n}

\chapter{Typing of Patterns}
\texttt{annotate\_pattern loc env t p} is \texttt{new\_p} and one of the following applies:
\begin{itemize}
\item TypingRule.PAll (see Section~\ref{sec:TypingRule.PAll}),
\item TypingRule.PAny (see Section~\ref{sec:TypingRule.PAny}),
\item TypingRule.PGeq (see Section~\ref{sec:TypingRule.PGeq}),
\item TypingRule.PLeq (see Section~\ref{sec:TypingRule.PLeq}),
\item TypingRule.PNot (see Section~\ref{sec:TypingRule.PNot}),
\item TypingRule.PRange (see Section~\ref{sec:TypingRule.PRange}),
\item TypingRule.PSingle (see Section~\ref{sec:TypingRule.PSingle}),
\item TypingRule.PMask (see Section~\ref{sec:TypingRule.PMask}),
\item TypingRule.PTupleBadArity (see Section~\ref{sec:TypingRule.P}),
\item TypingRule.PTuple (see Section~\ref{sec:TypingRule.PTuple}),
\item TypingRule.PTupleConflict (see Section~\ref{sec:TypingRule.PTupleConflict}),
\end{itemize}

\section{TypingRule.PAll \label{sec:TypingRule.PAll}}

    \subsection{Prose}
    All of the following apply:
   \begin{itemize}
   \item \texttt{p} is the pattern matching everything;
   \item \texttt{new\_p} is \texttt{p}.
   \end{itemize}

    \subsection{Examples}

  \subsection{Code}

  \subsection{Formally}

    \subsection{Comments}

 \section{TypingRule.PAny\label{sec:TypingRule.PAny}}

    \subsection{Prose}
    All of the following apply:
   \begin{itemize}
   \item \texttt{p} is the pattern which matches anything in a list \texttt{li};
   \item \texttt{new\_li} is the result of mapping the result of annotating \texttt{p} in \texttt{env} onto \texttt{li};
   \item \texttt{new\_p} is the pattern which matches anything in \texttt{new\_li}.
   \end{itemize}

    \subsection{Examples}

  \subsection{Code}

  \subsection{Formally}

    \subsection{Comments}



\section{TypingRule.PGeq \label{sec:TypingRule.PGeq}}

    \subsection{Prose}
    All of the following apply:
   \begin{itemize}
   \item \texttt{p} is the pattern which matches anything greater than or equal to an expression \texttt{e};
   \item \texttt{t\_e, e'} is the result of annotating \texttt{e} in \texttt{env}; 
   \item \texttt{e'} is a compile-time constant;
   \item \texttt{t} has the structure of an integer;
   \item \texttt{t\_e} has the structure of an integer;
   \item \texttt{new\_p} is the pattern which matches anything greater than or equal to \texttt{e'}.
   \end{itemize}

    \subsection{Examples}

  \subsection{Code}

  \subsection{Formally}

    \subsection{Comments}

 \section{TypingRule.PLeq \label{sec:TypingRule.PLeq}}

    \subsection{Prose}
    All of the following apply:
   \begin{itemize}
   \item \texttt{p} is the pattern which matches anything lesser than or equal to an expression \texttt{e};
   \item \texttt{t\_e, e'} is the result of annotating \texttt{e} in \texttt{env}; 
   \item \texttt{e'} is a compile-time constant;
   \item \texttt{t} has the structure of an integer;
   \item \texttt{t\_e} has the structure of an integer;
   \item \texttt{new\_p} is the pattern which matches anything lesser than or equal to \texttt{e'}.
   \end{itemize}

    \subsection{Examples}

  \subsection{Code}

  \subsection{Formally}

    \subsection{Comments}
 
 \section{TypingRule.PNot \label{sec:TypingRule.PNot}}

    \subsection{Prose}
    All of the following apply:
   \begin{itemize}
   \item \texttt{p} is the pattern which matches the negation of a pattern \texttt{q};
   \item \texttt{new\_q} is the result of annotating \texttt{q} in \texttt{env}; 
   \item \texttt{new\_p} is pattern which matches the negation of \texttt{new\_q}.
   \end{itemize}

    \subsection{Examples}

  \subsection{Code}

  \subsection{Formally}

    \subsection{Comments}
 \section{TypingRule.PRange \label{sec:TypingRule.PRange}}

    \subsection{Prose}
    All of the following apply:
   \begin{itemize}
   \item \texttt{p} is the pattern which matches anything within the range given by
      expressions \texttt{e1} and \texttt{e2};
   \item \texttt{t\_e1, e1'} is the result of annotating \texttt{e1} in \texttt{env};
   \item \texttt{t\_e2, e2'} is the result of annotating \texttt{e2} in \texttt{env};
   \item \texttt{t} has the structure of an integer;
   \item \texttt{t\_e1} has the structure of an integer;
   \item \texttt{t\_e2} has the structure of an integer;
   \item e1' and e2' are compile-time constants; 
   \item \texttt{new\_p} is the pattern which matches anything within the range given by
      expressions \texttt{e1'} and \texttt{e2'}.
   \end{itemize}

    \subsection{Examples}

  \subsection{Code}

  \subsection{Formally}

    \subsection{Comments}
 
\section{TypingRule.PSingle \label{sec:TypingRule.PSingle}}

    \subsection{Prose}

    \subsection{Examples}

  \subsection{Code}

  \subsection{Formally}

    \subsection{Comments}
 \section{TypingRule.PMask \label{sec:TypingRule.PMask}}

    \subsection{Prose}
    All of the following apply:
   \begin{itemize}
   \item \texttt{p} is the pattern which matches a mask \texttt{m};
   \item \texttt{t} has the structure of a bitvector type;
   \item \texttt{n} is the length of mask \texttt{m};
   \item \texttt{t\_m} is the bitvector type of width \texttt{n};
   \item \texttt{t} type-satisfies \texttt{t\_m};
   \item \texttt{new\_p} is \texttt{p}.
   \end{itemize}

    \subsection{Examples}

  \subsection{Code}

  \subsection{Formally}

    \subsection{Comments}

\section{TypingRule.PTupleBadArity \label{sec:TypingRule.PTupleBadArity}}

    \subsection{Prose}
    All of the following apply:
   \begin{itemize}
   \item \texttt{p} is the pattern which matches a tuple \texttt{li};
   \item \texttt{t} has the type structure of a tuple type \texttt{ts};
   \item \texttt{ts} is a list of different size to the size of \texttt{li};
   \item an error "Bad Arity" is raised. 
   \end{itemize}

    \subsection{Examples}

  \subsection{Code}

  \subsection{Formally}

    \subsection{Comments}


 \section{TypingRule.PTuple \label{sec:TypingRule.PTuple}}

    \subsection{Prose}
    All of the following apply:
   \begin{itemize}
   \item \texttt{p} is the pattern which matches a tuple \texttt{li};
   \item \texttt{t\_struct} is the type structure of \texttt{t};
   \item \texttt{t\_struct} is a tuple \texttt{ts};
   \item \texttt{ts} is a list of same size to the size of \texttt{li};  
   \item \texttt{new\_li} is the result of annotating \texttt{li} with \texttt{ts};
   \item \texttt{new\_p} is the pattern which matches the tuple \texttt{new\_li}.
   \end{itemize}

    \subsection{Examples}

  \subsection{Code}

  \subsection{Formally}

    \subsection{Comments}


\section{TypingRule.PTupleConflict \label{sec:TypingRule.PTupleConflict}}

    \subsection{Prose}
    All of the following apply:
   \begin{itemize}
   \item \texttt{p} is the pattern which matches a tuple \texttt{li};
   \item \texttt{t\_struct} is the type structure of \texttt{t};
   \item \texttt{t\_struct} is not a tuple type;
   \item an error "Conflicting Types" is raised. 
   \end{itemize}

    \subsection{Examples}

  \subsection{Code}

  \subsection{Formally}

    \subsection{Comments}



\chapter{Typing of Local Declarations}
Annotating a local declaration~\texttt{ldi}, given a type~\texttt{ty}, in an
environment~\texttt{env} results in \texttt{new\_env, new\_ldi}
(\texttt{annotate\_local\_decl\_item}) and one of the following applies:
\begin{itemize}
\item TypingRule.LDDiscardNone (see Section~\ref{sec:TypingRule.LDDiscardNone}),
\item TypingRule.LDDiscardSome (see Section~\ref{sec:TypingRule.LDDiscardSome}),
\item TypingRule.LDVar (see Section~\ref{sec:TypingRule.LDVar}),
\item TypingRule.LDUninitialisedTypedTuple (see Section~\ref{sec:TypingRule.LDUninitialisedTypedTuple}),
\item TypingRule.LDTuple (see Section~\ref{sec:TypingRule.LDTuple}).
\end{itemize}

This aims to encompass LRM Section 7.4.2 R\_YSPM.

\section{TypingRule.LDDiscardNone \label{sec:TypingRule.LDDiscardNone}}

  \subsection{Prose}
    Annotating a local declaration~\texttt{ldi}, given a type~\texttt{ty}, in
an environment~\texttt{env} results in \texttt{new\_env, new\_ldi} and all of
the following apply:
   \begin{itemize}
   \item \texttt{ldi} is a local declaration which can be discarded;
   \item \texttt{ldi} does not specify a type;
   \item \texttt{new\_env} is \texttt{env};
   \item \texttt{new\_ldi} is \texttt{ldi}.
   \end{itemize}

  \subsection{Examples}

  \subsection{Code}
    \VerbatimInput[firstline=\LDDiscardNoneBegin, lastline=\LDDiscardNoneEnd]{../Typing.ml}

  \subsection{Formally}

  \subsection{Comments}

\section{TypingRule.LDDiscardSome \label{sec:TypingRule.LDDiscardSome}}

  \subsection{Prose}
    Annotating a local declaration~\texttt{ldi}, given a type~\texttt{ty}, in
an environment~\texttt{env} results in \texttt{new\_env, new\_ldi} and all of
the following apply:
   \begin{itemize}
   \item \texttt{ldi} is a local declaration which can be discarded;
   \item \texttt{ldi} specifies a type \texttt{t};
   \item One of the following applies:
     \begin{itemize}
     \item All of the following apply:
       \begin{itemize}
       \item \texttt{t} can be initialised with \texttt{ty} in \texttt{env};
       \item \texttt{new\_env} is \texttt{env};
       \item \texttt{new\_ldi} is \texttt{ldi}.
       \end{itemize}
     \item All of the following apply:
       \begin{itemize}
       \item \texttt{t} cannot be initialised with \texttt{ty} in \texttt{env};
       \item an error ``\texttt{Conflicting Types}'' is raised.
       \end{itemize}
     \end{itemize}
   \end{itemize}

  \subsection{Examples}

  \subsection{Code}
    \VerbatimInput[firstline=\LDDiscardSomeBegin, lastline=\LDDiscardSomeEnd]{../Typing.ml}

  \subsection{Formally}

  \subsection{Comments}

\section{TypingRule.LDVar \label{sec:TypingRule.LDVar}}

  \subsection{Prose}
    Annotating a local declaration~\texttt{ldi}, given a type~\texttt{ty}, in
an environment~\texttt{env} results in \texttt{new\_env, new\_ldi} and all of
the following apply:
   \begin{itemize}
   \item \texttt{ldi} denotes a variable \texttt{x} with an optional type \texttt{ty\_opt};
   \item \texttt{x} is not declared in \texttt{env};
   \item One of the following applies:
     \begin{itemize}
     \item All of the following apply:
       \begin{itemize}
       \item \texttt{ty\_opt} is \texttt{None};
       \item \texttt{t} is \texttt{ty}
       \end{itemize}
     \item All of the following apply:
       \begin{itemize}
       \item \texttt{ty\_opt} is \texttt{Some t};
       \item \texttt{t} can be initialized with \texttt{ty} in \texttt{env};
       \end{itemize}
     \end{itemize}
   \item \texttt{new\_env} is \texttt{env} modified so that \texttt{x} is locally declared of type \texttt{t};
   \item \texttt{new\_ldi} is the declaration of variable \texttt{x} with type \texttt{t}.
   \end{itemize}

  \subsection{Examples}

  \subsection{Code}
    \VerbatimInput[firstline=\LDVarBegin, lastline=\LDVarEnd]{../Typing.ml}

  \subsection{Formally}

  \subsection{Comments}

\section{TypingRule.LDTuple \label{sec:TypingRule.LDTuple}}

  \subsection{Prose}
    Annotating a local declaration~\texttt{ldi}, given a type~\texttt{ty}, in
an environment~\texttt{env} results in \texttt{new\_env, new\_ldi} and all of
the following apply:
  \begin{itemize}
  \item \texttt{ldi} denotes a list \texttt{ldis};
  \item \texttt{ldi} does not specify a type;
  \item \texttt{ty} has the structure of a tuple type of the same length as~\texttt{ldis};
  \item \texttt{new\_env} is \texttt{env} modified so that each element in \texttt{ldis} is annotated with the corresponding type in \texttt{ty}; 
  \item \texttt{new\_ldi} is \texttt{ldis} where each element is declared with
the corresponding type in ~\texttt{ty}.
  \end{itemize} 

  \subsection{Examples}

  \subsection{Code}
    \VerbatimInput[firstline=\LDTupleBegin, lastline=\LDTupleEnd]{../Typing.ml}

  \subsection{Formally}

  \subsection{Comments}

\section{TypingRule.LDTypedTuple \label{sec:TypingRule.LDTypedTuple}}

  \subsection{Prose}
    Annotating a local declaration~\texttt{ldi}, given a type~\texttt{ty}, in
an environment~\texttt{env} results in \texttt{new\_env, new\_ldi} and all of
the following apply:
  \begin{itemize}
  \item \texttt{ldi} denotes a list \texttt{ldis};
  \item \texttt{ldi} specifies a type~\texttt{t};
  \end{itemize}
  
  \subsection{Examples}

  \subsection{Code}
    \VerbatimInput[firstline=\LDTypedTupleBegin, lastline=\LDTypedTupleEnd]{../Typing.ml}

  \subsection{Formally}

  \subsection{Comments}

\chapter{Typing of Statements}
Annotating a statement~\texttt{s} in an environment~\texttt{env}
(\texttt{annotate\_stmt env s}) results in a statement \texttt{new\_s} and an
environment \texttt{new\_env} and one of the following applies:
\begin{itemize}
\item TypingRule.SPass (see Section~\ref{sec:TypingRule.SPass}),
\item TypingRule.SAssign (see Section~\ref{sec:TypingRule.SAssign}),
\item TypingRule.SReturnNone (see Section~\ref{sec:TypingRule.SReturnNone}),
\item TypingRule.SReturnOne (see Section~\ref{sec:TypingRule.SReturnOne}),
\item TypingRule.SReturnSome (see Section~\ref{sec:TypingRule.SReturnSome}),
\item TypingRule.SSeq (see Section~\ref{sec:TypingRule.SSeq}),
\item TypingRule.SCall (see Section~\ref{sec:TypingRule.SCall}),
\item TypingRule.SCond (see Section~\ref{sec:TypingRule.SCond}),
\item TypingRule.SCase (see Section~\ref{sec:TypingRule.SCase}),
\item TypingRule.SAssert (see Section~\ref{sec:TypingRule.SAssert}),
\item TypingRule.SWhile (see Section~\ref{sec:TypingRule.SWhile}),
\item TypingRule.SRepeat (see Section~\ref{sec:TypingRule.SRepeat}),
\item TypingRule.SFor (see Section~\ref{sec:TypingRule.SFor}),
\item TypingRule.SThrowNone (see Section~\ref{sec:TypingRule.SThrowNone}),
\item TypingRule.SThrowSome (see Section~\ref{sec:TypingRule.SThrowSome}),
\item TypingRule.STry (see Section~\ref{sec:TypingRule.STry}).
\item TypingRule.SDeclSome (see Section~\ref{sec:TypingRule.SDeclSome}),
\item TypingRule.SDeclNone (see Section~\ref{sec:TypingRule.SDeclNone}),
\end{itemize}

\section{TypingRule.SPass \label{sec:TypingRule.SPass}}

    \subsection{Prose}
Annotating statement~\texttt{s} in an environment~\texttt{env}
(\texttt{annotate\_stmt env s}) results in a statement \texttt{new\_s} and an
environment \texttt{new\_env} and all of the following apply:
    \begin{itemize}
    \item \texttt{s} is a pass statement;
    \item \texttt{new\_s} is \texttt{s};
    \item \texttt{new\_env} is \texttt{env}.
    \end{itemize}

    \subsection{Examples}

    \subsection{Code}
    \VerbatimInput[firstline=\SPassBegin, lastline=\SPassEnd]{../Typing.ml}

    \subsection{Formally}

    \subsection{Comments}

\section{TypingRule.SAssign \label{sec:TypingRule.SAssign}}

  \subsection{Prose}
Annotating statement~\texttt{s} in an environment~\texttt{env}
(\texttt{annotate\_stmt env s}) results in a statement \texttt{new\_s} and an
environment \texttt{new\_env} and all of the following apply:
   \begin{itemize}
   \item \texttt{s} is an assignment \texttt{le = re} under language version \texttt{ver};
   \item \texttt{t\_e, e1} is the result of annotating \texttt{re} in \texttt{env};
   \item \texttt{reduced} is the result of inlining a setter call in \texttt{le};
   \item One of the following applies:
     \begin{itemize}
     \item All of the following apply:
       \begin{itemize}
       \item \texttt{reduced} gives a statement \texttt{s};
       \item \texttt{new\_s} is \texttt{s};
       \item \texttt{new\_env} is \texttt{env}.
       \end{itemize}     

     \item All of the following apply:
       \begin{itemize}
       \item \texttt{reduced} does not give a statement \texttt{s};
       \item One of the following applies:
         \begin{itemize}
         \item All of the following apply:
           \begin{itemize}
           \item \texttt{ver} is ASLv1;
           \item \texttt{env1} is \texttt{env};
           \end{itemize}
         \item All of the following apply:
           \begin{itemize}
           \item \texttt{ver} is ASLv0;
	   \item \texttt{env1} is the result of annotating undeclared variables by using
	      the first assignments to such variables as declarations;
           \end{itemize}
         \end{itemize} 

       \item \texttt{le1} is the result of annotating \texttt{le} with \texttt{t\_e} in \texttt{env1};
       \item \texttt{new\_s} is the assignment \texttt{le1 = e1};
       \item \texttt{new\_env} is \texttt{env1}.
       \end{itemize}
    \end{itemize}
  \end{itemize}

  \subsection{Examples}

  \subsection{Code}
  \VerbatimInput[firstline=\SAssignBegin, lastline=\SAssignEnd]{../Typing.ml}

  \subsection{Formally}

  \subsection{Comments}

\section{TypingRule.SReturnNone \label{sec:TypingRule.SReturnNone}}

  \subsection{Prose}
Annotating statement~\texttt{s} in an environment~\texttt{env}
(\texttt{annotate\_stmt env s}) results in a statement \texttt{new\_s} and an
environment \texttt{new\_env} and all of the following apply:
   \begin{itemize}
   \item \texttt{s} is a \texttt{return} statement with no value and no return type; 
   \item \texttt{new\_s} is a \texttt{return} statement with no value;
   \item the enclosing subprogram does not have a \texttt{return} type (it is either a setter
      or a procedure);
   \item \texttt{new\_env} is \texttt{env}.
   \end{itemize}

  \subsection{Examples}

  \subsection{Code}
    \VerbatimInput[firstline=\SReturnNoneBegin, lastline=\SReturnNoneEnd]{../Typing.ml}

  \subsection{Formally}

  \subsection{Comments}
    This aims to encompass LRM Section 7.4.3 R\_FTPK.


\section{TypingRule.SReturnOne \label{sec:TypingRule.SReturnOne}}

  \subsection{Prose}
Annotating statement~\texttt{s} in an environment~\texttt{env}
(\texttt{annotate\_stmt env s}) results in a statement \texttt{new\_s} and an
environment \texttt{new\_env} and all of the following apply:
   \begin{itemize}
   \item One of the following applies:
     \begin{itemize}
     \item All of the following apply:
       \begin{itemize}
       \item \texttt{s} is a \texttt{return} statement with some value;
       \item the enclosing subprogram does not have a return type;
       \end{itemize}
     \item All of the following apply:
       \begin{itemize}
       \item \texttt{s} is a \texttt{return} statement with no value;
       \item the enclosing subprogram has a \texttt{return} type;
       \end{itemize}
     \end{itemize}
   \item An error ``\texttt{Bad Return Statement}'' is raised.
   \end{itemize}

  \subsection{Examples}

  \subsection{Code}
    \VerbatimInput[firstline=\SReturnOneBegin, lastline=\SReturnOneEnd]{../Typing.ml}

  \subsection{Formally}

  \subsection{Comments}
    This aims to encompass LRM Section 7.4.3 R\_FTPK.
    
\section{TypingRule.SReturnSome \label{sec:TypingRule.SReturnSome}}

  \subsection{Prose}
Annotating statement~\texttt{s} in an environment~\texttt{env}
(\texttt{annotate\_stmt env s}) results in a statement \texttt{new\_s} and an
environment \texttt{new\_env} and all of the following apply:
   \begin{itemize}
   \item \texttt{s} is a \texttt{return} statement with some value \texttt{e};
   \item the enclosing subprogram has a return type \texttt{t};
   \item \texttt{t\_e',e'} is the result of annotating \texttt{e} in \texttt{env};
   \item One of the following applies:
     \item All of the following apply:
       \begin{itemize}
       \item \texttt{t\_e'} type-satisfies \texttt{t};
       \item \texttt{new\_s} is a \texttt{return} statement with value \texttt{e'};
       \item \texttt{new\_env} is \texttt{env}. 
       \end{itemize}
     \item an error ``\texttt{Typing Assumption Failed}'' is raised.
   \end{itemize}

  \subsection{Examples}

  \subsection{Code}
    \VerbatimInput[firstline=\SReturnSomeBegin, lastline=\SReturnSomeEnd]{../Typing.ml}

  \subsection{Formally}

  \subsection{Comments}

\section{TypingRule.SSeq \label{sec:TypingRule.SSeq}}

  \subsection{Prose}
Annotating statement~\texttt{s} in an environment~\texttt{env}
(\texttt{annotate\_stmt env s}) results in a statement \texttt{new\_s} and an
environment \texttt{new\_env} and all of the following apply:
   \begin{itemize}
   \item \texttt{s} is a statement \texttt{s1; s2};
   \item \texttt{new\_s1, env1} is the result of annotating \texttt{s1} in \texttt{env};
   \item \texttt{new\_s2, env2} is the result of annotating \texttt{s2} in \texttt{env1};
   \item \texttt{new\_s} is a then statement over two statements \texttt{new\_s1} and \texttt{new\_s2};
   \item \texttt{new\_env} is \texttt{env2}.
   \end{itemize}

  \subsection{Examples}

  \subsection{Code}
    \VerbatimInput[firstline=\SSeqBegin, lastline=\SSeqEnd]{../Typing.ml}

  \subsection{Formally}

  \subsection{Comments}

\section{TypingRule.SCall \label{sec:TypingRule.SCall}}

    \subsection{Prose}
Annotating statement~\texttt{s} in an environment~\texttt{env}
(\texttt{annotate\_stmt env s}) results in a statement \texttt{new\_s} and an
environment \texttt{new\_env} and all of the following apply:
   \begin{itemize}
   \item \texttt{s} is a call to a subprogram named \texttt{name} with arguments \texttt{args} and parameters \texttt{eqs};
   \item \texttt{new\_name, new\_args, new\_eqs} is the result of annotating the call
      to the procedure \texttt{name} with arguments \texttt{args} and parameters
\texttt{eqs};
   \item \texttt{new\_s} is the call to a subprogram named \texttt{new\_name} with arguments
      \texttt{new\_args} and parameters \texttt{new\_eqs};
   \item \texttt{new\_env} is \texttt{env}.
   \end{itemize}

  \subsection{Examples}

  \subsection{Code}
    \VerbatimInput[firstline=\SCallBegin, lastline=\SCallEnd]{../Typing.ml}

  \subsection{Formally}

  \subsection{Comments}

\section{TypingRule.SCond \label{sec:TypingRule.SCond}}

  \subsection{Prose}
Annotating statement~\texttt{s} in an environment~\texttt{env}
(\texttt{annotate\_stmt env s}) results in a statement \texttt{new\_s} and an
environment \texttt{new\_env} and all of the following apply:
   \begin{itemize}
   \item \texttt{s} is a condition \texttt{e} with two statements \texttt{s1} and \texttt{s2};
   \item \texttt{t\_cond, e\_cond} is the result of annotating \texttt{e} in \texttt{env};
   \item One of the following applies:
     \item All of the following apply:
       \begin{itemize}
       \item \texttt{t\_cond} type-satisfies \texttt{t\_bool}; 
       \item \texttt{s1'} is the result of annotating \texttt{s1} in \texttt{env};
       \item \texttt{s2'} is the result of annotating \texttt{s2} in \texttt{env};
       \item \texttt{new\_s} is the condition \texttt{e\_cond} with two statements \texttt{s1'} and \texttt{s2'};
       \item \texttt{new\_env} is \texttt{env}.
       \end{itemize}
     \item an error ``\texttt{Typing Assumption Failed}'' is raised.
   \end{itemize}

  \subsection{Examples}

  \subsection{Code}
    \VerbatimInput[firstline=\SCondBegin, lastline=\SCondEnd]{../Typing.ml}

  \subsection{Formally}

  \subsection{Comments}
    This aims to encompass LRM Section 7.4.3 R\_NBDJ.

\section{TypingRule.SCase \label{sec:TypingRule.SCase}}

  \subsection{Prose}
Annotating statement~\texttt{s} in an environment~\texttt{env}
(\texttt{annotate\_stmt env s}) results in a statement \texttt{new\_s} and an
environment \texttt{new\_env} and all of the following apply:
   \begin{itemize}
   \item \texttt{s} is a case statement with expression \texttt{e} and cases \texttt{cases};
   \item \texttt{t\_e, e1} is the result of annotating \texttt{e} in \texttt{env};
   \item \texttt{cases1, env1} is the result of annotating each case in \texttt{cases} given \texttt{t\_e};
   \item \texttt{new\_s} is a case statement with expression \texttt{e1} and cases \texttt{cases1};
   \item \texttt{new\_env} is \texttt{env1}.
   \end{itemize}

  \subsection{Examples}

  \subsection{Code}
    \VerbatimInput[firstline=\SCaseBegin, lastline=\SCaseEnd]{../Typing.ml}

  \subsection{Formally}

  \subsection{Comments}
    This aims to encompass LRM Section 7.4.3 R\_WGSY.
  
\section{TypingRule.SAssert \label{sec:TypingRule.SAssert}}

  \subsection{Prose}
Annotating statement~\texttt{s} in an environment~\texttt{env}
(\texttt{annotate\_stmt env s}) results in a statement \texttt{new\_s} and an
environment \texttt{new\_env} and all of the following apply:
   \begin{itemize}
   \item \texttt{s} is an assert statement with expression \texttt{e};
   \item \texttt{t\_e',e'} is the result of annotating \texttt{e} in \texttt{env};
   \item One of the following applies:
     \item All of the following apply:
       \begin{itemize}
       \item \texttt{t\_e'} type-satisfies \texttt{t\_bool};  
       \item \texttt{new\_s} is an assert statement with expression \texttt{e'};
       \item \texttt{new\_env} is \texttt{env}.
       \end{itemize}
     \item an error ``\texttt{Typing Assumption Failed}'' is raised.
   \end{itemize}

  \subsection{Examples}

  \subsection{Code}
    \VerbatimInput[firstline=\SAssertBegin, lastline=\SAssertEnd]{../Typing.ml}

  \subsection{Formally}

  \subsection{Comments}
    This aims to encompass LRM Section 7.4.3 R\_JQYF

    
\section{TypingRule.SWhile \label{sec:TypingRule.SWhile}}

  \subsection{Prose}
Annotating statement~\texttt{s} in an environment~\texttt{env}
(\texttt{annotate\_stmt env s}) results in a statement \texttt{new\_s} and an
environment \texttt{new\_env} and all of the following apply:
   \begin{itemize}
   \item \texttt{s} is a \texttt{while} statement with expression \texttt{e1} and statement block \texttt{s1};
   \item \texttt{t, e2} is the result of annotating \texttt{e1} in \texttt{env};
   \item One of the following applies:
     \item All of the following apply:
       \begin{itemize}
       \item \texttt{t} type-satisfies \texttt{t\_bool}; 
       \item \texttt{s2} is the result of annotating \texttt{s1} in \texttt{env};
       \item \texttt{new\_s} is a \texttt{while} statement with expression \texttt{e2} and statement block \texttt{s2};
       \item \texttt{new\_env} is \texttt{env}.
       \end{itemize}
     \item an error ``\texttt{Typing Assumption Failed}'' is raised.
   \end{itemize}

  \subsection{Examples}

  \subsection{Code}
    \VerbatimInput[firstline=\SWhileBegin, lastline=\SWhileEnd]{../Typing.ml}

  \subsection{Formally}

  \subsection{Comments}
    This aims to encompass LRM Section 7.4.3 R\_FTVN.

\section{TypingRule.SRepeat \label{sec:TypingRule.SRepeat}}

  \subsection{Prose}
Annotating statement~\texttt{s} in an environment~\texttt{env}
(\texttt{annotate\_stmt env s}) results in a statement \texttt{new\_s} and an
environment \texttt{new\_env} and all of the following apply:
   \begin{itemize}
   \item \texttt{s} is a \texttt{repeat} statement with expression \texttt{e1} and statement block \texttt{s1};
   \item \texttt{s2} is the result of annotating \texttt{s1} in \texttt{env};
   \item \texttt{t, e2} is the result of annotating \texttt{e1} in \texttt{env};
   \item One of the following applies:
     \item All of the following apply:
       \begin{itemize}
       \item \texttt{t} type-satisfies \texttt{t\_bool}; 
       \item \texttt{new\_s} is a \texttt{repeat} statement with expression \texttt{e2} and statement block \texttt{s2};
       \item \texttt{new\_env} is \texttt{env}.
       \end{itemize}
     \item an error ``\texttt{Typing Assumption Failed}'' is raised.
   \end{itemize}

  \subsection{Examples}

  \subsection{Code}
    \VerbatimInput[firstline=\SRepeatBegin, lastline=\SRepeatEnd]{../Typing.ml}

  \subsection{Formally}

  \subsection{Comments}
    This aims to encompass LRM Section 7.4.3 R\_FTVN.

\section{TypingRule.SFor \label{sec:TypingRule.SFor}}

  \subsection{Prose}
Annotating statement~\texttt{s} in an environment~\texttt{env}
(\texttt{annotate\_stmt env s}) results in a statement \texttt{new\_s} and an
environment \texttt{new\_env} and all of the following apply:
   \begin{itemize}
   \item \texttt{s} is a \texttt{for} statement with index \texttt{id}, direction \texttt{dir}, two expressions
      \texttt{e1} and \texttt{e2} and a statement block \texttt{s'};
   \item \texttt{t1,e1'} is the result of annotating \texttt{e1} in \texttt{env};
   \item \texttt{t2,e2'} is the result of annotating \texttt{e2} in \texttt{env};
   \item One of the following applies:
     \item All of the following apply:
       \begin{itemize}
       \item \texttt{t1} has the structure of an integer type;
       \item \texttt{t2} has the structure of an integer type;
       \item \texttt{ty} is ;
       \item \texttt{s''} is the result of annotating \texttt{s'} in \texttt{env};
       \item \texttt{new\_s} is a for statement with index \texttt{id}, direction \texttt{dir}, two expressions \texttt{e1'} and \texttt{e2'} and statement \texttt{s''};
       \item \texttt{new\_env} is \texttt{env}.
       \end{itemize}
     \item an error ``\texttt{Typing Assumption Failed}'' is raised.
   \end{itemize}

  \subsection{Examples}

  \subsection{Code}
    \VerbatimInput[firstline=\SForBegin, lastline=\SForEnd]{../Typing.ml}

  \subsection{Formally}

  \subsection{Comments}
    This aims to encompass LRM Section 7.4.3 R\_VTJW.
 

\section{TypingRule.SThrowNone \label{sec:TypingRule.SThrowNone}}

  \subsection{Prose}
Annotating statement~\texttt{s} in an environment~\texttt{env}
(\texttt{annotate\_stmt env s}) results in a statement \texttt{new\_s} and an
environment \texttt{new\_env} and all of the following apply:
   \begin{itemize}
   \item \texttt{s} is a throw statement with no expression;
   \item \texttt{new\_s} is \texttt{s};
   \item \texttt{new\_env} is \texttt{env}.
   \end{itemize}

  \subsection{Examples}

  \subsection{Code}
    \VerbatimInput[firstline=\SThrowNoneBegin, lastline=\SThrowNoneEnd]{../Typing.ml}

  \subsection{Formally}

  \subsection{Comments}


\section{TypingRule.SThrowSome \label{sec:TypingRule.SThrowSome}}

  \subsection{Prose}
Annotating statement~\texttt{s} in an environment~\texttt{env}
(\texttt{annotate\_stmt env s}) results in a statement \texttt{new\_s} and an
environment \texttt{new\_env} and all of the following apply:
   \begin{itemize}
   \item \texttt{s} is a throw statement with expression \texttt{e};
   \item \texttt{t\_e,e'} is the result of annotating \texttt{e} in \texttt{env};
   \item \texttt{t\_e} has the structure of an exception type;
   \item \texttt{new\_s} is a throw statement with expression \texttt{e'} and type \texttt{t\_e};
   \item \texttt{new\_env} is \texttt{env}.
   \end{itemize}

  \subsection{Examples}

  \subsection{Code}
    \VerbatimInput[firstline=\SThrowSomeBegin, lastline=\SThrowSomeEnd]{../Typing.ml}

  \subsection{Formally}

  \subsection{Comments}
    This aims to encompass LRM Section 7.4.3 R\_NXRC.


\section{TypingRule.STry \label{sec:TypingRule.STry}}

  \subsection{Prose}
Annotating statement~\texttt{s} in an environment~\texttt{env}
(\texttt{annotate\_stmt env s}) results in a statement \texttt{new\_s} and an
environment \texttt{new\_env} and all of the following apply:
   \begin{itemize}
   \item \texttt{s} is a try statement with statement \texttt{s'}, catchers \texttt{catchers} and block \texttt{otherwise};
   \item \texttt{s''} is the result of annotating \texttt{s'} in \texttt{env};
   \item \texttt{otherwise'} is the result of annotating \texttt{otherwise} in \texttt{env};
   \item \texttt{catchers'} is the result of annotating \texttt{catchers} in \texttt{env};
   \item \texttt{new\_s} is a try statement with statement \texttt{s''}, catchers \texttt{catchers'} and block \texttt{otherwise'};
   \item \texttt{new\_env} is \texttt{env}.
   \end{itemize}

  \subsection{Examples}

  \subsection{Code}
    \VerbatimInput[firstline=\STryBegin, lastline=\STryEnd]{../Typing.ml}

  \subsection{Formally}

  \subsection{Comments}
    This aims to encompass LRM Section 7.4.3 R\_WVXS.

\section{TypingRule.SDeclSome \label{sec:TypingRule.SDeclSome}}

  \subsection{Prose}
Annotating statement~\texttt{s} in an environment~\texttt{env}
(\texttt{annotate\_stmt env s}) results in a statement \texttt{new\_s} and an
environment \texttt{new\_env} and all of the following apply:
   \begin{itemize}
   \item \texttt{s} is a declaration with local identifiers \texttt{ldi} and an expression \texttt{e};
   \item \texttt{t\_e,e'} is the result of annotating \texttt{e} in \texttt{env};
   \item \texttt{env', ldi'} is the result of declaring the local identifiers of \texttt{ldi} in \texttt{env};
   \item \texttt{new\_s} is a declaration with \texttt{ldk}, \texttt{ldi'} and an expression \texttt{e'};
   \item \texttt{new\_env} is \texttt{env'}.
   \end{itemize}

  \subsection{Examples}

  \subsection{Code}
    \VerbatimInput[firstline=\SDeclSomeBegin, lastline=\SDeclSomeEnd]{../Typing.ml}

  \subsection{Formally}

  \subsection{Comments}

\section{TypingRule.SDeclNone \label{sec:TypingRule.SDeclNone}}

  \subsection{Prose}
Annotating statement~\texttt{s} in an environment~\texttt{env}
(\texttt{annotate\_stmt env s}) results in a statement \texttt{new\_s} and an
environment \texttt{new\_env} and all of the following apply:
   \begin{itemize}
   \item \texttt{s} is a declaration statement with local identifiers \texttt{ldi} and no initial expression;
   \item \texttt{env', s'} is the result of annotating uninitialised local declarations \texttt{ldi} in \texttt{env};
   \item \texttt{new\_s} is \texttt{s'};
   \item \texttt{new\_env} is \texttt{env'}.
   \end{itemize}

  \subsection{Examples}

  \subsection{Code}
    \VerbatimInput[firstline=\SDeclNoneBegin, lastline=\SDeclNoneEnd]{../Typing.ml}

  \subsection{Formally}

  \subsection{Comments}

\chapter{Typing of Blocks}

\section{TypingRule.Block \label{sec:TypingRule.Block}}
    
  \subsection{Prose}
    \texttt{annotate\_block env return\_type s} is the result of annotating the
    statement \texttt{s} in \texttt{env}.
   
  \subsection{Example: TypingRule.Block0.asl}
    \VerbatimInput{../tests/ASLTypingReference.t/TypingRule.Block0.asl}

  \subsection{Code}
    \VerbatimInput[firstline=\BlockBegin, lastline=\BlockEnd]{../Typing.ml}

  \subsection{Formally}

  \subsection{Comments}
    A local identifier declared with var, let or constant is in scope
from the point immediately after its declaration until the end of the
immediately enclosing block.

    From that follows that we can discard the environment at the end of
an enclosing block.
 

\chapter{Typing of Catchers}
  \texttt{annotate\_catchers env return\_type (name\_opt, ty, stmt)} is \texttt{(name\_opt, ty,
  new\_stmt)} and one of the following applies:
\begin{itemize}
\item TypingRule.CatcherNone (see Section~\ref{sec:TypingRule.CatcherNone}),
\item TypingRule.CatcherSome (see Section~\ref{sec:TypingRule.CatcherSome}).
\end{itemize}
   
\section{TypingRule.CatcherNone \label{sec:TypingRule.CatcherNone}}

  \subsection{Prose}
    All of the following apply:
   \begin{itemize}
   \item \texttt{ty} has the structure of an exception type;
   \item \texttt{name\_opt} gives no name;
   \item \texttt{env'} is \texttt{env};
   \item \texttt{new\_stmt} is the result of annotating \texttt{stmt} in \texttt{env'} with \texttt{return\_type}.
   \end{itemize}

  \subsection{Examples}

  \subsection{Code}

  \subsection{Formally}
    \VerbatimInput[firstline=\CatcherNoneBegin, lastline=\CatcherNoneEnd]{../Typing.ml}

  \subsection{Comments}
    This aims to encompass LRM Section 7.4.3 R\_SDJK.

\section{TypingRule.CatcherSome \label{sec:TypingRule.CatcherSome}}

  \subsection{Prose}
    All of the following apply:
   \begin{itemize}
   \item \texttt{ty} has the structure of an exception type;
   \item \texttt{name\_opt} gives a name \texttt{name};
   \item \texttt{name} is not already declared in \texttt{env};
   \item \texttt{name} is annotated with \texttt{ty} in \texttt{env};
   \item \texttt{env'} is \texttt{env} modified to have \texttt{name} locally declared as immutable of type \texttt{ty};
   \item \texttt{new\_stmt} is the result of annotating \texttt{stmt} in \texttt{env'} with \texttt{return\_type}.
   \end{itemize}

  \subsection{Examples}

  \subsection{Code}
    \VerbatimInput[firstline=\CatcherSomeBegin, lastline=\CatcherSomeEnd]{../Typing.ml}

  \subsection{Formally}

  \subsection{Comments}
    This aims to encompass LRM Section 7.4.3 R\_SDJK.

\chapter{Typing of Function Calls}
\texttt{annotate\_call loc en name args eqs call\_type} annotates the call to subprogram
\texttt{name} with arguments \texttt{args} and parameters \texttt{eqs}. Formally, \texttt{annotate\_call loc
en name args eqs call\_type} is \texttt{(name1, args, eqs2, ret\_ty1)} or an error is
raised and one of the following applies:
\begin{itemize}
\item TypingRule.FCallBadArity (see Section~\ref{sec:TypingRule.FCallBadArity}),
\item TypingRule.FCallGetter (see Section~\ref{sec:TypingRule.FCallGetter}),
\item TypingRule.FCallSetter (see Section~\ref{sec:TypingRule.FCallSetter}),
\item TypingRule.FCallMismatch (see Section~\ref{sec:TypingRule.FCallMismatch}).
\end{itemize}

\section{TypingRule.FCallBadArity \label{sec:TypingRule.FCallBadArity}}

    \subsection{Prose}
    All of the following apply:
   \begin{itemize}
   \item \texttt{caller\_arg\_types, args1} is the result of annotating \texttt{args} in \texttt{env};
   \item \texttt{name} is bound in \texttt{env} to a subprogram with a unique name \texttt{name1}
      whose argument types \texttt{callee\_arg\_types} type-clash
      \texttt{caller\_arg\_types} and whose return type is \texttt{ret\_ty};
   \item \texttt{eqs1} is the list made of both \texttt{eqs} and \texttt{extra\_nargs};
   \item the lists \texttt{callee\_arg\_types} and \texttt{args1} do not have the same length;
   \item an error "Bad Arity" is raised.
   \end{itemize}

    \subsection{Examples}


  \subsection{Code}

  \subsection{Formally}

    \subsection{Comments}

 \section{TypingRule.FCallGetter \label{sec:TypingRule.FCallGetter}}

    \subsection{Prose}
    All of the following apply:
   \begin{itemize}
   \item \texttt{caller\_arg\_types, arg1} is the result of annotating \texttt{args} in \texttt{env};
   \item \texttt{name} is bound in \texttt{env} to a subprogram with argument types
      \texttt{callee\_arg\_types};
   \item \texttt{eqs2} is \texttt{eqs1} modified to add all the type equations from the
      type-clash of \texttt{caller\_arg\_types} and \texttt{callee\_arg\_types};
   \item \texttt{call\_type} is a either a subprogram or a getter type;
   \item \texttt{ret\_ty1} is the result of renaming \texttt{ty} in \texttt{eqs2}.
   \end{itemize}

    \subsection{Examples}

  \subsection{Code}

  \subsection{Formally}

    \subsection{Comments}


\section{TypingRule.FCallSetter \label{sec:TypingRule.FCallSetter}}

    \subsection{Prose}
    All of the following apply:
   \begin{itemize}
   \item \texttt{caller\_arg\_types, arg1} is the result of annotating \texttt{args} in \texttt{env};
   \item \texttt{name} is bound in \texttt{env} to a subprogram with a unique name \texttt{name1} whose argument types \texttt{callee\_arg\_types} type-clash \texttt{caller\_arg\_types} and whose return type is \texttt{ret\_ty};
   \item \texttt{eqs1} is the list made of both \texttt{eqs} and \texttt{extra\_nargs};
   \item \texttt{eqs2} is ;
   \item \texttt{call\_type} is a setter or procedure type;
   \item \texttt{ret\_ty} is None;
   \item \texttt{ret\_ty1} is None.
   \end{itemize}

    \subsection{Examples}

  \subsection{Code}

  \subsection{Formally}

    \subsection{Comments}


\section{TypingRule.FCallMismatch \label{sec:TypingRule.FCallMismatch}}

    \subsection{Prose}
    All of the following apply:
   \begin{itemize}
   \item \texttt{caller\_arg\_types, arg1} is the result of annotating \texttt{args} in \texttt{env};
   \item \texttt{extra\_nargs, name1, callee\_arg\_types, ret\_ty} is the result of trying to
      find the name \texttt{name} with \texttt{caller\_arg\_types} in \texttt{env};
   \item \texttt{eqs1} is the list made of both \texttt{eqs} and \texttt{extra\_nargs};
   \item \texttt{eqs2} is ;
   \item \texttt{call\_type} is not a subprogram, getter, setter or procedure type;
   \item an error "Mismatched Return Value" is raised.
   \end{itemize}

    \subsection{Examples}

  \subsection{Code}

  \subsection{Formally}

    \subsection{Comments}


\chapter{Typing of Functions}
\texttt{annotate\_func loc env f} annotates the subprogram named \texttt{f} and returns \texttt{f},
\texttt{new\_body} and \texttt{name}.

\section{TypingRule.Func \label{sec:TypingRule.Func}}

  \subsection{Prose}
  All of the following apply:
 \begin{itemize}
 \item \texttt{env1} is \texttt{env} modified to have an empty local environment;
 \item \texttt{env2} is \texttt{env1} with every formal argument declared as immutable with its type;
 \item \texttt{env3} is \texttt{env2} modified to add explicit parameters;
 \item \texttt{env4} is \texttt{env3} modified to resolve dependently typed identifiers in the arguments;
 \item \texttt{env5} is \texttt{env4} modified to resolve dependently typed identifiers in the result type;
 \item \texttt{body} is the body given by \texttt{f};
 \item \texttt{new\_body} is the result of annotating \texttt{body} in \texttt{env5};
 \item \texttt{name} is the name \texttt{f} as found in \texttt{env5}.
 \end{itemize} 

  \subsection{Examples}

  \subsection{Code}

  \subsection{Formally}

    \subsection{Comments}


\end{document}
