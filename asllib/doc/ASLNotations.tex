\paragraph{Functions and Products \label{sec:FunctionsAndProducts}}

\begin{definition}[Partial Function\label{def:PartialFunction}]
  A \emph{partial function}, denoted $A \partialto B$, is a function from a \underline{subset} of $A$ to $B$.
  The domain of a partial function, denoted $\dom(A)$, is the subset of $A$ for which it defined.
  For a partial function $f$, we write $f(x) = \bot$ to denote that $x$ is not in the domain of $f$.
\end{definition}

The function with an empty domain is denoted as $\emptyfunc$.
The notation $\rightarrowfin$ stands for a function whose domain is finite.

\begin{definition}[Function Update\label{def:FunctionUpdate}]
  The notation $f[x \mapsto v]$ is the function $f$ modified so that $x$ is bound
  to $v$: $f[x \mapsto v] \triangleq \lambda\ z.\ \begin{cases}
    f(z) & \text{if } z \neq x\\
    v & \text{if } z = x \enspace .
  \end{cases}$

  The notation $\{i=1..k: a_i\mapsto b_i\}$ stands for the function formed from the corresponding input-output pairs:
  $\emptyfunc[a_1\mapsto b_1]\ldots[a_k\mapsto b_k]$.
\end{definition}

The notation $A \times B$ stands for the Cartesian product of sets $A$ and $B$: $A \times B \triangleq \{(a,b) \;|\; a \in A, b \in B\}$.

\paragraph{Sequences and Lists}
We use the notation $a..b$ as a shorthand for the sequence $a\ldots b$ and $x_{a..b}$ as a shorthand for the sequence $x_a \ldots x_b$.
%
We use the notation $p(x_{1..k})$ to denote that a predicate $p$ holds for the list of arguments $x_1,\ldots,x_k$.
%
We write $i=1..k: V(i)$, where $V(i)$ is a mathematical expression parameterized by $i$,
to denote the sequence of expressions $V(1) \ldots V(k)$.
The notation $[i=1..k: V(i)]$ stands the list $[V(1),\ldots,V(k)]$.
The notation $\seq{v\in r: f(v)}$, where $r$ is a sequence, stands for $\epsilon$ when $r=\epsilon$ and
$f(v_1) \ldots f(v_k)$ when $r=v_{1..k}$ and $f$ is a mathematical expression over $v$.
%
We use `+' to denote concatenation of lists, defined as follows: $\emptylist + L = L$, $L + \emptylist = L$, and $[i=1..k: l_i] + [j=1..n: m_j] = [l_{1..k}, m_{1..n}]$.

We write $\epsilon$ for an empty sequence,
$a_{1..k} \cdot b_{1..n} \triangleq a_1\ldots a_k b_1 \ldots b_n$ for the concatenation of two sequences,
$T^*$ for a sequence of zero or more values from $T$, and $T^+$ for a sequences of one or more values from $T$.

\paragraph{Rules}
The following notation states that when the \emph{premises} $P_{1..k}$ hold then the \emph{verdict} $V$ holds as well:
\begin{mathpar}
  \inferrule{P_1 \and \ldots \and P_k}{V}
\end{mathpar}
That is, the premises of a rule are interpreted as a conjunction.
The variables appearing in the premises and verdict are interpreted universally.
%
A set of rules is interpreted disjunctively. That is, each rule is used to determine whether its verdict
holds independent of other rules.
%
Some rules are \emph{axioms}. That is, their set of premises is empty. Premises are denoted by simply
stating the verdict.
%
Axioms can be used to conclude certain (simple) verdicts.
However, concluding more complex verdicts requires
starting from axioms to obtain verdicts, then using these verdicts as premises in non-axiom rules,
which are potentially themselves used as premises to other rules and so on until the desired verdict
is concluded.

To keep rules succinct, we will write `$\Ignore$' for a fresh universally quantified variable whose name is
irrelevant for understanding the rule, and can thus be omitted. That is, each \underline{occurrence}
of `$\Ignore$' represents a distinct variable.

\paragraph{Value Deconstruction}
We will often want to deconstruct compound values to refer to their internal values.
To achieve that, we will write $v \eqname \textit{deconstructed}(v)$ where the expression on the right
exposes the internal structure of $v$, allowing us to alias its internal components.
For example, suppose we know that $v$ is a pair of values.
Then, $v \eqname (a, b)$ allows us to alias $a$ and $b$.
Similarly, if $v$ is a non-empty list, then $v \eqname [h] + t$ deconstructs the list into the
head of the list $h$ and its tail $t$.
%
We also use $x \eqname y$ to have $x$ alias $y$, which helps match names appearing in rule premises to
names expected in the verdict of a rule.
