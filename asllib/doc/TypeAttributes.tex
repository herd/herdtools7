%%%%%%%%%%%%%%%%%%%%%%%%%%%%%%%%%%%%%%%%%%%%%%%%%%%%%%%%%%%%%%%%%%%%%%%%%%%%%%%%%%%%
\section{Basic Type Attributes\label{sec:BasicTypeAttributes}}
%%%%%%%%%%%%%%%%%%%%%%%%%%%%%%%%%%%%%%%%%%%%%%%%%%%%%%%%%%%%%%%%%%%%%%%%%%%%%%%%%%%%

This section defines some basic predicates for classifying types as well as
functions that inspect the structure of types:
\begin{itemize}
  \item Builtin singular types (\TypingRuleRef{BuiltinSingularType})
  \item Builtin aggregate types (\TypingRuleRef{BuiltinAggregateType})
  \item Builtin types (\TypingRuleRef{BuiltinSingularOrAggregate})
  \item Named types (\TypingRuleRef{NamedType})
  \item Anonymous types (\TypingRuleRef{AnonymousType})
  \item Singular types (\TypingRuleRef{SingularType})
  \item Aggregate types (\TypingRuleRef{AggregateType})
  \item Structured types (\TypingRuleRef{StructuredType})
  \item Non-primitive types (\TypingRuleRef{NonPrimitiveType})
  \item Primitive types (\TypingRuleRef{PrimitiveType})
  \item The structure of a type (\TypingRuleRef{Structure})
  \item The underlying type of a type (\TypingRuleRef{MakeAnonymous})
  \item Checked constrained integers (\TypingRuleRef{CheckConstrainedInteger})
\end{itemize}

\TypingRuleDef{BuiltinSingularType}
\hypertarget{def-isbuiltinsingular}{}
The predicate
\[
  \isbuiltinsingular(\overname{\ty}{\tty}) \;\aslto\; \Bool
\]
tests whether the type $\tty$ is a \emph{builtin singular type}.

\ProseParagraph
The \emph{builtin singular types} are:
\begin{itemize}
\item the \integertypesterm{};
\item the \realtypeterm{};
\item the \stringtypeterm{};
\item the \booleantypeterm{};
\item the \bitvectortypeterm{} (which includes \texttt{bit}, as a special case);
\item the \enumerationtypeterm{}.
\end{itemize}

\ExampleDef{Builtin singular types}
\listingref{typing-builtinsingulartype} defines variables of builtin singular types
\texttt{integer}, \texttt{real},
\texttt{boolean}, \texttt{bits(4)}, and~\texttt{bits(2)}
\ASLListing{Examples of builtin singular types}{typing-builtinsingulartype}{\typingtests/TypingRule.BuiltinSingularTypes.asl}

\ExampleDef{Builtin enumeration types}
In \listingref{typing-builtinenumerationtype},
the builtin singular type \texttt{Color} consists in two constants:
\texttt{RED} and~\texttt{BLACK}.
\ASLListing{An enumeration type}{typing-builtinenumerationtype}{\typingtests/TypingRule.EnumerationType.asl}

\FormallyParagraph
\begin{mathpar}
\inferrule{
  \vb \eqdef \astlabel(\tty) \in \{\TReal, \TString, \TBool, \TBits, \TEnum, \TInt\}
}{
  \isbuiltinsingular(\tty) \typearrow \vb
}
\end{mathpar}
\CodeSubsection{\BuiltinSingularBegin}{\BuiltinSingularEnd}{../types.ml}


\identd{PQCK} \identd{NZWT}

\TypingRuleDef{BuiltinAggregateType}
\hypertarget{def-isbuiltinaggregate}{}
The predicate
\[
  \isbuiltinaggregate(\overname{\ty}{\tty}) \;\aslto\; \Bool
\]
tests whether the type $\tty$ is a \emph{builtin aggregate type}.

\ProseParagraph
The builtin aggregate types are:
\begin{itemize}
\item tuple;
\item \texttt{array};
\item \texttt{record};
\item \texttt{exception};
\item \texttt{collection}.
\end{itemize}

\ExampleDef{Builtin Aggregate Types}
\listingref{typing-builtinaggregatetypes} provides examples of some builtin aggregate types.
\ASLListing{Builtin aggregate types}{typing-builtinaggregatetypes}{\typingtests/TypingRule.BuiltinAggregateTypes.asl}

Type \texttt{Pair} is the type of integer and boolean pairs.

Arrays are declared with indices that are either integer-typed
or enumeration-typed.  In the example above, \texttt{T} is
declared as an array with an integer-typed index (as indicated
by the used of the integer-typed constant \texttt{3}) whereas
\texttt{PointArray} is declared with the index of
\texttt{Coord}, which is an \enumerationtypeterm{}.

Arrays declared with integer-typed indices can be accessed only by integers ranging from $0$ to
the size of the array minus $1$. In the example above, $\texttt{T}$ can be accessed with
one of $0$, $1$, and $2$.

Arrays declared with an enumeration-typed index can only be accessed with labels from the corresponding
enumeration. In the example above, \texttt{PointArray} can only be accessed with one of the labels
\texttt{CX}, \texttt{CY}, and \texttt{CZ}.

The (builtin aggregate) type \verb|{ x : real, y : real, z : real }| is a record type with three fields
\texttt{x}, \texttt{y} and \texttt{z}.

\subsubsection{Builtin Aggregate Exception Types}
\listingref{typing-builtinexceptiontype} defines two (builtin aggregate) exception types:
\begin{itemize}
\item \verb|exception{}| (for \texttt{Not\_found}), which carries no value; and
\item \verb|exception { message:string }| (for \texttt{SyntaxException}), which carries a message.
\end{itemize}
Notice the similarity with record types and that the empty field list \verb|{}| can be
omitted in type declarations, as is the case for \texttt{Not\_found}.

\ASLListing{Exception types}{typing-builtinexceptiontype}{\typingtests/TypingRule.BuiltinExceptionType.asl}

\FormallyParagraph
\begin{mathpar}
\inferrule{ \vb \eqdef \astlabel(\tty) \in \{\TTuple, \TArray, \TRecord, \TException, \TCollection\} }
{ \isbuiltinaggregate(\tty) \typearrow \vb }
\end{mathpar}
\CodeSubsection{\BuiltinAggregateBegin}{\BuiltinAggregateEnd}{../types.ml}

\identd{PQCK} \identd{KNBD}

\TypingRuleDef{BuiltinSingularOrAggregate}
\hypertarget{def-isbuiltin}{}
The predicate
\[
  \isbuiltin(\overname{\ty}{\tty}) \;\aslto\; \overname{\Bool}{\vb}
\]
tests whether the type $\tty$ is a \emph{builtin type}, yielding the result in $\vb$.

\ExampleDef{Builtin Types}
In the specification
\begin{lstlisting}
type ticks of integer;
\end{lstlisting}
the type \texttt{integer} is a builtin type but the type of \texttt{ticks} is not.

\ProseParagraph
\Proseeqdef{$\vb$}{$\True$ if and only if either $\tty$ is singular or $\tty$ is builtin aggregate}.

\FormallyParagraph
\begin{mathpar}
\inferrule{
  \isbuiltinsingular(\tty) \lor \isbuiltinaggregate(\tty)
}{
  \isbuiltin(\tty) \typearrow \vbone \lor \vbtwo
}
\end{mathpar}
\CodeSubsection{\BuiltinSingularOrAggregateBegin}{\BuiltinSingularOrAggregateEnd}{../types.ml}

\TypingRuleDef{NamedType}
\hypertarget{def-isnamed}{}
The predicate
\[
  \isnamed(\overname{\ty}{\tty}) \;\aslto\; \Bool
\]
tests whether the type $\tty$ is a \emph{named type}.

\Enumerationtypesterm{}, record types, collection types, and exception types
must be declared and associated with a named type.

\ExampleDef{Named Types}
In the specification
\begin{lstlisting}
type ticks of integer;
\end{lstlisting}
\texttt{ticks} is a named type.

\ProseParagraph
A named type is a type that is declared by using the \texttt{type ... of ...} syntax.

\FormallyParagraph
\begin{mathpar}
\inferrule{
  \vb \eqdef \astlabel(\tty) = \TNamed
}{
  \isnamed(\tty) \typearrow \vb
}
\end{mathpar}
\CodeSubsection{\NamedBegin}{\NamedEnd}{../types.ml}
\identd{vmzx}

\TypingRuleDef{AnonymousType}
\hypertarget{def-isanonymous}{}
\identd{VMZX} \identi{SBCK}%
The predicate
\[
  \isanonymous(\overname{\ty}{\tty}) \;\aslto\; \Bool
\]
tests whether the type $\tty$ is an \anonymoustype.

\ExampleDef{Anonymous Types}
The well-typed specification in \listingref{AnonymousType} illustrates the use
of anonymous types as permitted by \TypingRuleRef{TypeSatisfaction}.
\ASLListing{Well-typed anonymous types}{AnonymousType}{\typingtests/TypingRule.AnonymousType.asl}

\ProseParagraph
\Anonymoustypes\ are types that are not declared using the \texttt{type ... of ...} syntax:
\integertypesterm{}, the \realtypeterm{}, the \stringtypeterm{}, the \booleantypeterm{},
\bitvectortypesterm{}, \tupletypesterm{}, and \arraytypesterm{}.

\FormallyParagraph
\begin{mathpar}
\inferrule{
  \vb \eqdef \astlabel(\tty) \neq \TNamed
}{
  \isanonymous(\tty) \typearrow \vb
}
\end{mathpar}
\CodeSubsection{\AnonymousBegin}{\AnonymousEnd}{../types.ml}

\TypingRuleDef{SingularType}
\hypertarget{def-issingular}{}
\identr{GVZK}%
The predicate
\[
  \issingular(\overname{\staticenvs}{\tenv} \aslsep \overname{\ty}{\tty}) \;\aslto\;
  \overname{\Bool}{\vb} \cup \overname{\typeerror}{\TypeErrorConfig}
\]
tests whether the type $\tty$ is a \emph{\singulartypeterm} in the \staticenvironmentterm{} $\tenv$,
yielding the result in $\vb$.
\ProseOtherwiseTypeError

\ExampleDef{Singular types}
In the following example, the types \texttt{A}, \texttt{B}, and \texttt{C} are all singular types:
\begin{lstlisting}
type A of integer;
type B of A;
type C of B;
\end{lstlisting}

\ProseParagraph
\AllApply
\begin{itemize}
  \item obtaining the \underlyingtypeterm\ of $\tty$ in the \staticenvironmentterm{} $\tenv$ yields $\vtone$\ProseOrTypeError;
  \item applying $\isbuiltinsingular$ to $\vtone$ yields $\vb$.
\end{itemize}

\FormallyParagraph
\begin{mathpar}
\inferrule{
  \makeanonymous(\tenv, \tty) \typearrow \vtone \OrTypeError\\\\
  \isbuiltinsingular(\vtone) \typearrow \vb
}{
  \issingular(\tenv, \tty) \typearrow \vb
}
\end{mathpar}
\CodeSubsection{\SingularBegin}{\SingularEnd}{../types.ml}

\TypingRuleDef{AggregateType}
\hypertarget{def-isaggregate}{}
The predicate
\[
  \isaggregate(\overname{\staticenvs}{\tenv} \aslsep \overname{\ty}{\tty}) \;\aslto\;
  \overname{\Bool}{\vb} \cup \overname{\typeerror}{\TypeErrorConfig}
\]
tests whether the type $\tty$ is an \emph{aggregate type} in the \staticenvironmentterm{} $\tenv$,
yielding the result in $\vb$.

\ExampleDef{Aggregate Types}
In the following example, the types \texttt{A}, \texttt{B}, and \texttt{C} are all aggregate types:
\begin{lstlisting}
type A of (integer, integer);
type B of A;
type C of B;
\end{lstlisting}

\ProseParagraph
\AllApply
\begin{itemize}
  \item obtaining the \underlyingtypeterm\ of $\tty$ in the environment $\tenv$ yields $\vtone$\ProseOrTypeError;
  \item $\vtone$ is a builtin aggregate.
\end{itemize}

\FormallyParagraph
\begin{mathpar}
\inferrule{
  \makeanonymous(\tenv, \tty) \typearrow \vtone \OrTypeError\\\\
  \isbuiltinaggregate(\vtone) \typearrow \vb
}{
  \isaggregate(\tenv, \tty) \typearrow \vb
}
\end{mathpar}
\identr{GVZK}
\CodeSubsection{\AggregateBegin}{\AggregateEnd}{../types.ml}

\TypingRuleDef{StructuredType}
\hypertarget{def-isstructured}{}
\hypertarget{def-structuredtype}{}
A \emph{\structuredtypeterm} is any type that consists of a list of field
identifiers that denote individual storage elements.
In ASL there are three such types --- record types, collection types, and
exception types.

The predicate
\[
  \isstructured(\overname{\ty}{\tty}) \;\aslto\; \overname{\Bool}{\vb}
\]
tests whether the type $\tty$ is a \structuredtypeterm\ and yields the result in $\vb$.

\ExampleDef{Structured Types}
In the following example, the types \texttt{SyntaxException} and \texttt{PointRecord}
are each an example of a \structuredtypeterm:
\begin{lstlisting}
type SyntaxException of exception {message: string };
type PointRecord of Record {x : real, y: real, z: real};
\end{lstlisting}

\ProseParagraph
The result $\vb$ is $\True$ if and only if $\tty$ is either a record type, a
collection type or an exception type, which is determined via the AST label of
$\tty$.

\FormallyParagraph
\begin{mathpar}
\inferrule{}{
  \isstructured(\tty) \typearrow \overname{\astlabel(\tty) \in \{\TRecord, \TException, \TCollection\}}{\vb}
}
\end{mathpar}

\identd{WGQS} \identd{QXYC}

\TypingRuleDef{NonPrimitiveType}
\hypertarget{def-isnonprimitive}{}
The predicate
\[
  \isnonprimitive(\overname{\ty}{\tty}) \;\aslto\; \overname{\Bool}{\vb}
\]
tests whether the type $\tty$ is a \emph{non-primitive type}.

\ExampleDef{Non-primitive Types}
The following types are non-primitive:

\begin{tabular}{ll}
\textbf{Type definition} & \textbf{Reason for being non-primitive}\\
\hline
\texttt{type A of integer}  & Named types are non-primitive\\
\texttt{(integer, A)}       & The second component, \texttt{A}, has non-primitive type\\
\texttt{array[6] of A}      & Element type \texttt{A} has a non-primitive type\\
\verb|record { a : A }|     & The field \texttt{a} has a non-primitive type
\end{tabular}

\ProseParagraph
\OneApplies
\begin{itemize}
  \item \AllApplyCase{singular}
  \begin{itemize}
  \item $\tty$ is a builtin singular type;
  \item $\vb$ is $\False$.
  \end{itemize}
  \item \AllApplyCase{named}
  \begin{itemize}
    \item $\tty$ is a named type;
    \item $\vb$ is $\True$.
  \end{itemize}
  \item \AllApplyCase{tuple}
  \begin{itemize}
    \item $\tty$ is a \tupletypeterm{} $\vli$;
    \item $\vb$ is $\True$ if and only if there exists a non-primitive type in $\vli$.
  \end{itemize}
  \item \AllApplyCase{array}
    \begin{itemize}
    \item $\tty$ is an array of type $\tty'$
    \item $\vb$ is $\True$ if and only if $\tty'$ is non-primitive.
    \end{itemize}
  \item \AllApplyCase{structured}
    \begin{itemize}
    \item $\tty$ is a \structuredtypeterm\ with fields $\fields$;
    \item $\vb$ is $\True$ if and only if there exists a non-primitive type in $\fields$.
    \end{itemize}
\end{itemize}

\FormallyParagraph
The cases \textsc{tuple} and \textsc{structured} below, use the notation $\vb_\vt$ to name
Boolean variables by using the types denoted by $\vt$ as a subscript.
\begin{mathpar}
\inferrule[singular]{
  \astlabel(\tty) \in \{\TReal, \TString, \TBool, \TBits, \TEnum, \TInt\}
}{
  \isnonprimitive(\tty) \typearrow \False
}
\end{mathpar}

\begin{mathpar}
\inferrule[named]{\astlabel(\tty) = \TNamed}{\isnonprimitive(\tty) \typearrow \True}
\end{mathpar}

\begin{mathpar}
\inferrule[tuple]{
  \vt \in \tys: \isnonprimitive(\vt) \typearrow \vb_{\vt}\\
  \vb \eqdef \bigvee_{\vt \in \tys} \vb_{\vt}
}{
  \isnonprimitive(\overname{\TTuple(\tys)}{\tty}) \typearrow \vb
}
\end{mathpar}

\begin{mathpar}
\inferrule[array]{
  \isnonprimitive(\tty') \typearrow \vb
}{
  \isnonprimitive(\overname{\TArray(\Ignore, \tty')}{\tty}) \typearrow \vb
}
\end{mathpar}

\begin{mathpar}
\inferrule[structured]{
  L \in \{\TRecord, \TException, \TCollection\}\\
  (\Ignore,\vt) \in \fields : \isnonprimitive(\vt) \typearrow \vb_\vt\\
  \vb \eqdef \bigvee_{\vt \in \vli} \vb_{\vt}
}{
  \isnonprimitive(\overname{L(\fields)}{\tty}) \typearrow \vb
}
\end{mathpar}
\CodeSubsection{\NonPrimitiveBegin}{\NonPrimitiveEnd}{../types.ml}
\identd{GWXK}

\TypingRuleDef{PrimitiveType}
\hypertarget{def-isprimitive}{}
The predicate
\[
  \isprimitive(\overname{\ty}{\tty}) \;\aslto\; \Bool
\]
tests whether the type $\tty$ is a \emph{primitive type}.

\ExampleDef{Primitive Types}
The following types are primitive:

\begin{tabular}{ll}
\textbf{Type definition} & \textbf{Reason for being primitive}\\
\hline
\texttt{integer} & Integers are primitive\\
\texttt{(integer, integer)} & All tuple elements are primitive\\
\texttt{array[5] of integer} & The array element type is primitive\\
\verb|record {ticks : integer}| & The single field \texttt{ticks} has a primitive type
\end{tabular}

\ProseParagraph
A type $\tty$ is primitive if it is not non-primitive.

\FormallyParagraph
\begin{mathpar}
\inferrule{
  \isnonprimitive(\tty) \typearrow \vb
}{
  \isprimitive(\tty) \typearrow \neg\vb
}
\end{mathpar}
\CodeSubsection{\PrimitiveBegin}{\PrimitiveEnd}{../types.ml}
\identd{GWXK}

\TypingRuleDef{Structure}
\RenderRelation{get_structure}
\BackupOriginalRelation{
The function
\[
  \tstruct(\overname{\staticenvs}{\tenv} \aslsep \overname{\ty}{\tty}) \aslto \overname{\ty}{\vt} \cup \overname{\typeerror}{\TypeErrorConfig}
\]
assigns a type to its \hypertarget{def-tstruct}{\emph{\structureterm}}, which is the type formed by
recursively replacing named types by their type definition in the \staticenvironmentterm{} $\tenv$.
If a named type is not associated with a declared type in $\tenv$, a \typingerrorterm{} is returned.
} % END_OF_BACKUP_RELATION

\TypingRuleRef{TypeCheckAST} ensures the absence of circular type definitions,
which ensures that \TypingRuleRef{Structure} terminates\footnote{In mathematical terms,
this ensures that \TypingRuleRef{Structure} is a proper \emph{structural induction.}}.

\ExampleDef{The Structure of a Type}
\listingref{Structure} shows examples of types and their structures.
\ASLListing{Types and their structure}{Structure}{\typingtests/TypingRule.Structure.asl}

\ProseParagraph
\OneApplies
\begin{itemize}
\item \AllApplyCase{named}
  \begin{itemize}
  \item $\tty$ is a named type $\vx$;
  \item obtaining the declared type associated with $\vx$ in the \staticenvironmentterm{} $\tenv$ yields $\vtone$\ProseOrTypeError;
  \item obtaining the structure of $\vtone$ \staticenvironmentterm{} $\tenv$ yields $\vt$\ProseOrTypeError;
  \end{itemize}
\item \AllApplyCase{builtin\_singular}
  \begin{itemize}
  \item $\tty$ is a builtin singular type;
  \item $\vt$ is $\tty$.
  \end{itemize}
\item \AllApplyCase{tuple}
  \begin{itemize}
  \item $\tty$ is a \tupletypeterm{} with list of types $\tys$;
  \item the types in $\tys$ are indexed as $\vt_i$, for $i=1..k$;
  \item obtaining the structure of each type $\vt_i$, for $i=1..k$, in $\tys$ in the \staticenvironmentterm{} $\tenv$,
  yields $\vtp_i$\ProseOrTypeError;
  \item $\vt$ is a \tupletypeterm{} with the list of types $\vtp_i$, for $i=1..k$.
  \end{itemize}
\item \AllApplyCase{array}
  \begin{itemize}
    \item $\tty$ is an array type of length $\ve$ with element type $\vt$;
    \item obtaining the structure of $\vt$ yields $\vtone$\ProseOrTypeError;
    \item $\vt$ is an array type with of length $\ve$ with element type $\vtone$.
  \end{itemize}
\item \AllApplyCase{structured}
  \begin{itemize}
  \item $\tty$ is a \structuredtypeterm\ with fields $\fields$;
  \item obtaining the structure for each type $\vt$ associated with field $\id$ yields a type $\vt_\id$\ProseOrTypeError;
  \item $\vt$ is a record, a collection or an exception, in correspondence to $\tty$, with the list of pairs $(\id, \vt\_\id)$;
  \end{itemize}
\end{itemize}

\FormallyParagraph
\begin{mathpar}
\inferrule[named]{
  \declaredtype(\tenv, \vx) \typearrow \vtone \OrTypeError\\\\
  \tstruct(\tenv, \vtone)\typearrow\vt \OrTypeError
}{
  \tstruct(\tenv, \TNamed(\vx)) \typearrow \vt
}
\end{mathpar}

\begin{mathpar}
\inferrule[builtin\_singular]{
  \isbuiltinsingular(\tty) \typearrow \True
}{
  \tstruct(\tenv, \tty) \typearrow \tty
}
\end{mathpar}

\begin{mathpar}
\inferrule[tuple]{
  \tys \eqname \vt_{1..k}\\
  i=1..k: \tstruct(\tenv, \vt_i) \typearrow \vtp_i \OrTypeError
}{
  \tstruct(\tenv, \TTuple(\tys)) \typearrow  \TTuple(i=1..k: \vtp_i)
}
\end{mathpar}

\begin{mathpar}
\inferrule[array]{
  \tstruct(\tenv, \vt) \typearrow \vtone \OrTypeError
}{
  \tstruct(\tenv, \TArray(\ve, \vt)) \typearrow \TArray(\ve, \vtone)
}
\end{mathpar}

\begin{mathpar}
\inferrule[structured]{
  L \in \{\TRecord, \TException, \TCollection\}\\\\
  (\id,\vt) \in \fields : \tstruct(\tenv, \vt) \typearrow \vt_\id \OrTypeError
}{
  \tstruct(\tenv, L(\fields)) \typearrow
 L([ (\id,\vt) \in \fields : (\id,\vt_\id) ])
}
\end{mathpar}
\CodeSubsection{\StructureBegin}{\StructureEnd}{../types.ml}
\identd{FXQV}

\TypingRuleDef{MakeAnonymous}
\RenderRelation{make_anonymous}
\BackupOriginalRelation{
The function
\[
  \makeanonymous(\overname{\staticenvs}{\tenv} \aslsep \overname{\ty}{\tty}) \aslto \overname{\ty}{\vt} \cup \overname{\typeerror}{\TypeErrorConfig}
\]
returns the \emph{\underlyingtypeterm} --- $\vt$ --- of the type $\tty$ in the \staticenvironmentterm{} $\tenv$ or a \typingerrorterm{}.
Intuitively, $\tty$ is the first non-named type that is used to define $\tty$. Unlike $\tstruct$,
$\makeanonymous$ replaces named types by their definition until the first non-named type is found but
does not recurse further.
} % END_OF_BACKUP_RELATION

\ExampleDef{The Underlying Type of a Type}
Consider the following example:
\begin{lstlisting}
type T1 of integer;
type T2 of T1;
type T3 of (integer, T2);
\end{lstlisting}

The \underlyingtypesterm\ of \texttt{integer}, \texttt{T1}, and \texttt{T2} is \texttt{integer}.

The \underlyingtypeterm{} of \texttt{(integer, T2)} and \texttt{T3} is
\texttt{(integer, T2)}.  Notice how the \underlyingtypeterm{} does not replace
\texttt{T2} with its own \underlyingtypeterm, in contrast to the \structureterm{} of
\texttt{T2}, which is \texttt{(integer, integer)}.

\ProseParagraph
\OneApplies
\begin{itemize}
  \item \AllApplyCase{named}
  \begin{itemize}
    \item $\tty$ is a named type $\vx$;
    \item obtaining the type declared for $\vx$ yields $\vtone$\ProseOrTypeError;
    \item the \underlyingtypeterm\ of $\vtone$ is $\vt$.
  \end{itemize}

  \item \AllApplyCase{non-named}
  \begin{itemize}
    \item $\tty$ is not a named type $\vx$;
    \item $\vt$ is $\tty$.
  \end{itemize}
\end{itemize}

\FormallyParagraph
\begin{mathpar}
\inferrule[named]{
  \tty \eqname \TNamed(\vx) \\
  \declaredtype(\tenv, \vx) \typearrow \vtone \OrTypeError \\\\
  \makeanonymous(\tenv, \vtone) \typearrow \vt
}{
  \makeanonymous(\tenv, \tty) \typearrow \vt
}
\and
\inferrule[non-named]{
  \astlabel(\tty) \neq \TNamed
}{
  \makeanonymous(\tenv, \tty) \typearrow \tty
}
\end{mathpar}
\CodeSubsection{\MakeAnonymousBegin}{\MakeAnonymousEnd}{../types.ml}

\TypingRuleDef{CheckConstrainedInteger}
A type is a \emph{\constrainedintegerterm} if it is either a \wellconstrainedintegertypeterm{}
or a \parameterizedintegertypeterm.

\RenderRelation{check_constrained_integer}
\BackupOriginalRelation{
The function
\[
  \checkconstrainedinteger(\overname{\staticenvs}{\tenv} \aslsep \overname{\ty}{\tty}) \aslto \{\True\} \cup \overname{\typeerror}{\TypeErrorConfig}
\]
checks whether the type $\vt$ is a \constrainedintegerterm{} type.
If so, the result is $\True$, otherwise the result is a \typingerrorterm.
} % END_OF_BACKUP_RELATION

\ExampleDef{Checking for Constrained Integers}
\listingref{check-constrained-integer}
shows examples of checking whether a type (used for the width of a bitvector type)
is a \constrainedintegerterm{} type.
\ASLListing{Checking for constrained integers}{check-constrained-integer}{\typingtests/TypingRule.CheckConstrainedInteger.asl}

\ProseParagraph
\OneApplies
\begin{itemize}
  \item \AllApplyCase{well-constrained}
  \begin{itemize}
    \item $\vt$ is a well-constrained integer;
    \item the result is $\True$.
  \end{itemize}

  \item \AllApplyCase{parameterized}
  \begin{itemize}
    \item $\vt$ is a \parameterizedintegertypeterm;
    \item the result is $\True$.
  \end{itemize}

  \item \AllApplyCase{unconstrained}
  \begin{itemize}
    \item $\vt$ is an unconstrained integer or pending constrained integer;
    \item the result is a \typingerrorterm{} indicating that a constrained integer type is expected.
  \end{itemize}

  \item \AllApplyCase{conflicting\_type}
  \begin{itemize}
    \item $\vt$ is not an integer type;
    \item the result is a \typingerrorterm{} indicating the type conflict.
  \end{itemize}
\end{itemize}

\FormallyParagraph
\begin{mathpar}
\inferrule[well-constrained]{}
{
  \checkconstrainedinteger(\tenv, \TInt(\WellConstrained(\Ignore))) \typearrow \True
}
\and
\inferrule[parameterized]{}
{
  \checkconstrainedinteger(\tenv, \TInt(\Parameterized(\Ignore))) \typearrow \True
}
\and
\inferrule[unconstrained]{
  \astlabel(\vc) = \Unconstrained \;\lor\; \astlabel(\vc) = \PendingConstrained
}{
  \checkconstrainedinteger(\tenv, \TInt(\vc)) \typearrow \TypeErrorVal{\UnexpectedType}
}
\and
\inferrule[conflicting\_type]{
  \astlabel(\vt) \neq \TInt
}{
  \checkconstrainedinteger(\tenv, \vt) \typearrow \TypeErrorVal{\UnexpectedType}
}
\end{mathpar}
\CodeSubsection{\CheckConstrainedIntegerBegin}{\CheckConstrainedIntegerEnd}{../Typing.ml}
