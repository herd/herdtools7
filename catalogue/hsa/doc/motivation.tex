We think this formalisation effort is worth pursuing, for several reasons:
\begin{enumerate}
\item documentation and pedagogical purposes;
\item test generation and conformance testing;
\item simulation purposes.
\end{enumerate}

\subsection*{Documentation and pedagogical purposes}

First, the formal model itself can act as a documentation, especially in tandem
with the {\sf herd} simulation tool that we are developing. For example, the
following link gives access to our simulator for the current state of the HSA
model:

\centerline{\url{virginia.cs.ucl.ac.uk/herd/?record=hsa}}

This should allow developers or compiler writers to get a sense of how HSA
programs can behave.

\subsection*{Test generation and conformance testing}

Second, once we have a formal model, we can automatically generate conformance
tests. Our {\sf diy} tool (see \url{diy.inria.fr}) generates systematic
families of litmus tests that can be used to test the conformance of a chip to
a model, or the compatibility of two different implementations of the same
model.

This modelling effort, and this document, are a first step towards automating
the generation of conformance tests for HSA.

\subsection*{Simulation purposes}
Our model is written in our {\tt cat} language (see~\cite{amt14} and
\url{diy.inria.fr/tst7/doc/herd.html}). This language is a domain specific
language for writing consistency models such as the ones implemented by
multiprocessors. Writing our models in this language allows us to use them as
input to our {\sf herd} tool (see~\cite{amt14}, \url{diy.inria.fr/herd} and
\url{virginia.cs.ucl.ac.uk/herd}), which thus becomes a simulator for the input
model.
